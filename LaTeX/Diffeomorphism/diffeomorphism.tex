%% sudo yum install tetex
%% sudo yum install texlive-elsarticle.noarch
%%  pdflatex diffeomorphism.tex 
%%  bibtex diffeomorphism.aux 
%%  pdflatex diffeomorphism.tex 
%%  pdflatex diffeomorphism.tex
%% pdflatex diffeomorphism.tex && bibtex diffeomorphism.aux && pdflatex diffeomorphism.tex && pdflatex diffeomorphism.tex


%\documentclass[a4paper,12pt]{}
\documentclass[final, paper=letter,5p,times,twocolumn]{elsarticle}
%\documentclass[preprint,review,8pt,times]{elsarticle}


%% or use the graphicx package for more complicated commands
%\usepackage{changebar}
\usepackage{graphicx}
\usepackage{caption}
\usepackage{subcaption}
\usepackage{multirow}
%% or use the epsfig package if you prefer to use the old commands
%% \usepackage{epsfig}

%% The amssymb package provides various useful mathematical symbols
\usepackage{tikz}
\usepackage{amsmath,amsfonts,amsthm,multicol,bm} % Math packages
%\usepackage{dsfont} % mathds{1}
%\usepackage{widetext} % 
\usepackage{listings}
\usepackage{amssymb}
\usepackage{hyperref}
%
%\usepackage[]{algorithm2e}
%% Macro
\newcommand{\ToDo}[1]{ToDo: \textbf{\textit{#1}}}
\newcommand{\CA}{computational anatomy}
%
\newtheorem{theorem}{Theorem} % reset theorem numbering for each section
\newdefinition{definition}{Definition}%

\theoremstyle{definition}
%\newtheorem{defn}[thm]{Definition} % definition numbers are dependent on theorem numbers
\newtheorem{example}[theorem]{Example} % same for example numbers

%\newtheorem*{example}{Example}
%\newtheorem{theorem}{Theorem}%
\newtheorem{corollary}{Corollary}[theorem]
\newtheorem{lemma}[theorem]{Lemma}
%\newproposition{proposition}{Proposition}%
%\newlemma{lemma}{Lemma}%
%\AtEndEnvironment{theorem}{\null\hfill\qedsymbol}%

\begin{document}
%%%%%%%%%%%%%%%%%%%%%%%%%%%%%%%%%%%%%%%%%%%%%%%%%%%%%%%%%%%%%%%%%%%%%%%%%%
%%%%%%%%%%%%%%%%%%%%%%%%%%%%%%%%%%%%%%%%%%%%%%%%%%%%%%%%%%%%%%%%%%%%%%%%%%
%%%%%%%%%%%%%%%%%%%%%%%%%%%%%%%%%%%%%%%%%%%%%%%%%%%%%%%%%%%%%%%%%%%%%%%%%%
%%%%%%%%%%%%%%%%%%%%%%%%%%%%%%%%%%%%%%%%%%%%%%%%%%%%%%%%%%%%%%%%%%%%%%%%%%
\begin{frontmatter}

\title{Diffeomorphism for image matching}

\author[label1]{Yann Cobigo\corref{cor1}}
\address[label1]{University of California, San Francisco | ucsf.edu}
%\address[label2]{Address Two\fnref{label4}}

%\cortext[cor1]{I am corresponding author}
%\fntext[label3]{I also want to inform about\ldots}
%\fntext[label4]{Small city}

\ead{yann.cobigo@ucsf.edu}
\ead[url]{https://github.com/YannCobigo}

%% \author[label5]{Author Two}
%% \address[label5]{Some University}
%% \ead{author.two@mail.com}
%% 
%% \author[label1,label5]{Author Three}
%% \ead{author.three@mail.com}

\begin{abstract}
In this report we will \dots
\end{abstract}

\begin{keyword}
%% keywords here, in the form: keyword \sep keyword
Fijee \sep electrode \sep PEM \sep CEM
%% MSC codes here, in the form: \MSC code \sep code
%% or \MSC[2008] code \sep code (2000 is the default)
\end{keyword}

\end{frontmatter}

%%%%%%%%%%%%%%%%%%%%%%%%%%%%%%%%%%%%%%%%%%%%%%%%%%%%%%%%%%%%%%%%%%%%%%%%%%
%%%%%%%%%%%%%%%%%%%%%%%%%%%%%%%%%%%%%%%%%%%%%%%%%%%%%%%%%%%%%%%%%%%%%%%%%%

\section{Mathematical preliminaries}

In this section we are laying bases on fundamental mathematical principals. Those foundations are not proven and serve as basis to understand the rest of this report. For the most part, we will use the Einstein summing convention, meaning repeating indices represent a sum.\\
We consider a map $f: X \rightarrow Y$ as an homomorphism if the algebraic structure of the set $X$, {\it e.g.} addition or multiplication, is preserved under the mapping $f$. For example $f$ is an homomorphism if $\forall a,b \in X: f(ab) = f(a)f(b)$ on $Y$. The mapping $f$ is an isomorphism if it is a bijective homomorphism. A map is said to be continuous if the inverse image of an open set $f(U)$ on $Y$ is an open set on $X$.

%%%%%%%%%%%%%%%%%%%%%%%%%%%%%%%%%%%%%%%%%%%%%%%%%%%%%%%%%%%%%%%%%%%%%%%%%%
\subsection{Vector spaces}

A vector space $V$ is defined on a corp $\mathbb{K}$ ({\it i.e.} a field), the elements of the vector space are vectors defined on the field $\mathbb{K}$. A vector space has a structure of a group. In addition, we have the addition of vectors and the multiplication with a scalar of $\mathbb{K}$. Let $\{e_{\mu}\}$ a set of vectors on $V$ linearly independent if the only solution of the equation $x^{\mu}e_{\mu} = 0$ is $x^{\mu} = 0$ for any $\mu$. A set of linearly independent vectors is a basis of $V$. Any element $\bm{v}$ of $V$ can be written in an unique way:

$$
v = v^{\mu}e_{\mu},~~~v^{\mu} \in \mathbb{K}
$$

The linear map is an example of homomorphism that preserves the vector addition and the scalar multiplication. Let $\bm{v}$ and $\bm{w}$ two vectors of the vector space $V$ and $f : V \rightarrow W$ a linear map between the vector spaces $V$ and $W$. Then for all $a$ and $b$ in the field $\mathbb{K}$, $f(a\bm{v} + b\bm{w}) = af(\bm{v}) + bf(\bm{w})$. The linear map on the vector space is an extension of the linear application on the field $\mathbb{K}$. The image, $Img~f$, of the map is define as $f(V) \subset W$. and the kernel, $Ker~f$, is defined as $\{ \bm{v} \in V / f(\bm{v}) = \bm{0}\}$. The kernel $Ker~f$ is never empty because $f(\bm{0}) = \bm{0}$. \\
Let $f : V \rightarrow W$ a linear map and $\bm{v} = v^{\mu} e_{\mu}$ a vector of $V$ defined of the basis $e_{\mu}$. We defined the dual space, $V^{*}$, through the map $f(\bm{v}) = v^{\mu} f(e_{\mu})$. The dual basis become $f(e_{\mu})$ if, for each $\mu$ the $f(e_{\mu})$ is defined. We introduce $e^{\nu} = f(e_{\mu})$ it can be completely specified by: $e^{\nu}e_{\mu} = \delta_{~\mu}^{\nu}$. $f$ is a dual vector $f = f_{\nu}e^{\nu}$. The action of the dual vector $f$ on a vector $\bm{v}$ is the inner product:

$$
f(\bm{v}) = v^{\mu} f(e_{\mu}) = v^{\mu} f_{\nu}e^{\nu}(e_{\mu}) = v^{\mu} f_{\mu}
$$

The dual vector is a linear object that maps a vector to a scalar. We usually use the notation $<,> : V^{*} \times V \rightarrow \mathbb{K}$. 

%%%%%%%%%%%%%%%%%%%%%%%%%%%%%%%%%%%%%%%%%%%%%%%%%%%%%%%%%%%%%%%%%%%%%%%%%%
\subsection{Tensors and Tensor product space}

A tensor is a multi-linear map from multiple cartesian product vector spaces to the corp $\mathbb{K}$: $\otimes : V_{1} \times V_{2} \times \cdots \times V_{n} \times V_{1}^{*} \times V_{2}^{*} \times \cdots \times V_{m}^{*} \rightarrow \mathbb{K}$. A tensor is a generalization of the linear map, that can map several vectors and dual vectors to a scalar. For example, for $(\alpha, \beta) \in V^{*} \times V^{*}$:

\begin{eqnarray*}
\begin{array}{rcl}
  \otimes : V \times V & \rightarrow & \mathbb{K} \\
  \alpha \otimes \beta(\bm{v},\bm{w}) & \rightarrow & <\alpha, \bm{v}><\beta, \bm{w}>
\end{array}
\end{eqnarray*}

Tensors are their own vector space. We can demonstrate the vector addition and the scalar multiplication rules. As any vecor space, we can build the basis vectors. In the case of $\otimes : V_{1} \times V_{2} \rightarrow \mathbb{K}$ the basis vectors can be written:

\begin{eqnarray*}
  \left( 
  \begin{array}{cccc}
    e^{0} \otimes e^{0} & e^{0} \otimes e^{1} & \cdots & e^{0} \otimes e^{n} \\ 
    e^{1} \otimes e^{0} & e^{1} \otimes e^{1} & \cdots & e^{1} \otimes e^{n} \\
    \vdots &&& \\
    e^{n} \otimes e^{0} & e^{n} \otimes e^{1} & \cdots & e^{n} \otimes e^{n} \\ 
  \end{array}
  \right)
\end{eqnarray*}

According to this definition of the basis, we can write any arbitrary tensor: $\alpha \otimes \beta = T_{\mu\nu}e^{\mu} \otimes e^{\nu}$. As an example, we can take two arbetrary vectors $(\bm{v}, \bm{w}) \in V \times V$ and one arbitrary dual vector $\bm{u} \in V^{*}$, where $\bm{v} = v^{\mu}e_{mu}$, $\bm{w} = w^{\nu}e_{nu}$ and $\bm{u} = u_{\gamma}e^{\gamma}$. And project them into $e^{1} \otimes e^{5}\otimes e_{12}$, then:

\begin{eqnarray*}
  \begin{array}{rcl}
    T_{1,5}^{~~12}e^{1} \otimes e^{5}\otimes e_{12}(\bm{v}, \bm{w}, \bm{u}) & = & T_{1,5}^{~~12} <e^{1},v^{\mu}e_{\mu}><e^{5},w^{\nu}e_{\nu}><e_{12},u_{\gamma}e^{\gamma}> \\
    & = & T_{1,5}^{~~12} v^{\mu}w^{\nu} \delta_{\mu}^{1} \delta_{\nu}^{5} \delta_{12}^{\gamma} \\
    & = & T_{1,5}^{~~12} v^{1}w^{5}u_{12} \in \mathbb{K}
  \end{array}
\end{eqnarray*}



%%%%%%%%%%%%%%%%%%%%%%%%%%%%%%%%%%%%%%%%%%%%%%%%%%%%%%%%%%%%%%%%%%%%%%%%%%
%%%%%%%%%%%%%%%%%%%%%%%%%%%%%%%%%%%%%%%%%%%%%%%%%%%%%%%%%%%%%%%%%%%%%%%%%%
\section{Topological spaces}

Topological spaces are, mathematically, a general space. The goal of topologies is to classify the equivalence of shapes. Shapes stays the same under deformation. Topology generalize the notion of distance into a notion of proximity. \\
If we can build a continuous application $f : X \rightarrow Y$ between two topologies $X$ and $Y$, and the inverse $f^{-1}$ of the application $f$ exists and is continuous, then $f$ is called an homeomorphism and the two spaces, $X$ and $Y$, are homeomorphic.
We tend to work with metric spaces that are subsets of manifold that are subsets of topological spaces. In this section we are going to describe manifold that can be seen locally as a vector space and global as a topology.

\paragraph{Definition}{Let $X$ be a set. A topological space, $\mathcal{T}$, is a set of subset $U_{i}$ of $X$. $X$ and $\varnothing$ are in $\mathcal{T}$. The intersections and unions of sets $U_{i}$ are in $\mathcal{T}$. The Topology is said connected if it contains only $\varnothing$ and $X$.\\
}

If the union of all the subset converses $X$, $\cup_{i} U_{i} = X$, the family $\{U_{i} \}$ is called a covering of $X$.

\paragraph{Definition}{A subset $A$ of a topological space $(X,\mathcal{T})$ is called a {\it neighbourhood} of $x \in X$ if and only if it exists an open $U$ such as $x \in U \subset A$. \\
}

\paragraph{Definition -- Hausdorff Space}{We say a topological space $X$ is a Hausdorff space ($T_{2}$-space) if for two disctinct points $x$ and $y$ of $X$, their respective neighbourhood sets, $U$ and $V$, are distinct. This property must be respected for all combinaisons of pairs or points in $X$.\\
}

This is the weakest separation axiom, and will consider all our spaces as Hausdorff spaces.

\paragraph{Definition}{A system of open sets $\mathcal{B}$ of a topological space $(X,\mathcal{T})$ is called a basis if and only if all the opens of $\mathcal{T}$ can be written as a an union of sets from $\mathcal{B}$. $X$ being an open set, $X = \cup_{U \subset \mathcal{B}}U$, $\mathcal{B}$ covers $X$.}

\paragraph{Example}{A basis on an Euclidian space $\mathbb{R}^{n}$ can be build from a system of open balls $\{\mathcal{B}_{x_{i}}\}$, where $\mathcal{B}_{x_{i}} = \{x \in \mathbb{R}^{n}, \exists \varepsilon \in \mathbb{R}^{n} such~as~ \|x-x_{i}\| < \varepsilon \}$. The notion of distance in an Euclidian space is $\|x\| = \sqrt{\sum_{j=1}^{n} x_{j}^{2}}$.\\

  If we restrict $n=1$, the Euclidian space is a lign, we can divide the lign into $m$ non-overlapping segments (open balls) centered on $x_{i}$. The coordinates of $x$ inside each open is $(x'_{1}, x'_{2}, \dots,x'_{m})$ and the length $\|x\| = \sqrt{(x'_{1}-x_{1})^{2} + (x'_{2}-x_{2})^{2} + \dots + (x'_{m}-x_{m})^{2}}$.\\
}

\paragraph{Product of topologies}{Let $X$ and $Y$ topological spaces, $X \times Y$ their cartesian product, such as the set of all ordered pairs $(x,y)$, with $x\in X$ and $y\in Y$, belong to $X\times Y$. The sets of forms $U\times V$, such that $U\subset X$ and $V\subset V$, forms a basis of the product of topologies $X \times Y$. }

For $x,~y,~z \in X$, we define a metric as the symmetric application $d : X \times X \rightarrow \mathbb{R}$ and $d(x,y) = d(y,x)$ such that $d(x,x) \ge 0$ and $d(x,y) + d(y,z) \ge d(x,z)$. A topological space $\mathcal{T}$ with a metric $d$ is a metric topological space. The subset of the topological space are the usual open ({\it i.e.} the open balls) $U_{i} = \{ x \in \mathcal{T} / \forall y \in U_{i} ~~ d(x,y) < \varepsilon \}$.
  
%%%%%%%%%%%%%%%%%%%%%%%%%%%%%%%%%%%%%%%%%%%%%%%%%%%%%%%%%%%%%%%%%%%%%%%%%%
\subsection{Manifolds}

A manifold $\mathcal{M}$ is a topological space that is locally homeomorphic to a subset of $\mathbb{R}^{d}$, not necessarly globaly. We usually consider the open ball $U_{i}$ homeomorphic to $\mathbb{R}^{d}$ through the mapping $\gamma_{i}$. We call a chart $\{U_{i}, \gamma_{i}\}$, and $\gamma_{i}$ represents a special set of coordinates $\{x_{\gamma_{i}}^{\mu}\}$. If the maniflod is not globaly homeomorphic to $\mathbb{R}^{d}$, then several sets of coordinates will need to be built. It can happen that overlaping open balls ofer different coordinates for the points in the intersection of the balls.

\paragraph{Definition}{A manifold $\mathcal{M}$ is a differentiable manifold, if it is provided with a family $\{U_{i}, \gamma_{i}\}$ covering the manifold. And given $U_{i}$ and $U_{j}$ such that $U_{i} \cap U_{j} \ne \varnothing$, the mapping $\psi = \gamma_{i} \circ \gamma_{j}^{-1}$, from $\gamma_{i}(U_{i} \cap U_{j})$ and $\gamma_{j}(U_{i} \cap U_{j})$, is $C^{\infty}$.}

%%%%%%%%%%%%%%%%%%%%%%%%%%%%%%%%%%%%%%%%%%%%%%%%%%%%%%%%%%%%%%%%%%%%%%%%%%
\subsection{Tangent space on a manifold}

We are going to define two important mapping on the manifold: the curves and functions. A curve $c: (a,b) \in \mathbb{R} \rightarrow \mathcal{M}$ is a mapping from an open $(a,b)$ of the real line to the manifold $\mathcal{M}$. The open $(a,b)$ can represent the entire real line. A function $f: U_{i} \in \mathcal{M} \rightarrow \mathbb{R}$ is a smooth mapping ($C^{\infty}$) from the open $U_{i}$ of $\mathcal{M}$ to the real line.\\
Now that we defined our special mapping, we are going to introduce the notion of derivative on the manifold.

\begin{eqnarray*}
\begin{array}{rcl}
  \frac{df}{dt} |_{p} & = & \frac{d}{dt}|_{t = t_{0}} (f \circ c) = \frac{d}{dt}|_{t = t_{0}} (f \circ \gamma^{-1} \circ \gamma \circ c)\\
  & = &  \left(\frac{d(\gamma \circ c)}{dt}|_{t = t_{0}}\right)^{\mu}\frac{\partial (f \circ \gamma^{-1})}{\partial x^{\mu}}|_{\gamma(p) = x_{0}^{\mu}} = \dot{x}^{\mu}\frac{\partial f }{\partial x^{\mu}}|_{p}\\
\end{array}
\end{eqnarray*}

The derivative of $f$ is equivalent to apply the operator $\dot{x}^{\mu}\partial / \partial x^{\mu}$ on $f$ at the point $p$. This operator represents the tangent vector to $\mathcal{M}$ at $p = c(t_{0})$ along the direction given by the curve $c$. All the tangent vectors at $p$ form the tangent space, $T_{p}\mathcal{M}$, of $\mathcal{M}$ ant $p$. The basis vector of $T_{p}\mathcal{M}$ are $e_{\mu} = \partial / \partial x^{\mu}$. And any vector $\bm{v}$ in $T_{p}\mathcal{M}$ can be writtent $\bm{v} = v^{\mu} e_{\mu}$. \\
%[change coordinates 151/509]

%%%%%%%%%%%%%%%%%%%%%%%%%%%%%%%%%%%%%%%%%%%%%%%%%%%%%%%%%%%%%%%%%%%%%%%%%%
%%%%%%%%%%%%%%%%%%%%%%%%%%%%%%%%%%%%%%%%%%%%%%%%%%%%%%%%%%%%%%%%%%%%%%%%%%

\section{Notes}

\ToDo{Def. geodisic} \\
\ToDo{Def. Manifold} \\
\ToDo{Def. Atlas} \\


\begin{enumerate}
\item Calculate new estimate of velocity $v^{k+1} = v^{k} - \varepsilon \nabla_{v^{k}} E$
\item Reparametrize the velocity field to be constant speed. This is done every 10 simulations.
\item Calculate for $j = N-1$ to $j = 0$ the mapping $\phi_{t_{j},T}^{k+1}$
\item Calculate for $j = 0$ to $j = N-1$ the mapping $\phi_{t_{j},0}^{k+1}$
\item Calculate for $j = 0$ to $j = N-1$ the mapping $J_{t_{j}}^{0} = I_{0} \circ \phi_{t_{j},0}^{k+1}$
\item Calculate for $j = N-1$ to $j = 0$ the mapping $J_{t_{j}}^{1} = I_{1} \circ \phi_{t_{j},T}^{k+1}$
\item Calculate for $j = 0$ to $j = N-1$ the gradient of the image $DJ_{t_{j}}^{0}$.
\item Calculate for for $j = 0$ to $j = N-1$ the Jacobian of the transformation $|D \phi_{t_{j}}|$.
\item Calculate for $j = 0$ to $j = N-1$ the gradient $\nabla_{v^{k+1}}E$ for $v^{k+1}$.
\item Calculate the norm of the new gradient $\| \nabla_{v^{k+1}E} \|$. Stop if below threshold.
\item Calculate the new Energy using Eq. (14).
\item If the number of simulations greater than specified number then Stop. Else re-iterate with $k = k +1.$
\item Denote the final velocity field as $\hat{v}$ which gives the estimate of the desired optimizer of Eq. (8).
\item Calculate the length of the path on the manifold using Eq. (15). This is the length of the geodesic and hence the estimated metric between the given images.
\end{enumerate}


%%%%%%%%%%%%%%%%%%%%%%%%%%%%%%%%%%%%%%%%%%%%%%%%%%%%%%%%%%%%%%%%%%%%%%%%%%
%%%%%%%%%%%%%%%%%%%%%%%%%%%%%%%%%%%%%%%%%%%%%%%%%%%%%%%%%%%%%%%%%%%%%%%%%%

\section{Introduction}

The goal of registration is to compute a transformation $g:\Omega \rightarrow \Omega \subset \mathbb{R}^{n}$ where is the domain ($n = 2$ for $2D$ or $n = 3$ for $3D$) on which the data (structural, orientational and functional) are defined. Let images representing this data be functions $I: \Omega \rightarrow \mathbb{R}^{d}$ defined on $\Omega$. Structural images acquired via MRI are scalar valued $d = 1$, color images such as RGB are vector valued $d = 3$ and anisotropy or orientational images acquired from diffusion tensor MRI are matrix valued. Let $I_{0}$ and $I_{1}$ denote the template and target images. The transformation of the template image $I_{0}$ under such a transformation is the pullback image defined to be $g.I_{0} = I_{0} \circ g^{-1} = I_{0}(g^{-1})$.\\
In this setting, the transformation $g$ of the domain is generated as the end-point $\phi_{1} = g(x,t=1)$ of the flow of a time-dependent velocity vector field $v_{t}: \Omega \rightarrow \mathbb{R}^{n}$, $t \in [0, 1]$ specified by the ODE $\partial_{t} g = v_{g}(g)$. This gives a path $g: \Omega \rightarrow \Omega$, $t \in [0, 1]$ in the space of transformations starting with $g(t = 0) = Id$, where $Id$ is the identity transformation $Id(x) = x$, $\forall x \in \Omega$, and terminating at the end-point $t = 1$ of the flow to the particular transformation $g = g( t = 0 ) + \int_{0}^{1} v_{g}(g) \, dt$ matching the given images.\\
The  starting  point  for  our approach to the analysis of shape and size in anatomical images is modelling anatomy  as  a  deformable  template  (Grenander  and Miller, 1998), {\it i.e.} the observed anatomical imagery $I$ is an orbit under diffeomorphic transformations $\mathcal{G}$ acting on the coordinate space of a family of exemplars. A homeomorphism on the background space $\Omega$ is a bijective (invertible) function $g: \Omega \rightarrow \Omega$, with its inverse $g^{-1}$ is continuous. Let the set of homeomorphisms acting on the background space be denoted by $Hom(\Omega)$. The homeomorphisms form a group for the usual law of composition $\psi \cdot \varphi = \varphi \circ \psi$. Moreover, for any $g \in Hom(\Omega)$ and any image $I: \Omega \rightarrow \mathbb{R}^{d}$, $g \cdot I = I \circ g^{-1}$ defines an action of $Hom(\Omega)$ on the set of all images. Let $\mathcal{G}$ be a sub-group of $Hom(\Omega)$ (for instance the set $Diff(\Omega)$ of any $\varphi \in Hom(\Omega)$ which are, with its inverse, continuously differentiable). Given a template $I_{template}$, the deformable template model of an anatomical ensemble is the orbit

$$
\mathcal{I} = \{ I_{template} \circ g^{-1} \, / \, g \in \mathcal{G} \}
$$

\begin{definition}
  The mathematical anatomy are functions $I \in \mathcal{I}, I: x \in \Omega \rightarrow I(x)$, an orbit under the geometric transformations: $\mathcal{I} = \lbrace I':I(\cdot) = I(g(\cdot)),~I \in \mathcal{I},~g \in \mathcal{G} \rbrace$.
\end{definition}

%%%%%%%%%%%%%%%%%%%%%%%%%%%%%%%%%%%%%%%%%%%%%%%%%%%%%%%%%%%%%%%%%%%%%%%%%%
%%%%%%%%%%%%%%%%%%%%%%%%%%%%%%%%%%%%%%%%%%%%%%%%%%%%%%%%%%%%%%%%%%%%%%%%%%
\section{Sobolev-space}


\subsection{Hilbert spaces}

Let $V$ be a vector-space ({\it i.e.} $x, y \in V \Rightarrow x + y \in V$, and $kx \in V$ for all $k \in \mathbb{R}$) with a scalar product denoted $\cdot$ (also called inner product). This scalar product induces a norm $\| x \|_{V} = (x \cdot x)^{1/2}$. If $V$ is complete with respect to this norm then $V$ is called a Hilbert-space. Roughly speaking, we say that a space $V$ is complete if for any sequence $\{ x_{h} \}$ of elements in $V$ converging to an element $x$; {\it i.e.} $ \| x - x_{h}\|_{V} \underset{h \rightarrow \infty}{\rightarrow} 0$, we have that $x$ also belongs to $V$.

\begin{example}
  Soient $x = (x_{1}, x_{2}, \cdots, x_{n})^{T}$ and $y = (y_{1}, y_{2}, \cdots, y_{n})^{T}$ two vectors of an Hilbert space, then $x \cdot y = x_{1}y_{1} + \cdots + x_{n}y_{n}$, and the norm is $\| x \| = \sqrt{x_{1}x_{1} + \cdots + x_{n}x_{n}}$.
\end{example}
  

\subsection{Sobolev- and Lebesgue-spaces}

Sobolev- and Lebesgue-spaces are important examples of Hilbert-spaces. Let be $\Omega$ a bounded domain in $\mathbb{R}$: $L^{2}(\Omega)$ is made of all functions $v$ that has the property so that: \\

$$
\int_{\Omega} |v|^{2} \, dx < \infty
$$

$W^{1,2}(\Omega)$ is made of all functions $u$ where $u \in L^{2}(\Omega)$ and $\partial u / \partial x_{i} \in L^{2}(\Omega)$. The number 1 is used is used to indicate that the partial derivatives of order 1 should belong to $L^{2}(\Omega)$. The number 2 refers to power of the integrand. Functions belonging to $W^{1,2}(\Omega)$ do not have to be differentiable at every point; {\it e.g.} it is enough if they are continuous with piecewise continuous partial derivatives in the domain of definition and satisfy the above conditions. $W^{1,2}_{0}(\Omega) = \{ u~is~0~on~the~boundary~\partial \Omega\}$.\\

The $W$- spaces are often called Sobolev-spaces. The space $L^{2}(\Omega)$ is called a Lebesgue-space.
The scalar product ($*$) on $W^{1,2}(\Omega)$-space is defined by:

$$
u*v = \sqrt{\int_{\Omega} u \cdot v \, dx + \int_{\Omega} \nabla u \cdot \nabla v \, dx}
$$

and hence the norm is defined as

$$
\| u \|_{W^{1,2}}= \sqrt{\int_{\Omega} u^{2} \, dx + \int_{\Omega} |\nabla u|^{2} \, dx}
$$

In this section we establish sufficient conditions for the vector fields $v(\cdot,t):\Omega \leftarrow \mathbb{R}^{3}$ to generate diffeomorphisms. We first decribe the space $V$ in wich $v(\cdot,t)$ will take values. $V$ is taken to be a Sobolev space $W_{0}^{3,2}(\Omega)$, which is the closure of $C_{0}^{\infty}(\Omega)$ with the norm

$$
\| f \|_{W_{0}^{3,2}(\Omega)} = \sqrt{\int_{\Omega} \sum_{| \alpha | \le 3 } |D^{\alpha} f(x)|^{2} \, dx }
$$

The norm of the product space $V$ is denoted $\| \cdot \|_{W_{0}^{3,2}(\Omega)}$. A mesurable function $v:[0,t] \rightarrow V$ will be called an admissible velocity field or a admisible control or control process.

%%%%%%%%%%%%%%%%%%%%%%%%%%%%%%%%%%%%%%%%%%%%%%%%%%%%%%%%%%%%%%%%%%%%%%%%%%
%%%%%%%%%%%%%%%%%%%%%%%%%%%%%%%%%%%%%%%%%%%%%%%%%%%%%%%%%%%%%%%%%%%%%%%%%%
\section{Metric space of anatomical images}

The transformations arise from an evolution in time, or a flow $g(t)$, $t \in [0, 1]$ corresponding to the transport equations from continuum mechanics at the rate $v_{g}$. The forward and inverse maps are uniquely defined according to $g^{-1}(g(x, t), t) = x$ for all $t \in [0, 1]$, $x \in X \subset \mathbb{R}^{n}$, implying the equations of flow are linked according to

\begin{eqnarray}
\left \lbrace
\begin{array}{rcl}
\frac{\partial}{\partial t}g(x,t) &=& v(g(x,t),t) \\
\frac{\partial}{\partial t}g^{-1}(y,t) &=& -Dg^{-1}(y,t)v(y,t) \\
g(0) &=& g^{-1}(0) = Id \\
\end{array}
\right .
\label{Metric}
\end{eqnarray}

$Id$ the identity map, the Jacobian operator giving the $d \times d$ matrix for $\mathbb{R}^{d}$-valued functions $(Df)_{ij} = (\frac{\partial f_{i}}{\partial x_{j}})$ and the $d \times 1$ row vector $Df = (\frac{\partial f}{\partial x_{1}}, \cdots, \frac{\partial f}{\partial x_{n}} )$ for scalar valued functions defined on $d$-dimensional domain. The flow equations are depicted in panel 1 of Figure~\ref{Euler-Lagrange-description}~\cite{pmid12117763}.\\
The goal is to find the velocity field which, after integration of (\ref{Metric}), gives the diffeomorphism $g$. The second equation can be understood as a symetric speed rate:

\begin{eqnarray*}
\frac{\partial}{\partial t}g^{-1}(y,t) & = & \frac{\partial g^{-1}(y,t)}{\partial g}\frac{\partial g}{\partial t} \\
& = & - D_{g} g^{-1}(y,t) (v_{g}(y,t))
\end{eqnarray*}

Bijective flows go in opposite directions.

\begin{definition}
We define the group of transformations $\mathcal{G}$ to be diffeomorphisms $g(1) \in \mathcal{G}: x \rightarrow g(x, 1) \in \Omega$, $1-1$ (one-to-one, bijective), onto, with continuous inverse and differentiable solutions of Equation~(\ref{Metric}), with $v(t)$, $t \in [0, 1]$ sufficiently smooth and vanishing at the boundary of $\Omega$ ($\partial \Omega$) for each $t$ to generate smooth solutions $g(1)$, $g^{-1}(1)$
\end{definition}

\subsection{Least action principal}

The variational least action principal allow physicists to find the Euler-Lagrange motion equation

$$
S = \int_{q_{0}}^{q_{1}} \mathcal{L}(q, \dot{q}) \, dt
$$

Where $S$ is the action, $\mathcal{L}$ is the Lagrangian. Applying a infinitesimal variation on the Lagrangian.

$$
\delta S =  \int_{q_{0}}^{q_{1}} \frac{\partial\mathcal{L}(q, \dot{q})}{\partial q} \delta q + \frac{\partial\mathcal{L}(q, \dot{q})}{\partial \dot{q}} \delta \dot{q} \, dt = 0
$$

Using the linearity of the integration

\begin{eqnarray*}
    \frac{\partial\mathcal{L}(q, \dot{q})}{\partial \dot{q}} \delta \dot{q} & = & \frac{\partial\mathcal{L}(q, \dot{q})}{\partial \dot{q}} \frac{\partial }{\partial t} \delta q \\
\end{eqnarray*}


and derivation operator and the part integral

\begin{eqnarray*}
    \int_{q_{0}}^{q_{1}} \frac{\partial\mathcal{L}(q, \dot{q})}{\partial \dot{q}} \frac{\partial }{\partial t} \delta q  \, dt & = & \left \lbrack \frac{\partial\mathcal{L}(q, \dot{q})}{\partial \dot{q}} \delta q \right \rbrack_{q_{0}}^{q_{1}} \\
    & - &  \int_{q_{0}}^{q_{1}} \frac{d}{dt} \frac{\partial\mathcal{L}(q, \dot{q})}{\partial \dot{q}} \delta q \, dt
\end{eqnarray*}

The backet expression is zero because of the null variation at the starting and ending point of the path. We can write the Euler-Lagrange equation of motion:

$$
\frac{d}{dt} \frac{\partial\mathcal{L}(q, \dot{q})}{\partial \dot{q}} - \frac{\partial\mathcal{L}(q, \dot{q})}{\partial q} = 0 
\label{Euler_Lagrange}
$$

Using the Mecanics lagragien $\mathcal{L} = \mathcal{E}_{c} - \mathcal{E}_{p}$, where $\mathcal{E}_{c}$ is the kinetics energy and $\mathcal{E}_{p}$ is the potential energy, we recover the Newton equation of motion.  


\subsection{Least action principal in anatomical imaging}


%%\begin{figure}[htbp]
%%   \begin{center}
%%      \includegraphics[width=8cm,height=6cm, angle=0]{images/pmid12117763.png}\hfill
%%   \end{center}
%%   \caption{Panel 1 (left) shows the Lagrangian description of the flow; panel 2 (right) shows the variation of the group flow element $g(\cdot)$ by $\eta(\cdot)$~\cite{pmid12117763}.}
%%\label{Euler-Lagrange-description}
%%\end{figure}


One principal aspect of \CA{} is to define the metric distance between anatomies through the mappings between them. We define distance between them via the geodesic length of the flows $g(\cdot): [0, 1] \rightarrow \mathcal{G}$ which connects them. The geodesic length is defined through the least action principal 

$$
S_{g} = \int_{0}^{1} \|v(t)\|_{L}^{2} \,dt = \int_{0}^{1} E_{v}(t) \,dt
$$ 

$v(t) = \partial_{t} g$ with $\| \cdot \|_{L}$ a proper norm on the vector fields on $\Omega$ ({\it i.e.}, a Sobolev space with $L$ a differential operator). Avoid confusion between operator $L$ and the Lebesgue space $L^{2}$.\\

Investigators have used linear differential operators $L$ operating on the vector fields to enforce smoothness on the maps and to define the finite norm, generally differentiating only in space and constructed from the Laplacian and its powers. $L$ is a $d \times d$ matrix of differential operators $L = (L_{ij})$ on d $\mathbb{R}^{d}$ valued vector fields of the form $(Lv)_{j} = \sum_{i = 1}^{d} L_{ij}v_{i}$ inducing a finite norm constraint at each time $t$. Denote the norm-squared energy density according to 

\begin{eqnarray*}
  E_{v}(t) & = & \int_{\Omega}E_{v}(x,t) \, dx = \| v(t)\|_{L}^{2} \\
  & = & \langle Lv(\cdot, t), Lv(\cdot, t) \rangle < \infty, ~~t \in [0,1]
\label{EnergyDensity}
\end{eqnarray*}

The energy density has been defined through powers of the Laplacian for the classic {\bf thin-plate splines}(??) and the Cauchy-Navier operator for 3-dimensional elasticity. Differential operators with sufficient derivatives and proper boundary conditions insure the existence of solutions of the transport equation in the space of diffeomorphic flows.\\

Now the metric between transformations $g_{0}, g_{1} \in \mathcal{G}$ is defined via the length of the shortest path $g(t)$, $t \in [0, 1]$ with the boundary conditions $g(0) = g_{0}$, $g(1) = g_{1}$. A crucial property of the metric $\rho_{\mathcal{G}}: \mathcal{G} \times \mathcal{G} \rightarrow \mathbb{R}^{+}$ is that it is invariant to $\mathcal{G}$, so that $\rho_{\mathcal{G}}(g_{0}, g_{1}) = \rho_{\mathcal{G}}(g' \cdot g_{0}, g' \cdot g_{1})$ for $g \in \mathcal{G}$, where $g' \cdot g = g \circ g'$. Then preserve the Euler-Lagrange motion equation, therefore the geodesics.\\
The left-invariance follows from that fact that translation by another group element leaves the distance unchanged because $g(\cdot)$ satisfies $\partial g(\cdot, t) / \partial t = v(g(\cdot, t), t)$, $g(0) = Id$, $g(1) = g_{1}$ implying $g(0) \circ h = h$, $g(1) = g_{1} \circ h$, with an identical velocity field so that $\partial g(h(x),t)/\partial t = v(g(h(x),t),t)$, $g(0) = h, g(1) = g_{1}(h)$. The distances $\rho_{\mathcal{G}}(Id, g) = \rho_{\mathcal{G}}(h, g \circ h)$ are equal for all $h \in \mathcal{G}$.

\begin{theorem}
  The function $\rho_{\mathcal{G}}(\cdot, \cdot): \mathcal{G} \times \mathcal{G} \rightarrow \mathbb{R}^{+}$ between elements $g_{0}$, $g_{1} \in \mathcal{G}$ defined as



\begin{eqnarray*}
  \rho_{\mathcal{G}}^{2}(g_{0}, g_{1}) & = & \inf S_{g} = \underset{g(\cdot), g(0)=g_{0}, g(1) = g_{1}}{\inf} \int_{0}^{1} E_{v}(t) \, dt \\
\end{eqnarray*}


is a left-invariant metric distance on $\mathcal{G}$. The geodesics satisfy the Euler-Lagrange equations:

\begin{eqnarray}
  \begin{array}{l}
  \frac{\partial}{\partial t} \nabla_{v}E_{v}(\cdot, t) + (Dv(\cdot, t))^{t} \nabla_{v}E_{v}(\cdot, t) \\
  + (D \nabla_{v}E_{v}(\cdot, t))v(\cdot, t) + div( v(\cdot, t)) \nabla_{v}E_{v}(\cdot, t) = 0
  \end{array}
  \label{Euler_Lagrange}
\end{eqnarray}

where $\nabla$ is the gradient operator delivering a vector, $\nabla_{v}E_{v}(t) = 2L^{\dag}Lv(t)$ with the adjoint defined as $\langle Lf, g\rangle = \langle f, L^{\dag}g \rangle$, $div(v) = \sum_{i} \partial v/ \partial x_{i}$ the divergence operator.
\end{theorem}



\subsubsection{Partial demonstration}

To derive Euler-Lagrange equations for the velocity examine perturbations on the group elements and velocity fields, $g \rightarrow g + \epsilon \eta$, $v \rightarrow v + \epsilon \psi$. For exact correspondence of the group elements, $\eta(0) = \eta(1) = 0$. For inexact image matching, only $\eta(0) = 0$ with $\eta(x, t) = 0$, $\forall x \in \partial \Omega$, $\psi(x, t) = 0$, $\forall x \in \partial \Omega$. If $\eta$ is a perturbation of $g$, define the Gateaux differential of $E(v g): \mathcal{G} \rightarrow \mathbb{R}^{+}$ in the direction $\eta$ to be the limit, as the perturbation tends to 0. Also if $\psi$ is a perturbation of $v$, define the Gateaux differential of $E(g_{v}): \mathcal{V} \rightarrow \mathbb{R}^{+}$ in the direction $\psi$ to be the limit, as the perturbation tends to 0.

\begin{lemma}
\label{lemma}

\begin{eqnarray*}
\psi(x,t) = \partial_{\eta} v_{g}(x,t) & = & \underset{\epsilon \rightarrow 0}{\lim} \frac{v_{g+\epsilon \eta}(x,t) - v_{g}(x,t)}{\epsilon} \\
& = & \frac{d}{dt} \eta(g^{-1}(x,t),t) \\
& - & D_{g} v_{g}(x,t) \eta (g^{-1}(x,t),t)
\end{eqnarray*}

\begin{equation}
  \begin{array}{lcl}
\eta(x,t) = \partial_{\psi} g_{v}(x,t) & = & \underset{\epsilon \rightarrow 0}{\lim} \frac{g_{v + \epsilon \psi}(x,t) - g_{v}(x,t)}{\epsilon} \\
& = & Dg_{v}(x,t) \\
& \times & \int_{0}^{1} du \, (Dg_{v}(x,u))^{-1} \psi_{u} \circ g_{v}(x,u)
\end{array}
\label{lemma_eq}
\end{equation}
\end{lemma}

\begin{proof}
  \begin{equation*}
    \begin{split}
      \frac{d}{dt} (g+\varepsilon \eta) & =  v_{g}(g(x,t),t) + \varepsilon \frac{d}{dt} \eta(x,t) \\
      & = v_{g+\varepsilon \eta}(g+\varepsilon \eta)
    \end{split}
  \end{equation*}

  $$
  v_{g+\varepsilon \eta}(g+\varepsilon \eta) = v_{g}(g+\varepsilon \eta) + \varepsilon \psi
  $$

  Given $\psi = \partial_{\eta} v_{g} (x,t)$ (= $\delta v$ variational calculation in physics). Using the directional derivative:

  $$
  df = \underset{\varepsilon \rightarrow 0}{\lim} \frac{f(x + \varepsilon) - f(x)}{\varepsilon}
  $$

  The Taylor developpment gives:

  $$
  f(x + \varepsilon) = f(x) + \varepsilon df = f(x) + \varepsilon Df
  $$
  $$
  v_{g+\varepsilon \eta}(g+\varepsilon \eta) \sim v_{g}(g(x,t),t) + D_{g} v_{g}(g) \varepsilon \eta + \varepsilon \partial_{\eta} v_{g}
  $$
  
Therefore 

  $$
  \frac{d}{dt} \eta(x,t) = D_{g}v_{g}(g) \eta(x,t) + \partial_{\eta} v_{g}
  $$


  Definition gives $\partial_{t} g = v_{g}(g,t)$. $D \partial_{t} g = \partial_{t} Dg = Dg D_{g}v_{g}(g)$
  $$
  \frac{\partial}{\partial t} \eta = D v_{g} (g) \eta + \psi\\
  = \left[ D \frac{\partial g}{\partial t} \right](Dg)^{-1} \eta + \psi
  $$

  Taking the equation (26) and derivating, we have:

  \begin{equation*}
    \begin{split}
    \partial_{t} \eta(x) & = \partial_{t} \left[ D g_{v}(x,t) \int_{0}^{t} \left[ Dg_{v}(x,u) \right]^{-1} \psi(g_{v}(x,u), u) \, du \right]\\
    & = \partial_{t}D g_{v}(x,t) \int_{0}^{t} \left[ Dg_{v}(x,u) \right]^{-1} \psi(g_{v}(x,u), u) \, du  \\
    & + D g_{v}(x,t) \left[ Dg_{v}(x,t) \right]^{-1} \psi(g_{v}(x,t), t) \\
    & = \partial_{t}D g_{v}(x,t) \left[ Dg_{v}(x,t) \right]^{-1} \eta(x,t) + \psi(g_{v}(x,t), t) \\
    \end{split}
  \end{equation*}

   
\end{proof}


%%%%%%%%%%%%%%%%%%%%%%%%%%%%%%%%%%%%%%%%%%%%%%%%%%%%%%%%%%%%%%%%%%%%%%%%%%

\subsection{Inducing the metric space on anatomical images}

The group of geometric transformations are not directly observable. Rather images are observed via sensors which measure physical properties of the tissues. Carrying out Grenander's metric pattern theory program, the space or orbit of anatomical images $I \in \mathcal{I}$ must be made into a metric space. The metric distance between anatomical imagery $I \in \mathcal{I}$ is constructed from distances between the mappings $g \in \mathcal{G}$. First, the anatomical orbit of all images is defined.


This is a group action on I with $g \cdot I = I \circ g$ and group product $g \cdot g(\cdot) = g \circ g(\cdot) = g(g(\cdot))$, which defines the equivalence relation $I_{1} \sim I_{2}$ if $\exists g \in \mathcal{G}$ such that $I_{1}(\cdot) = I_{2}(g(\cdot))$, dividing $I$ into disjoint orbits.



\begin{theorem}
The function $\rho(\cdot, \cdot): \mathcal{I} \times \mathcal{I} \rightarrow \mathbb{R}^{+}$ between elements $I, I' \in \mathcal{I}$ defined as

\begin{equation}
\rho^{2}(I,I') = \underset{g(\cdot):\frac{\partial}{\partial t}g^{-1}(t) = -Dg^{-1}(t)v(t),\, I(g^{-1}(\cdot,1)) = I'(\cdot), g^{-1}(0) = Id}{\inf} \int_{0}^{1} E_{v}(t) \, dt
\end{equation}

is a metric distance on I satisfying symmetry and the triangle inequality.
\end{theorem}


The fact that $\rho_{\mathcal{G}}$ is left-invariant to $\mathcal{G}$ implies that for all $g \in \mathcal{G}$, $\rho(I \circ g, I' \circ g) = ρ(I, I')$. This also implies any element in the orbit can be taken as the template; all elements are equally good.

\subsection{Expanding the Metric Space to Incorporate Photometric Variation}

Thus far, the metric depends only on the geometric transformations of the background space $\Omega$.

\paragraph{Theorem}{\it Defining the image evolution in the orbit as $J(y, t) = I(g^{−1}(y, t), t)$, $\frac{\partial}{\partial t} g^{-1}(y, t) = −Dg^{−1}(y, t)v(y, t)$, then the function $\rho(\cdot, \cdot): I \times I \rightarrow \mathbb{R}^{+}$ between elements $I$, $I' \in \mathcal{I}$ defined as 

$$
\rho^{2}(I,I') = \underset{v(\cdot),I(\cdot):J(0) = I, J(1) = I'}{\inf} \int_{0}^{1} dt \, \left( \| v(t) \|_{L}^{2} \| \frac{\partial }{\partial t} J(t) + \nabla J^{T} (t)v(t)\|^{2} \right)
$$

is a metric satisfying symmetry and the triangle inequality. Defining

$$
\nabla_{v}E(y,t) = 2L^{\dag}Lv(y,t) + 2 \left( \frac{\partial}{\partial t} J(y,t) + \nabla J^{T}(y,t) v(y,t) \right) \nabla J(y,t)
$$

then the Euler Equation~(\ref{Euler_Lagrange}) holds with boundary term $\nabla E (\cdot, 1) = 0$ with the geodesics for photometric evolution satisfying:

$$
L^{\dag}Lv(\cdot,t) + \left( \frac{\partial}{\partial t} J(\cdot,t) + \nabla J^{T}(\cdot,t) v(\cdot,t) \right) \nabla J(\cdot,t) = 0
$$

$$
\frac{\partial}{\partial t}\left( \frac{\partial J(\cdot,t)}{\partial t} + \nabla J^{T}(\cdot,t) v(\cdot,t) \right) + div\left( \frac{\partial J(\cdot,t)}{\partial t} v(\cdot, t) + (\nabla J^{T}(\cdot,t) v(\cdot,t)) v(\cdot, t) \right) = 0
$$
}

\subsection{Beg's notation}

Let the notation $g_{s,t}: \Omega \leftarrow \Omega$ ($s$ represent the step ) denote the composition $g_{s,t} = g_{t} \circ g_{s}^{-1}$. The interpretation of $g_{s,t}(y)$ is that it is the position at time $t$ of a particle that is at position $y$ at time $s$. Therefore $g_{1}^{v}(x) = g_{0,1}^{v}(x)$  is the function that denotes the position at time $t = 1$ of particle that is at position $x$ at time 0. Let the Jacobian of mapping $g_{s,t}$, the matrix composed with the space derivatives of $g_{s,t}$, be denoted by $Dg_{s,t}$. The lemma equation~(\ref{lemma_eq}) n be written, in Beg's notations

\begin{equation*}
\begin{array}{lcl}
\partial_{h} g_{s,t}^{v} = \underset{\epsilon \leftarrow 0}{\lim} \frac{g_{s,t}^{v + \epsilon h} - g_{s,t}^{v}}{\epsilon}\\
D g_{s,t}^{v} \int_{s}^{t} \left( D g_{s,u}^{v} \right)^{-1} h_{u} \circ g_{s,u}^{v} \, du
\end{array}
\end{equation*}

Given a continuously differentiable idealized template image $I_{0}$ and a noisy observation of anatomy $I_{1}$, then $\hat{v} \in L^{2}([0,1],V)$ for inexact matching of $I_{0}$ and $I_{1}$ is given by

\begin{equation*}
\begin{array}{rcl}
\hat{v} & = &  \underset{v \in L^{2}([0,1],V)}{\arg\min} E(v)\\
& = & \int_{0}^{1} \| v_{t} \|_{V}^{2}\, dt + \frac{1}{\sigma^{2}} \| I_{0} \circ g_{1,0}^{v} - I_{1}\|_{L^{2}}^{2}
\end{array}
\end{equation*}

which satisfies the Euler-Lagrange equation given by

\begin{equation}
\begin{array}{rcl}
2\hat{v}_{t} - K \left( \frac{2}{\sigma^{2}} |D g_{t,1}^{\hat{v}}| \nabla J_{t}^{0}(J_{t}^{0} - J_{t}^{1}) \right) = 0
\end{array}
\end{equation}

where $J_{t}^{0} = I_{0} \circ g_{1,0}^{v}$ and $J_{t}^{1} = I_{1} \circ g_{t,1}^{v}$

\begin{proof}
  Let the velocity $v \in L^{2}([0,1],V)$ be perturbated by an $\epsilon$ amount along the direction $h \in L^{2}([0,1],V)$. the G\^ateau variation $\partial_{h}E(v)$ of the enregy functional related to the Fr\'echet derivative $\nabla_{v}E$ by

  \begin{equation*}
    \begin{array}{rcl}
      \partial_{h}E(v) & = & \underset{\epsilon \leftarrow 0}{\lim} \frac{E(v + vh) - E(v)}{\epsilon} \\
      & = & \int_{0}^{1} <\nabla_{v}E_{t}, h_{t}>\, dt 
      \end{array}
    \end{equation*}

    \paragraph{Nota -- Physics}{$dE = \nabla E \cdot dl$}

    The variation of $E_{1} = \int_{0}^{1} \| v_{t} \|_{V}^{2} \, dt$ is given by: $\partial_{h} E_{1}(v) = 2 \int_{0}^{1} <v_{t},h_{t}> \, dt$. Variation of $E_{2}(v) = \frac{1}{\sigma^{2}} \| I_{0} \circ g_{1,0}^{v} - I_{1}\|_{L^{2}}^{2}$ is

  \begin{equation*}
    \begin{array}{lcl}
      \partial_{h} E_{2}(v) & = & \frac{2}{\sigma^{2}} \left< I_{0} \circ g_{1,0}^{v} - I_{1}, D\left( I_{0} \circ g_{1,0}^{v} \right) \partial_{h} g_{1,0}^{v}  \right>_{L^{2}} \\
      & = & \frac{2}{\sigma^{2}} \left< I_{0} \circ g_{1,0}^{v} - I_{1}, D I_{0} \circ g_{1,0}^{v} \right . \\
      & \times & \left. \left( -D g_{1,0}^{v} \int_{0}^{1} \left( D g_{1,t}^{v} \right)^{-1} h_{t} \circ g_{1,t}^{v} \, dt \right) \right>_{L^{2}}\\
      & = & \frac{2}{\sigma^{2}} \int_{0}^{1} \left< I_{0} \circ g_{1,0}^{v} - I_{1}, D\left( I_{0} \circ g_{1,0}^{v} \right) \right . \\
      & \times & \left. \left( D g_{1,t}^{v} \right)^{-1} h_{t} \circ g_{1,t}^{v}\right>_{L^{2}} \, dt \\
      \end{array}
  \end{equation*}

  In the second equality we substitued of $\partial_{h} g_{1,0}$ is made using Lemma (\ref{lemma_eq}) and in third equality collecting $D(I_{0} \circ g_{1,0}) = DI_{0} \circ g_{1,0}Dg_{1,0}$. Setting $g_{1,t}(y) = x$ {\it i.e}. $g_{t,1}(x) = y$, gives the Jacobian change of variables $|D g_{t,1}(x)| \, dx = dy$. With this, $g_{1,0} \leftarrow g_{1,0} \circ g_{t,1} = g_{t,0}$ and substituting in above, we get:


    \begin{equation*}
    \begin{array}{lcl}
      \partial_{h} E_{2}(v) & = & \\
      & = & - \frac{2}{\sigma^{2}} \int_{0}^{1} \left< |D g_{t,1}^{v}| (I_{0} \circ g_{t,0}^{v} - I_{1} \circ g_{t,1}^{v}), \right . \\  
      & \times & \left. D \left( I_{0} \circ g_{t,0}^{v} \right)^{-1} h_{t} \right>_{L^{2}} \, dt \\
      & = & - \frac{2}{\sigma^{2}} \int_{0}^{1} \left< |D g_{t,1}^{v}| (J_{t}^{0}  - J_{t}^{1}) \nabla J_{t}^{0}, h_{t} \right>_{L^{2}} \, dt \\
      & = & - \frac{2}{\sigma^{2}} \int_{0}^{1} \left< K \left( |D g_{t,1}^{v}| (J_{t}^{0}  - J_{t}^{1}) \nabla J_{t}^{0} \right), h_{t} \right>_{V} \, dt \\
      \end{array}
    \end{equation*}
    
    Collecting terms, the gradient of the energy functional is thus

    \begin{equation}
      \begin{array}{lcl}
        \left( \nabla_{v} E_{t} \right)_{V} = 2v_{t} - \frac{2}{\sigma^{2}} K \left( |D g_{t,1}^{v}| (J_{t}^{0}  - J_{t}^{1}) \nabla J_{t}^{0} \right)
      \end{array}
      \label{gradient_descent}
    \end{equation}

    where the subscript $V$ in $\left( \nabla_{v} E_{t} \right)_{V}$ indicates the gradient is in the space $V$. The optimizing velocity field satisfies the Euler-Lagrange equation

     \begin{equation}
      \begin{array}{lcl}
        \partial_{h}E(\hat{v}) & = & \int_{0}^{1} \left< 2\hat{v}_{t} \right .\\ 
        & - & \left. \frac{2}{\sigma^{2}} K \left( |D g_{t,1}^{\hat{v}}| (J_{t}^{0}  - J_{t}^{1}) \nabla J_{t}^{0} \right), h_{t} \right>_{V} \, dt \\
        & = & 0
      \end{array}
      \label{gradient_descent}
    \end{equation}
   
    
\end{proof}

%%%%%%%%%%%%%%%%%%%%%%%%%%%%%%%%%%%%%%%%%%%%%%%%%%%%%%%%%%%%%%%%%%%%%%%%%%
%%%%%%%%%%%%%%%%%%%%%%%%%%%%%%%%%%%%%%%%%%%%%%%%%%%%%%%%%%%%%%%%%%%%%%%%%%
\section{Euler-Lagrange equation for the inexact image matching and growth}

A central problem in \CA{} is essentially diffeomorphic image interpolation, i.e., to infer the geometric image evolution that connects two elements $I_{0}$, $I_{1} \in \mathcal{I}$ under pure geometric evolution. For this a function of the path is defined; call it $I_{0}(g^{-1}(t))$, $t \in [0, 1]$. The goal is to construct the shortest length curve $g(t)$, $t \in [0, 1]$ which  the target norm squared $\| I_{0} (g^{-1}(1)) - I_{1} \|^{2}$. This is of course inexact matching, since there is a balance between metric length of the path and target correspondence. We can view the image evolution defined by the geodesic as an interpolation between images via the geodesic. Because of the introduction of the free boundary, there is a boundary term which is introduced.

\begin{theorem}
  {\bf Inexact Image Matching} The minimizer

  $$
\underset{g(\cdot):  g(0) = g_{0}}{\inf} \int_{0}^{1} E_{v}(t) \, dt + \| I_{1} - I_{0}(g^{-1}(1))\|^{2}
$$

satisfies the Euler-Lagrange Equation~(\ref{Euler_Lagrange}) with boundary term:

$$
\nabla_{v}E_{v}(\cdot, 1) + 2(I_{1}(\cdot) - I_{0}(g^{-1}(\cdot, 1)))D(I_{0}(g^{-1}(\cdot, 1)))^{T} = 0
$$

where

$$
D(I_{0}(g^{-1}(\cdot,t))) = \nabla I_{0}^{T}(g^{-1}(\cdot, t))Dg^{-1}(\cdot, t)
$$
\end{theorem}





%%%%%%%%%%%%%%%%%%%%%%%%%%%%%%%%%%%%%%%%%%%%%%%%%%%%%%%%%%%%%%%%%%%%%%%%%%
%%%%%%%%%%%%%%%%%%%%%%%%%%%%%%%%%%%%%%%%%%%%%%%%%%%%%%%%%%%%%%%%%%%%%%%%%%

\section{Computational image matching}

\subsection{Beg's Geometric Transformations via Inexact Matching}


In this section, we examine the results of solving the Euler-Lagrange equations for generating the geodesics. Faisal Beg solves inexact image matching (Equation 10) via variations with respect to the velocity field exploiting the vector space structure.

\paragraph{Algorithm}{
Fixed points of the following algorithm satisfy Equations 4 and 11. Initialize $v^{old} = 0$, choose constant $\epsilon$, then for all $t \in [0, 1]$,

Step~1:

$$
\frac{\partial }{\partial t} g^{new}(t) = v^{old}(g^{new}(t), t) = v^{old}(g^{new}(t), t), \frac{\partial }{\partial t} (g^{new})^{-1}(t)v^{old}(t)
$$

$$
\chi^{new}(t) = g^{new}((g^{new}(t))^{-1}, 1)
$$

Step~2: Compute

$$
v^{new}(t) = v^{old}(t) - \epsilon \nabla_{v}E(t)
$$

Set $v^{old} \leftarrow v^{new}$, return to Step 1. Where

$$
\nabla_{v} E(t) = v^{old}(t) + (L^{\dag}L)^{-1} \left( |D \chi^{new}(t)| \times D(I_{0}((g^{new}(t))^{-1}))^{T}(I_{1}(\chi^{new}(t)) - I_{0}((g^{new}(t))^{-1})) \right)
$$

Here $(L^{\dag}L)^{−1} f = Kf$ where $K$ is the Green's kernel. The space-time solution of Equation 13 has gradient


\dots

Where $\chi(t, \tau) = g(g^{-1}(t), \tau)$

}

%%%%%%%%%%%%%%%%%%%%%%%%%%%%%%%%%%%%%%%%%%%%%%%%%%%%%%%%%%%%%%%%%%%%%%%%%%
\section{Conclusion}


%%%%%%%%%%%%%%%%%%%%%%%%%%%%%%%%%%%%%%%%%%%%%%%%%%%%%%%%%%%%%%%%%%%%%%%%%%
%%%%%%%%%%%%%%%%%%%%%%%%%%%%%%%%%%%%%%%%%%%%%%%%%%%%%%%%%%%%%%%%%%%%%%%%%%
\appendix
%%%%%%%%%%%%%%%%%%%%%%%%%%%%%%%%%%%%%%%%%%%%%%%%%%%%%%%%%%%%%%%%%%%%%%%%%%
%%%%%%%%%%%%%%%%%%%%%%%%%%%%%%%%%%%%%%%%%%%%%%%%%%%%%%%%%%%%%%%%%%%%%%%%%%
\section{Green functions}

A Green's function is a solution to an inhomogenous differential equation with a ``driving term'' given by a delta function. It is used as a convenient method for solving more complicated inhomogenous differential equations. Mathematically, it is the kernel of an integral operator that represents the inverse of a differential operator. Physically, it is the response of a system when a unit point source is applied to the system. In this section we show how these two apparently different interpretations are actually the same. \\
In a more general setting, we want to solve the inhomogeous differential equation $\mathcal{L} \psi = f$, where the solution is formally given by: $\psi = \mathcal{L}^{-1} f$. The inverse of a differential operator is an intergral operator we write:

$$
\psi(\bm{r}_{1}) = \int G(\bm{r}_{1}, \bm{r}_{2}) f(\bm{r}_{2}) \, d\nu
$$

The Green function $G(\bm{r}_{1}, \bm{r}_{2})$ is refered as the kernel of the integral operator. By considering the Green's function $G(\bm{r}_{1}, \bm{r}_{2})$ is solution to the Green's function eqaution $\mathcal{L} G(\bm{r}_{1}, \bm{r}_{2}) = \delta(\bm{r}_{1} - \bm{r}_{2})$, as we just introduiced before. We can see that:

$$
\mathcal{L} \psi = \int \mathcal{L} G(\bm{r}_{1}, \bm{r}_{2}) f(\bm{r}_{2}) \, d\nu = f
$$

The original inhomogeneous differential equation. The idea is any source can be decomposed on a superposition of ponctual sources.


%%%%%%%%%%%%%%%%%%%%%%%%%%%%%%%%%%%%%%%%%%%%%%%%%%%%%%%%%%%%%%%%%%%%%%%%%%
\subsection{Motivation observation}

Considering for a moment, the Laplace equation in spherical coordinates, of a function $\psi$ depending only of the radius $\bm{r}$.

$$
\frac{1}{r^{d-1}}\frac{\partial}{\partial r} \left(r^{d-1} \frac{\partial \psi}{\partial r} \right) = 0  \Leftrightarrow for~d>2,~~\psi = \frac{c}{2-d}r^{2-d}.
$$

Where $c$ is a constant. In the three dimensions case, we have a singularity when $r \rightarrow 0$ the potential tends to infinity. Let's consider a very small volume volume tending to the space around the the position zero:

\begin{equation}
  \begin{array}{rcl}
    <\mathcal{L} \psi, h > & = & \int_{r\rightarrow 0} d^{3}\bm{r} \, \Delta \frac{c}{r} \times h(r) \sim h(0)  \times c \int_{r\rightarrow 0} d^{3}\bm{r} \, \Delta \frac{1}{r} \\
    & \sim &  h(0)  \times c \oint_{r\rightarrow 0} d\bm{S} \cdot \bm{\nabla} \frac{1}{r} \sim - h(0)  \times c \frac{4\pi\varepsilon^{2}}{\varepsilon^{2}} \\
  \end{array}
\end{equation}

We can observe $\mathcal{L} \psi$ bhave like a Dirac distribution. We write $\mathcal{L} G(\bm{r}_{1}, \bm{r}_{2}) = \delta(\bm{r}_{1} - \bm{r}_{2})$ where $\bm{r}_{1}$ is the postion of the source and $ \bm{r}_{2}$ is the position where the field is measured. The solution of this equation is the Green function


\begin{equation}
G(\bm{r}_{1}, \bm{r}_{2}) = -\frac{1}{4\pi}\frac{1}{|\bm{r}_{1} - \bm{r}_{2}|}.
\label{eq:Green_Function_laplace}
\end{equation}


In our example, $d=2$, does not have any solution. In this particular case, we build a small disk around the point charge. In Topology, we can detect this point charge with a lasso we pulled: this is the {\it Homotopy}.\\


%%%%%%%%%%%%%%%%%%%%%%%%%%%%%%%%%%%%%%%%%%%%%%%%%%%%%%%%%%%%%%%%%%%%%%%%%%
\subsection{Solution of the Laplace and Poisson equation}

In this section we are going to find the second identity of Green and solve the Poisson equation. Let's start by using the theorem of the divergence on a field $\bm{A}$ in a volume $V$ surrounded by a closed surface $S$.

$$
\int_{V} \bm{\nabla} \cdot \bm{A} \, d\nu = \oint_{S} \bm{A} \cdot d\bm{S}
$$

At this point, to make the Laplace, or Poisson, equation appeared, we write the field $\bm{A} = \varphi \bm{\nabla} \psi$. the theorem of the divergence can be written:

$$
\int_{V} \bm{\nabla}\varphi \cdot \bm{\nabla} \psi + \varphi \Delta \psi\, d\nu = \oint_{S}\varphi \bm{\nabla} \psi \cdot d\bm{S}
$$

Since we arbitrary defined the field $\bm{A}$ with the gradient of $\psi$. We could instead define $\bm{A}$ with the gradient of $\varphi$. Doing so, we make the difference between the two expressions and get Green's second identity:

\begin{equation}
  \int_{V} \varphi \Delta \psi - \psi \Delta \varphi \, d\nu = \oint_{S} \left( \varphi \bm{\nabla} \psi  - \psi \bm{\nabla} \varphi \right) \cdot d\bm{S}
  \label{eq:Green_S_second_identity}
\end{equation}

In the situation of solving the Poisson equation, $\Delta \psi(\bm{r}_{1}) = - \frac{\rho(\bm{r}_{2})}{\varepsilon_{0}}$, we consider $\varphi = G(\bm{r}_{1}, \bm{r}_{2})$ and $\Delta G(\bm{r}_{1}, \bm{r}_{2}) = \delta(\bm{r}_{1} - \bm{r}_{2})$. And replacing these expression in the Green's secod identity, eq.~(\ref{eq:Green_S_second_identity}) we have:

\begin{equation}
  \begin{array}{rcl}
    \psi(\bm{r}_{1}) & = & - \int_{V} G(\bm{r}_{1}, \bm{r}_{2}) \frac{\rho(\bm{r}_{2})}{\varepsilon_{0}} \, d^{3}\bm{r}_{2} \\
    & - & \oint_{S} \left(  G(\bm{r}_{1}, \bm{r}_{2})  \bm{\nabla} \psi(\bm{r}_{2})  - \psi(\bm{r}_{2}) \bm{\nabla} G(\bm{r}_{1}, \bm{r}_{2}) \right) \cdot d\bm{S}
  \end{array}
\end{equation}

If the volume tend to the infinity the surface term vanishing and the comparing the Comlomb potential to the integral we deduce the Green function has the same form as eq.~(\ref{eq:Green_Function_laplace}).

\begin{equation}
  \psi(\bm{r}_{1}) = \int_{V} \frac{\rho(\bm{r}_{2})}{4\pi\varepsilon_{0} |\bm{r}_{1} - \bm{r}_{2}|} \, d^{3}\bm{r}_{2}
  \label{eq:Coulomb_potential}
\end{equation}

%%%%%%%%%%%%%%%%%%%%%%%%%%%%%%%%%%%%%%%%%%%%%%%%%%%%%%%%%%%%%%%%%%%%%%%%%%
%%%%%%%%%%%%%%%%%%%%%%%%%%%%%%%%%%%%%%%%%%%%%%%%%%%%%%%%%%%%%%%%%%%%%%%%%%
\section*{References}
%% References with bibTeX database:
\bibliographystyle{Bibliography/elsarticle-num}

\bibliography{Bibliography/sample}


\end{document}
