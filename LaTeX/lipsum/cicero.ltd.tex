%%
%% This is file `cicero.ltd.tex',
%% generated with the docstrip utility.
%%
%% The original source files were:
%%
%% lipsum.dtx  (with options: `cicero')
%% 
%% This file is part of the package lipsum for use with LaTeX2e.
%% 
%% Function: Access to 150 paragraphs of the well known Lorem Ipsum dummy text.
%% 
%% This program may be distributed and/or modified under the
%% conditions of the LaTeX Project Public License, either version 1.3
%% of this license or (at your option) any later version.
%% The latest version of this license is in
%%    http://www.latex-project.org/lppl.txt
%% and version 1.3 or later is part of all distributions of LaTeX
%% version 2005/12/01 or later.
%% 
%% Please send error reports and suggestions for improvements to
%%     Patrick Happel <patrick.happel@rub.de>
%% 
%% Alternatively, use github:
%%     https://github.com/patta42/lipsum
%% 
\ProvidesFile{cicero.ltd.tex}[2019/01/02 v2.2 Text of Cicero's speech]


%%
%% This text was provided by github-user
%% svenper. See https://github.com/patta42/lipsum/issues/13
%%

\NewLipsumPar{Non eram nescius, Brute, cum, quae summis ingeniis exquisitaque
  doctrina philosophi Graeco sermone tractavissent, ea Latinis litteris
  mandaremus, fore ut hic noster labor in varias reprehensiones incurreret. Nam
  quibusdam, et iis quidem non admodum indoctis, totum hoc displicet
  philosophari. Quidam autem non tam id reprehendunt, si remissius agatur, sed
  tantum studium tamque multam operam ponendam in eo non arbitrantur. Erunt
  etiam, et ii quidem eruditi Graecis litteris, contemnentes Latinas, qui se
  dicant in Graecis legendis operam malle consumere. Postremo aliquos futuros
  suspicor, qui me ad alias litteras vocent, genus hoc scribendi, etsi sit
  elegans, personae tamen et dignitatis esse negent.}
 %{2}
\NewLipsumPar{Contra quos omnis dicendum breviter existimo. Quamquam
  philosophiae quidem vituperatoribus satis responsum est eo libro, quo a nobis
  philosophia defensa et collaudata est, cum esset accusata et vituperata ab
  Hortensio. Qui liber cum et tibi probatus videretur et iis, quos ego posse
  iudicare arbitrarer, plura suscepi veritus ne movere hominum studia viderer,
  retinere non posse. Qui autem, si maxime hoc placeat, moderatius tamen id
  volunt fieri, difficilem quandam temperantiam postulant in eo, quod semel
  admissum coerceri reprimique non potest, ut propemodum iustioribus utamur
  illis, qui omnino avocent a philosophia, quam his, qui rebus infinitis modum
  constituant in reque eo meliore, quo maior sit, mediocritatem desiderent.}
 %{3}
\NewLipsumPar{Sive enim ad sapientiam perveniri potest, non paranda nobis
  solum ea, sed fruenda etiam [sapientia] est; sive hoc difficile est, tamen nec
  modus est ullus investigandi veri, nisi inveneris, et quaerendi defatigatio
  turpis est, cum id, quod quaeritur, sit pulcherrimum. Etenim si delectamur,
  cum scribimus, quis est tam invidus, qui ab eo nos abducat? Sin laboramus,
  quis est, qui alienae modum statuat industriae? Nam ut Terentianus Chremes non
  inhumanus, qui novum vicinum non vult \textquoteleft fodere aut arare aut
  aliquid ferre denique\textquoteright\space\textendash\space non enim illum ab
  industria, sed ab inliberali labore deterret \textendash, sic isti curiosi,
  quos offendit noster minime nobis iniucundus labor.}
 %{4}
\NewLipsumPar{Iis igitur est difficilius satis facere, qui se Latina scripta
  dicunt contemnere. In quibus hoc primum est in quo admirer, cur in gravissimis
  rebus non delectet eos sermo patrius, cum idem fabellas Latinas ad verbum e
  Graecis expressas non inviti legant. Quis enim tam inimicus paene nomini
  Romano est, qui Ennii Medeam aut Antiopam Pacuvii spernat aut reiciat, quod se
  isdem Euripidis fabulis delectari dicat, Latinas litteras oderit?}
 %{5}
\NewLipsumPar{Synephebos ego, inquit, potius Caecilii aut Andriam Terentii
  quam utramque Menandri legam?}
 %{6}
\NewLipsumPar{A quibus tantum dissentio, ut, cum Sophocles vel optime
  scripserit Electram, tamen male conversam Atilii mihi legendam putem, de quo
  Lucilius: \textquoteleft ferreum scriptorem\textquoteright, verum, opinor,
  scriptorem tamen, ut legendus sit. Rudem enim esse omnino in nostris poetis
  aut inertissimae segnitiae est aut fastidii delicatissimi. Mihi quidem nulli
  satis eruditi videntur, quibus nostra ignota sunt. An \textquoteleft Utinam ne
  in nemore\ldots\textquoteright\space nihilo minus legimus quam hoc idem
  Graecum, quae autem de bene beateque vivendo a Platone disputata sunt, haec
  explicari non placebit Latine?}
 %{7}
\NewLipsumPar{Quid? Si nos non interpretum fungimur munere, sed tuemur ea,
  quae dicta sunt ab iis quos probamus, eisque nostrum iudicium et nostrum
  scribendi ordinem adiungimus, quid habent, cur Graeca anteponant iis, quae et
  splendide dicta sint neque sint conversa de Graecis? Nam si dicent ab illis
  has res esse tractatas, ne ipsos quidem Graecos est cur tam multos legant,
  quam legendi sunt. Quid enim est a Chrysippo praetermissum in Stoicis? Legimus
  tamen Diogenem, Antipatrum, Mnesarchum, Panaetium, multos alios in primisque
  familiarem nostrum Posidonium. Quid? Theophrastus mediocriterne delectat, cum
  tractat locos ab Aristotele ante tractatos? Quid? Epicurei num desistunt de
  isdem, de quibus et ab Epicuro scriptum est et ab antiquis, ad arbitrium suum
  scribere? Quodsi Graeci leguntur a Graecis isdem de rebus alia ratione
  compositis, quid est, cur nostri a nostris non legantur?}
 %{8}
\NewLipsumPar{Quamquam, si plane sic verterem Platonem aut Aristotelem, ut
  verterunt nostri poetae fabulas, male, credo, mererer de meis civibus, si ad
  eorum cognitionem divina illa ingenia transferrem. Sed id neque feci adhuc nec
  mihi tamen, ne faciam, interdictum puto. Locos quidem quosdam, si videbitur,
  transferam, et maxime ab iis, quos modo nominavi, cum inciderit, ut id apte
  fieri possit, ut ab Homero Ennius, Afranius a Menandro solet. Nec vero, ut
  noster Lucilius, recusabo, quo minus omnes mea legant. Utinam esset ille
  Persius, Scipio vero et Rutilius multo etiam magis, quorum ille iudicium
  reformidans Tarentinis ait se et Consentinis et Siculis scribere. Facete is
  quidem, sicut alia; sed neque tam docti tum erant, ad quorum iudicium
  elaboraret, et sunt illius scripta leviora, ut urbanitas summa appareat,
  doctrina mediocris.}
 %{9}
\NewLipsumPar{Ego autem quem timeam lectorem, cum ad te ne Graecis quidem
  cedentem in philosophia audeam scribere? Quamquam a te ipso id quidem facio
  provocatus gratissimo mihi libro, quem ad me de virtute misisti. Sed ex eo
  credo quibusdam usu venire; ut abhorreant a Latinis, quod inciderint in
  inculta quaedam et horrida, de malis Graecis Latine scripta deterius. Quibus
  ego assentior, dum modo de isdem rebus ne Graecos quidem legendos putent. Res
  vero bonas verbis electis graviter ornateque dictas quis non legat? Nisi qui
  se plane Graecum dici velit, ut a Scaevola est praetore salutatus Athenis
  Albucius.}
 %{10}
\NewLipsumPar{Quem quidem locum comit multa venustate et omni sale idem
  Lucilius, apud quem praeclare Scaevola:}
 %{11}
\NewLipsumPar{Graecum te, Albuci, quam Romanum atque Sabinum, municipem Ponti,
  Tritani, centurionum, praeclarorum hominum ac primorum signiferumque, maluisti
  dici. Graece ergo praetor Athenis, id quod maluisti, te, cum ad me accedis,
  saluto: \textquoteleft[\ldots],\textquoteright\space inquam, \textquoteleft
  Tite!\textquoteright\space lictores, turma omnis chorusque:
  \textquoteleft[\ldots], Tite!\textquoteright\space hinc hostis mi Albucius,
  hinc inimicus.}
 %{12}
\NewLipsumPar{Sed iure Mucius. Ego autem mirari [satis] non queo unde hoc sit
  tam insolens domesticarum rerum fastidium. Non est omnino hic docendi locus;
  sed ita sentio et saepe disserui, Latinam linguam non modo non inopem, ut
  vulgo putarent, sed locupletiorem etiam esse quam Graecam. Quando enim nobis,
  vel dicam aut oratoribus bonis aut poetis, postea quidem quam fuit quem
  imitarentur, ullus orationis vel copiosae vel elegantis ornatus defuit? Ego
  vero, quoniam forensibus operis, laboribus, periculis non deseruisse mihi
  videor praesidium, in quo a populo Romano locatus sum, debeo profecto,
  quantumcumque possum, in eo quoque elaborare, ut sint opera, studio, labore
  meo doctiores cives mei, nec cum istis tantopere pugnare, qui Graeca legere
  malint, modo legant illa ipsa, ne simulent, et iis servire, qui vel utrisque
  litteris uti velint vel, si suas habent, illas non magnopere desiderent.}
 %{13}
\NewLipsumPar{Qui autem alia malunt scribi a nobis, aequi esse debent, quod et
  scripta multa sunt, sic ut plura nemini e nostris, et scribentur fortasse
  plura, si vita suppetet; et tamen, qui diligenter haec, quae de philosophia
  litteris mandamus, legere assueverit, iudicabit nulla ad legendum his esse
  potiora. Quid est enim in vita tantopere quaerendum quam cum omnia in
  philosophia, tum id, quod his libris quaeritur, qui sit finis, quid extremum,
  quid ultimum, quo sint omnia bene vivendi recteque faciendi consilia
  referenda, quid sequatur natura ut summum ex rebus expetendis, quid fugiat ut
  extremum malorum? Qua de re cum sit inter doctissimos summa dissensio, quis
  alienum putet eius esse dignitatis, quam mihi quisque tribuat, quid in omni
  munere vitae optimum et verissimum sit, exquirere?}
 %{14}
\NewLipsumPar{An, partus ancillae sitne in fructu habendus, disseretur inter
  principes civitatis, P. Scaevolam M\textquoteright.que Manilium, ab iisque M.
  Brutus dissentiet \textendash\space quod et acutum genus est et ad usus civium
  non inutile, nosque ea scripta reliquaque eiusdem generis et legimus libenter
  et legemus \textendash, haec, quae vitam omnem continent, neglegentur? Nam, ut
  sint illa vendibiliora, haec uberiora certe sunt. Quamquam id quidem licebit
  iis existimare, qui legerint. Nos autem hanc omnem quaestionem de finibus
  bonorum et malorum fere a nobis explicatam esse his litteris arbitramur, in
  quibus, quantum potuimus, non modo quid nobis probaretur, sed etiam quid a
  singulis philosophiae disciplinis diceretur, persecuti sumus.}
 %{15}
\NewLipsumPar{Ut autem a facillimis ordiamur, prima veniat in medium Epicuri
  ratio, quae plerisque notissima est. Quam a nobis sic intelleges eitam, ut ab
  ipsis, qui eam disciplinam probant, non soleat accuratius explicari; verum
  enim invenire volumus, non tamquam adversarium aliquem convincere. Accurate
  autem quondam a L. Torquato, homine omni doctrina erudito, defensa est Epicuri
  sententia de voluptate, a meque ei responsum, cum C. Triarius, in primis
  gravis et doctus adolescens, ei disputationi interesset.}
 %{16}
\NewLipsumPar{Nam cum ad me in Cumanum salutandi causa uterque venisset, pauca
  primo inter nos de litteris, quarum summum erat in utroque studium, deinde
  Torquatus: Quoniam nacti te, inquit, sumus aliquando otiosum, certe audiam,
  quid sit, quod Epicurum nostrum non tu quidem oderis, ut fere faciunt, qui ab
  eo dissentiunt, sed certe non probes, eum quem ego arbitror unum vidisse verum
  maximisque erroribus animos hominum liberavisse et omnia tradidisse, quae
  pertinerent ad bene beateque vivendum. Sed existimo te, sicut nostrum
  Triarium, minus ab eo delectari, quod ista Platonis, Aristoteli, Theophrasti
  orationis ornamenta neglexerit. Nam illud quidem adduci vix possum, ut ea,
  quae senserit ille, tibi non vera videantur.}
 %{17}
\NewLipsumPar{Vide, quantum, inquam, fallare, Torquate. Oratio me istius
  philosophi non offendit; nam et complectitur verbis, quod vult, et dicit
  plane, quod intellegam; et tamen ego a philosopho, si afferat eloquentiam, non
  asperner, si non habeat, non admodum flagitem. Re mihi non aeque satisfacit,
  et quidem locis pluribus. Sed quot homines, tot sententiae; falli igitur
  possumus.}
 %{18}
\NewLipsumPar{Quam ob rem tandem, inquit, non satisfacit? Te enim iudicem
  aequum puto, modo quae dicat ille bene noris.}
 %{19}
\NewLipsumPar{Nisi mihi Phaedrum, inquam, tu mentitum aut Zenonem putas,
  quorum utrumque audivi, cum mihi nihil sane praeter sedulitatem probarent,
  omnes mihi Epicuri sententiae satis notae sunt. Atque eos, quos nominavi, cum
  Attico nostro frequenter audivi, cum miraretur ille quidem utrumque, Phaedrum
  autem etiam amaret, cotidieque inter nos ea, quae audiebamus, conferebamus,
  neque erat umquam controversia, quid ego intellegerem, sed quid probarem.}
 %{20}
\NewLipsumPar{Quid igitur est? Inquit; audire enim cupio, quid non probes.
  Principio, inquam, in physicis, quibus maxime gloriatur, primum totus est
  alienus. Democritea dicit perpauca mutans, sed ita, ut ea, quae corrigere
  vult, mihi quidem depravare videatur. Ille atomos quas appellat, id est
  corpora individua propter soliditatem, censet in infinito inani, in quo nihil
  nec summum nec infimum nec medium nec ultimum nec extremum sit, ita ferri, ut
  concursionibus inter se cohaerescant, ex quo efficiantur ea, quae sint quaeque
  cernantur, omnia, eumque motum atomorum nullo a principio, sed ex aeterno
  tempore intellegi convenire.}
 %{21}
\NewLipsumPar{Epicurus autem, in quibus sequitur Democritum, non fere labitur.
  Quamquam utriusque cum multa non probo, tum illud in primis, quod, cum in
  rerum natura duo quaerenda sint, unum, quae materia sit, ex qua quaeque res
  efficiatur, alterum, quae vis sit, quae quidque efficiat, de materia
  disseruerunt, vim et causam efficiendi reliquerunt. Sed hoc commune vitium,
  illae Epicuri propriae ruinae: censet enim eadem illa individua et solida
  corpora ferri deorsum suo pondere ad lineam, hunc naturalem esse omnium
  corporum motum.}
 %{22}
\NewLipsumPar{Deinde ibidem homo acutus, cum illud ocurreret, si omnia deorsus
  e regione ferrentur et, ut dixi, ad lineam, numquam fore ut atomus altera
  alteram posset attingere itaque ** attulit rem commenticiam: declinare dixit
  atomum perpaulum, quo nihil posset fieri minus; ita effici complexiones et
  copulationes et adhaesiones atomorum inter se, ex quo efficeretur mundus
  omnesque partes mundi, quaeque in eo essent. Quae cum tota res (est) ficta
  pueriliter, tum ne efficit [quidem], quod vult. Nam et ipsa declinatio ad
  libidinem fingitur \textendash\space ait enim declinare atomum sine causa; quo
  nihil turpius physico, quam fieri quicquam sine causa dicere,
  \textendash\space et illum motum naturalem omnium ponderum, ut ipse
  constituit, e regione inferiorem locum petentium sine causa eripuit atomis nec
  tamen id, cuius causa haec finxerat, assecutus est.}
 %{23}
\NewLipsumPar{Nam si omnes atomi declinabunt, nullae umquam cohaerescent, sive
  aliae declinabunt, aliae suo nutu recte ferentur, primum erit hoc quasi,
  provincias atomis dare, quae recte, quae oblique ferantur, deinde eadem illa
  atomorum, in quo etiam Democritus haeret, turbulenta concursio hunc mundi
  ornatum efficere non poterit. Ne illud quidem physici, credere aliquid esse
  minimum, quod profecto numquam putavisset, si a Polyaeno, familiari suo,
  geometrica discere maluisset quam illum etiam ipsum dedocere. Sol Democrito
  magnus videtur, quippe homini erudito in geometriaque perfecto, huic pedalis
  fortasse; tantum enim esse censet, quantus videtur, vel paulo aut maiorem aut
  minorem.}
 %{24}
\NewLipsumPar{Ita, quae mutat, ea corrumpit, quae sequitur sunt tota
  Democriti, atomi, inane, imagines, quae [\ldots] nominant, quorum incursione
  non solum videamus, sed etiam cogitemus; infinitio ipsa, quam [\ldots] vocant,
  tota ab illo est, tum innumerabiles mundi, qui et oriantur et intereant
  cotidie. Quae etsi mihi nullo modo probantur, tamen Democritum laudatum a
  ceteris ab hoc, qui eum unum secutus esset, nollem vituperatum.}
 %{25}
\NewLipsumPar{Iam in altera philosophiae parte. Quae est quaerendi ac
  disserendi, quae [\ldots] dicitur, iste vester plane, ut mihi quidem videtur,
  inermis ac nudus est. Tollit definitiones, nihil de dividendo ac partiendo
  docet, non quo modo efficiatur concludaturque ratio tradit, non qua via
  captiosa solvantur ambigua distinguantur ostendit; iudicia rerum in sensibus
  ponit, quibus si semel aliquid falsi pro vero probatum sit, sublatum esse omne
  iudicium veri et falsi putat.}
 %{26}
\NewLipsumPar{Confirmat autem illud vel maxime, quod ipsa natura, ut ait ille,
  sciscat et probet, id est voluptatem et dolorem. Ad haec et quae sequamur et
  quae fugiamus refert omnia. Quod quamquam Aristippi est a Cyrenaicisque melius
  liberiusque defenditur, tamen eius modi esse iudico, ut nihil homine videatur
  indignius. Ad maiora enim quaedam nos natura genuit et conformavit, ut mihi
  quidem videtur. Ac fieri potest, ut errem, sed ita prorsus existimo, neque eum
  Torquatum, qui hoc primus cognomen invenerit, aut torquem illum hosti
  detraxisse, ut aliquam ex eo perciperet corpore voluptatem, aut cum Latinis
  tertio consulatu conflixisse apud Veserim propter voluptatem; quod vero securi
  percussit filium, privavisse se etiam videtur multis voluptatibus, cum ipsi
  naturae patrioque amori praetulerit ius maiestatis atque imperii.}
 %{27}
\NewLipsumPar{Quid? T. Torquatus, is qui consul cum Cn.\ Octavio fuit, cum
  illam severitatem in eo filio adhibuit, quem in adoptionem D. Silano
  emancipaverat, ut eum Macedonum legatis accusantibus, quod pecunias praetorem
  in provincia cepisse arguerent, causam apud se dicere iuberet reque ex utraque
  parte audita pronuntiaret eum non talem videri fuisse in imperio, quales eius
  maiores fuissent, et in conspectum suum venire vetuit, numquid tibi videtur de
  voluptatibus suis cogitavisse?}
 %{28}
\NewLipsumPar{Sed ut omittam pericula, labores, dolorem etiam, quem optimus
  quisque pro patria et pro suis suscipit, ut non modo nullam captet, sed etiam
  praetereat omnes voluptates, dolores denique quosvis suscipere malit quam
  deserere ullam officii partem, ad ea, quae hoc non minus declarant, sed
  videntur leviora, veniamus.}
 %{29}
\NewLipsumPar{Quid tibi, Torquate, quid huic Triario litterae, quid historiae
  cognitioque rerum, quid poetarum evolutio, quid tanta tot versuum memoria
  voluptatis affert? Nec mihi illud dixeris: \textquoteleft Haec enim ipsa mihi
  sunt voluptati, et erant illa Torquatis.\textquoteright\space Numquam hoc ita
  defendit Epicurus neque Metrodorus aut quisquam eorum, qui aut saperet aliquid
  aut ista didicisset. Et quod quaeritur saepe, cur tam multi sint Epicurei,
  sunt aliae quoque causae, sed multitudinem haec maxime allicit, quod ita
  putant dici ab illo, recta et honesta quae sint, ea facere ipsa per se
  laetitiam, id est voluptatem. Homines optimi non intellegunt totam rationem
  everti, si ita res se habeat. Nam si concederetur, etiamsi ad corpus nihil
  referatur, ista sua sponte et per se esse iucunda, per se esset et virtus et
  cognitio rerum, quod minime ille vult expetenda.}
 %{30}
\NewLipsumPar{Haec igitur Epicuri non probo, inquam. De cetero vellem equidem
  aut ipse doctrinis fuisset instructior \textendash\space est enim, quod tibi
  ita videri necesse est, non satis politus iis artibus, quas qui tenent,
  eruditi appellantur \textendash\space aut ne deterruisset alios a studiis.
  Quamquam te quidem video minime esse deterritum.}
 %{31}
\NewLipsumPar{Quae cum dixissem, magis ut illum provocarem quam ut ipse
  loquerer, tum Triarius leniter arridens: Tu quidem, inquit, totum Epicurum
  paene e philosophorum choro sustulisti. Quid ei reliquisti, nisi te, quoquo
  modo loqueretur, intellegere, quid diceret? Aliena dixit in physicis nec ea
  ipsa, quae tibi probarentur; si qua in iis corrigere voluit, deteriora fecit.
  Disserendi artem nullam habuit. Voluptatem cum summum bonum diceret, primum in
  eo ipso parum vidit, deinde hoc quoque alienum; nam ante Aristippus, et ille
  melius. Addidisti ad extremum etiam indoctum fuisse.}
 %{32}
\NewLipsumPar{Fieri, inquam, Triari, nullo pacto potest, ut non dicas, quid
  non probes eius, a quo dissentias. Quid enim me prohiberet Epicureum esse, si
  probarem, quae ille diceret? Cum praesertim illa perdiscere ludus esset. Quam
  ob rem dissentientium inter se reprehensiones non sunt vituperandae,
  maledicta, contumeliae, tum iracundiae, contentiones concertationesque in
  disputando pertinaces indignae philosophia mihi videri solent.}
 %{33}
\NewLipsumPar{Tum Torquatus: Prorsus, inquit, assentior; neque enim disputari
  sine reprehensione nec cum iracundia aut pertinacia recte disputari potest.
  Sed ad haec, nisi molestum est, habeo quae velim. An me, inquam, nisi te
  audire vellem, censes haec dicturum fuisse? Utrum igitur percurri omnem
  Epicuri disciplinam placet an de una voluptate quaeri, de qua omne certamen
  est? Tuo vero id quidem, inquam, arbitratu. Sic faciam igitur, inquit: unam
  rem explicabo, eamque maximam, de physicis alias, et quidem tibi et
  declinationem istam atomorum et magnitudinem solis probabo et Democriti errata
  ab Epicuro reprehensa et correcta permulta. Nunc dicam de voluptate, nihil
  scilicet novi, ea tamen, quae te ipsum probaturum esse confidam.}
 %{34}
\NewLipsumPar{Certe, inquam, pertinax non ero tibique, si mihi probabis ea,
  quae dices, libenter assentiar. Probabo, inquit, modo ista sis aequitate, quam
  ostendis. Sed uti oratione perpetua malo quam interrogare aut interrogari. Ut
  placet, inquam. Tum dicere exorsus est. Primum igitur, inquit, sic agam, ut
  ipsi auctori huius disciplinae placet: constituam, quid et quale sit id, de
  quo quaerimus, non quo ignorare vos arbitrer, sed ut ratione et via procedat
  oratio. Quaerimus igitur, quid sit extremum et ultimum bonorum, quod omnium
  philosophorum sententia tale debet esse, ut ad id omnia referri oporteat,
  ipsum autem nusquam. Hoc Epicurus in voluptate ponit, quod summum bonum esse
  vult, summumque malum dolorem, idque instituit docere sic:}
 %{35}
\NewLipsumPar{Omne animal, simul atque natum sit, voluptatem appetere eaque
  gaudere ut summo bono, dolorem aspernari ut summum malum et, quantum possit, a
  se repellere, idque facere nondum depravatum ipsa natura incorrupte atque
  integre iudicante. Itaque negat opus esse ratione neque disputatione, quam ob
  rem voluptas expetenda, fugiendus dolor sit. Sentiri haec putat, ut calere
  ignem, nivem esse albam, dulce mel. Quorum nihil oportere exquisitis
  rationibus confirmare, tantum satis esse admonere. Interesse enim inter
  argumentum conclusionemque rationis et inter mediocrem animadversionem atque
  admonitionem. Altera occulta quaedam et quasi involuta aperiri, altera prompta
  et aperta iudicari. Etenim quoniam detractis de homine sensibus reliqui nihil
  est, necesse est, quid aut ad naturam aut contra sit, a natura ipsa iudicari.
  Ea quid percipit aut quid iudicat, quo aut petat aut fugiat aliquid, praeter
  voluptatem et dolorem?}
 %{36}
\NewLipsumPar{Sunt autem quidam e nostris, qui haec subtilius velint tradere
  et negent satis esse, quid bonum sit aut quid malum, sensu iudicari, sed animo
  etiam ac ratione intellegi posse et voluptatem ipsam per se esse expetendam et
  dolorem ipsum per se esse fugiendum. Itaque aiunt hanc quasi naturalem atque
  insitam in animis nostris inesse notionem, ut alterum esse appetendum, alterum
  aspernandum sentiamus. Alii autem, quibus ego assentior, cum a philosophis
  compluribus permulta dicantur, cur nec voluptas in bonis sit numeranda nec in
  malis dolor, non existimant oportere nimium nos causae confidere, sed et
  argumentandum et accurate disserendum et rationibus conquisitis de voluptate
  et dolore disputandum putant.}
 %{37}
\NewLipsumPar{Sed ut perspiciatis, unde omnis iste natus error sit voluptatem
  accusantium doloremque laudantium, totam rem aperiam eaque ipsa, quae ab illo
  inventore veritatis et quasi architecto beatae vitae dicta sunt, explicabo.
  Nemo enim ipsam voluptatem, quia voluptas sit, aspernatur aut odit aut fugit,
  sed quia consequuntur magni dolores eos, qui ratione voluptatem sequi
  nesciunt, neque porro quisquam est, qui dolorem ipsum, quia dolor sit, amet,
  consectetur, adipisci velit, sed quia non numquam eius modi tempora incidunt,
  ut labore et dolore magnam aliquam quaerat voluptatem. Ut enim ad minima
  veniam, quis nostrum exercitationem ullam corporis suscipit laboriosam, nisi
  ut aliquid ex ea commodi consequatur? Quis autem vel eum iure reprehenderit,
  qui in ea voluptate velit esse, quam nihil molestiae consequatur, vel illum,
  qui dolorem eum fugiat, quo voluptas nulla pariatur?}
 %{38}
\NewLipsumPar{At vero eos et accusamus et iusto odio dignissimos ducimus, qui
  blanditiis praesentium voluptatum deleniti atque corrupti, quos dolores et
  quas molestias excepturi sint, obcaecati cupiditate non provident, similique
  sunt in culpa, qui officia deserunt mollitia animi, id est laborum et dolorum
  fuga. Et harum quidem rerum facilis est et expedita distinctio. Nam libero
  tempore, cum soluta nobis est eligendi optio, cumque nihil impedit, quo minus
  id, quod maxime placeat, facere possimus, omnis voluptas assumenda est, omnis
  dolor repellendus. Temporibus autem quibusdam et aut officiis debitis aut
  rerum necessitatibus saepe eveniet, ut et voluptates repudiandae sint et
  molestiae non recusandae. Itaque earum rerum hic tenetur a sapiente delectus,
  ut aut reiciendis voluptatibus maiores alias consequatur aut perferendis
  doloribus asperiores repellat.}
 %{39}
\NewLipsumPar{Hanc ego cum teneam sententiam, quid est cur verear, ne ad eam
  non possim accommodare Torquatos nostros? Quos tu paulo ante cum memoriter,
  tum etiam erga nos amice et benivole collegisti, nec me tamen laudandis
  maioribus meis corrupisti nec segniorem ad respondendum reddidisti. Quorum
  facta quem ad modum, quaeso, interpretaris? Sicine eos censes aut in armatum
  hostem impetum fecisse aut in liberos atque in sanguinem suum tam crudelis
  fuisse, nihil ut de utilitatibus, nihil ut de commodis suis cogitarent? At id
  ne ferae quidem faciunt, ut ita ruant itaque turbent, ut earum motus et
  impetus quo pertineant non intellegamus, tu tam egregios viros censes tantas
  res gessisse sine causa?}
 %{40}
\NewLipsumPar{Quae fuerit causa, mox videro; interea hoc tenebo, si ob aliquam
  causam ista, quae sine dubio praeclara sunt, fecerint, virtutem iis per se
  ipsam causam non fuisse. \textendash\space Torquem detraxit hosti.
  \textendash\space Et quidem se texit, ne interiret. \textendash\space At
  magnum periculum adiit. \textendash\space In oculis quidem exercitus.
  \textendash\space Quid ex eo est consecutus? \textendash\space Laudem et
  caritatem, quae sunt vitae sine metu degendae praesidia firmissima.
  \textendash\space Filium morte multavit. \textendash\space Si sine causa,
  nollem me ab eo ortum, tam inportuno tamque crudeli; sin, ut dolore suo
  sanciret militaris imperii disciplinam exercitumque in gravissimo bello
  animadversionis metu contineret, saluti prospexit civium, qua intellegebat
  contineri suam. Atque haec ratio late patet.}
 %{41}
\NewLipsumPar{In quo enim maxime consuevit iactare vestra se oratio, tua
  praesertim, qui studiose antiqua persequeris, claris et fortibus viris
  commemorandis eorumque factis non emolumento aliquo, sed ipsius honestatis
  decore laudandis, id totum evertitur eo delectu rerum, quem modo dixi,
  constituto, ut aut voluptates omittantur maiorum voluptatum adipiscendarum
  causa aut dolores suscipiantur maiorum dolorum effugiendorum gratia.}
 %{42}
\NewLipsumPar{Sed de clarorum hominum factis illustribus et gloriosis satis
  hoc loco dictum sit. Erit enim iam de omnium virtutum cursu ad voluptatem
  proprius disserendi locus. Nunc autem explicabo, voluptas ipsa quae qualisque
  sit, ut tollatur error omnis imperitorum intellegaturque ea, quae voluptaria,
  delicata, mollis habeatur disciplina, quam gravis, quam continens, quam severa
  sit. Non enim hanc solam sequimur, quae suavitate aliqua naturam ipsam movet
  et cum iucunditate quadam percipitur sensibus, sed maximam voluptatem illam
  habemus, quae percipitur omni dolore detracto, nam quoniam, cum privamur
  dolore, ipsa liberatione et vacuitate omnis molestiae gaudemus, omne autem id,
  quo gaudemus, voluptas est, ut omne, quo offendimur, dolor, doloris omnis
  privatio recte nominata est voluptas. Ut enim, cum cibo et potione fames
  sitisque depulsa est, ipsa detractio molestiae consecutionem affert
  voluptatis, sic in omni re doloris amotio successionem efficit voluptatis.}
 %{43}
\NewLipsumPar{Itaque non placuit Epicuro medium esse quiddam inter dolorem et
  voluptatem; illud enim ipsum, quod quibusdam medium videretur, cum omni dolore
  careret, non modo voluptatem esse, verum etiam summam voluptatem. Quisquis
  enim sentit, quem ad modum sit affectus, eum necesse est aut in voluptate esse
  aut in dolore. Omnis autem privatione doloris putat Epicurus terminari summam
  voluptatem, ut postea variari voluptas distinguique possit, augeri
  amplificarique non possit.}
 %{44}
\NewLipsumPar{At etiam Athenis, ut e patre audiebam facete et urbane Stoicos
  irridente, statua est in Ceramico Chrysippi sedentis porrecta manu, quae manus
  significet illum in hae esse rogatiuncula delectatum: \textquoteleft
  Numquidnam manus tua sic affecta, quem ad modum affecta nunc est,
  desiderat?\textquoteright\space \textendash\space Nihil sane.
  \textendash\space\textquoteleft At, si voluptas esset bonum,
  desideraret.\textquoteright\space\textendash\space Ita credo.
  \textendash\space\textquoteleft Non est igitur voluptas
  bonum.\textquoteright\space Hoc ne statuam quidem dicturam pater aiebat, si
  loqui posset. Conclusum est enim contra Cyrenaicos satis acute, nihil ad
  Epicurum. Nam si ea sola voluptas esset, quae quasi titillaret sensus, ut ita
  dicam, et ad eos cum suavitate afflueret et illaberetur, nec manus esse
  contenta posset nec ulla pars vacuitate doloris sine iucundo motu voluptatis.
  Sin autem summa voluptas est, ut Epicuro placet, nihil dolere, primum tibi
  recte, Chrysippe, concessum est nihil desiderare manum, cum ita esset affecta,
  secundum non recte, si voluptas esset bonum, fuisse desideraturam. Idcirco
  enim non desideraret, quia, quod dolore caret, id in voluptate est.}
 %{45}
\NewLipsumPar{Extremum autem esse bonorum voluptatem ex hoc facillime perspici
  potest: Constituamus aliquem magnis, multis, perpetuis fruentem et animo et
  corpore voluptatibus nullo dolore nec impediente nec inpendente, quem tandem
  hoc statu praestabiliorem aut magis expetendum possimus dicere? Inesse enim
  necesse est in eo, qui ita sit affectus, et firmitatem animi nec mortem nec
  dolorem timentis, quod mors sensu careat, dolor in longinquitate levis, in
  gravitate brevis soleat esse, ut eius magnitudinem celeritas, diuturnitatem
  allevatio consoletur.}
 %{46}
\NewLipsumPar{Ad ea cum accedit, ut neque divinum numen horreat nec
  praeteritas voluptates effluere patiatur earumque assidua recordatione
  laetetur, quid est, quod huc possit, quod melius sit, accedere? Statue contra
  aliquem confectum tantis animi corporisque doloribus, quanti in hominem maximi
  cadere possunt, nulla spe proposita fore levius aliquando, nulla praeterea
  neque praesenti nec expectata voluptate, quid eo miserius dici aut fingi
  potest? Quodsi vita doloribus referta maxime fugienda est, summum profecto
  malum est vivere cum dolore, cui sententiae consentaneum est ultimum esse
  bonorum eum voluptate vivere. Nec enim habet nostra mens quicquam, ubi
  consistat tamquam in extremo, omnesque et metus et aegritudines ad dolorem
  referuntur, nec praeterea est res ulla, quae sua natura aut sollicitare possit
  aut angere.}
 %{47}
\NewLipsumPar{Praeterea et appetendi et refugiendi et omnino rerum gerendarum
  initia proficiscuntur aut a voluptate aut a dolore. Quod cum ita sit,
  perspicuum est omnis rectas res atque laudabilis eo referri, ut cum voluptate
  vivatur. Quoniam autem id est vel summum bonorum vel ultimum vel extremum
  \textendash\space quod Graeci [\ldots] nominant \textendash, quod ipsum nullam
  ad aliam rem, ad id autem res referuntur omnes, fatendum est summum esse bonum
  iucunde vivere.}
 %{48}
\NewLipsumPar{Id qui in una virtute ponunt et splendore nominis capti quid
  natura postulet non intellegunt, errore maximo, si Epicurum audire voluerint,
  liberabuntur: istae enim vestrae eximiae pulchraeque virtutes nisi voluptatem
  efficerent, quis eas aut laudabilis aut expetendas arbitraretur? Ut enim
  medicorum scientiam non ipsius artis, sed bonae valetudinis causa probamus, et
  gubernatoris ars, quia bene navigandi rationem habet, utilitate, non arte
  laudatur, sic sapientia, quae ars vivendi putanda est, non expeteretur, si
  nihil efficeret; nunc expetitur, quod est tamquam artifex conquirendae et
  comparandae voluptatis \textendash}
 %{49}
\NewLipsumPar{Quam autem ego dicam voluptatem, iam videtis, ne invidia verbi
  labefactetur oratio mea \textendash. Nam cum ignoratione rerum bonarum et
  malarum maxime hominum vita vexetur, ob eumque errorem et voluptatibus maximis
  saepe priventur et durissimis animi doloribus torqueantur, sapientia est
  adhibenda, quae et terroribus cupiditatibusque detractis et omnium falsarum
  opinionum temeritate derepta certissimam se nobis ducem praebeat ad
  voluptatem. Sapientia enim est una, quae maestitiam pellat ex animis, quae nos
  exhorrescere metu non sinat. Qua praeceptrice in tranquillitate vivi potest
  omnium cupiditatum ardore restincto. Cupiditates enim sunt insatiabiles, quae
  non modo singulos homines, sed universas familias evertunt, totam etiam
  labefactant saepe rem publicam.}
 %{50}
\NewLipsumPar{Ex cupiditatibus odia, discidia, discordiae, seditiones, bella
  nascuntur, nec eae se foris solum iactant nec tantum in alios caeco impetu
  incurrunt, sed intus etiam in animis inclusae inter se dissident atque
  discordant, ex quo vitam amarissimam necesse est effici, ut sapiens solum
  amputata circumcisaque inanitate omni et errore naturae finibus contentus sine
  aegritudine possit et sine metu vivere.}
 %{51}
\NewLipsumPar{Quae est enim aut utilior aut ad bene vivendum aptior partitio
  quam illa, qua est usus Epicurus? Qui unum genus posuit earum cupiditatum,
  quae essent et naturales et necessariae, alterum, quae naturales essent nec
  tamen necessariae, tertium, quae nec naturales nec necessariae. Quarum ea
  ratio est, ut necessariae nec opera multa nec impensa expleantur; ne naturales
  quidem multa desiderant, propterea quod ipsa natura divitias, quibus contenta
  sit, et parabilis et terminatas habet; inanium autem cupiditatum nec modus
  ullus nec finis inveniri potest.}
 %{52}
\NewLipsumPar{Quodsi vitam omnem perturbari videmus errore et inscientia,
  sapientiamque esse solam, quae nos a libidinum impetu et a formidinum terrore
  vindicet et ipsius fortunae modice ferre doceat iniurias et omnis monstret
  vias, quae ad quietem et ad tranquillitatem ferant, quid est cur dubitemus
  dicere et sapientiam propter voluptates expetendam et insipientiam propter
  molestias esse fugiendam?}
 %{53}
\NewLipsumPar{Eademque ratione ne temperantiam quidem propter se expetendam
  esse dicemus, sed quia pacem animis afferat et eos quasi concordia quadam
  placet ac leniat. Temperantia est enim, quae in rebus aut expetendis aut
  fugiendis ut rationem sequamur monet. Nec enim satis est iudicare quid
  faciendum non faciendumve sit, sed stare etiam oportet in eo, quod sit
  iudicatum. Plerique autem, quod tenere atque servare id, quod ipsi statuerunt,
  non possunt, victi et debilitati obiecta specie voluptatis tradunt se
  libidinibus constringendos nec quid eventurum sit provident ob eamque causam
  propter voluptatem et parvam et non necessariam et quae vel aliter pararetur
  et qua etiam carere possent sine dolore tum in morbos gravis, tum in damna,
  tum in dedecora incurrunt, saepe etiam legum iudiciorumque poenis obligantur.}
 %{54}
\NewLipsumPar{Qui autem ita frui volunt voluptatibus, ut nulli propter eas
  consequantur dolores, et qui suum iudicium retinent, ne voluptate victi
  faciant id, quod sentiant non esse faciendum, ii voluptatem maximam
  adipiscuntur praetermittenda voluptate. Idem etiam dolorem saepe perpetiuntur,
  ne, si id non faciant, incidant in maiorem. Ex quo intellegitur nec
  intemperantiam propter se esse fugiendam temperantiamque expetendam, non quia
  voluptates fugiat, sed quia maiores consequatur.}
 %{55}
\NewLipsumPar{Eadem fortitudinis ratio reperietur. Nam neque laborum
  perfunctio neque perpessio dolorum per se ipsa allicit nec patientia nec
  assiduitas nec vigiliae nec ea ipsa, quae laudatur, industria, ne fortitudo
  quidem, sed ista sequimur, ut sine cura metuque vivamus animumque et corpus,
  quantum efficere possimus, molestia liberemus. Ut enim mortis metu omnis
  quietae vitae status perturbatur, et ut succumbere doloribus eosque humili
  animo inbecilloque ferre miserum est, ob eamque debilitatem animi multi
  parentes, multi amicos, non nulli patriam, plerique autem se ipsos penitus
  perdiderunt, sic robustus animus et excelsus omni est liber cura et angore,
  cum et mortem contemnit, qua qui affecti sunt in eadem causa sunt, qua ante
  quam nati, et ad dolores ita paratus est, ut meminerit maximos morte finiri,
  parvos multa habere intervalla requietis, mediocrium nos esse dominos, ut, si
  tolerabiles sint, feramus, si minus, animo aequo e vita, cum ea non placeat,
  tamquam e theatro exeamus. Quibus rebus intellegitur nec timiditatem
  ignaviamque vituperari nec fortitudinem patientiamque laudari suo nomine, sed
  illas reici, quia dolorem pariant, has optari, quia voluptatem.}
 %{56}
\NewLipsumPar{Iustitia restat, ut de omni virtute sit dictum. Sed similia fere
  dici possunt. Ut enim sapientiam, temperantiam, fortitudinem copulatas esse
  docui cum voluptate, ut ab ea nullo modo nec divelli nec distrahi possint, sic
  de iustitia iudicandum est, quae non modo numquam nocet cuiquam, sed contra
  semper afficit cum vi sua atque natura, quod tranquillat animos, tum spe nihil
  earum rerum defuturum, quas natura non depravata desiderat. [et] quem ad modum
  temeritas et libido et ignavia semper animum excruciant et semper sollicitant
  turbulentaeque sunt, sic [inprobitas si] cuius in mente consedit, hoc ipso,
  quod adest, turbulenta est; si vero molita quippiam est, quamvis occulte
  fecerit, numquam tamen id confidet fore semper occultum. Plerumque improborum
  facta primo suspicio insequitur, dein sermo atque fama, tum accusator, tum
  iudex;}
 %{57}
\NewLipsumPar{Multi etiam, ut te consule, ipsi se indicaverunt. Quodsi qui
  satis sibi contra hominum conscientiam saepti esse et muniti videntur, deorum
  tamen horrent easque ipsas sollicitudines, quibus eorum animi noctesque
  diesque exeduntur, a diis inmortalibus supplicii causa importari putant. Quae
  autem tanta ex improbis factis ad minuendas vitae molestias accessio potest
  fieri, quanta ad augendas, cum conscientia factorum, tum poena legum odioque
  civium? Et tamen in quibusdam neque pecuniae modus est neque honoris neque
  imperii nec libidinum nec epularum nec reliquarum cupiditatum, quas nulla
  praeda umquam improbe parta minuit, [sed] potius inflammat, ut coercendi magis
  quam dedocendi esse videantur.}
 %{58}
\NewLipsumPar{Invitat igitur vera ratio bene sanos ad iustitiam, aequitatem,
  fidem, neque homini infanti aut inpotenti iniuste facta conducunt, qui nec
  facile efficere possit, quod conetur, nec optinere, si effecerit, et opes vel
  fortunae vel ingenii liberalitati magis conveniunt, qua qui utuntur,
  benivolentiam sibi conciliant et, quod aptissimum est ad quiete vivendum,
  caritatem, praesertim cum omnino nulla sit causa peccandi.}
 %{59}
\NewLipsumPar{Quae enim cupiditates a natura proficiscuntur, facile explentur
  sine ulla iniuria, quae autem inanes sunt, iis parendum non est. Nihil enim
  desiderabile concupiscunt, plusque in ipsa iniuria detrimenti est quam in iis
  rebus emolumenti, quae pariuntur iniuria. Itaque ne iustitiam quidem recte
  quis dixerit per se ipsam optabilem, sed quia iucunditatis vel plurimum
  afferat. Nam diligi et carum esse iucundum est propterea, quia tutiorem vitam
  et voluptatem pleniorem efficit. Itaque non ob ea solum incommoda, quae
  eveniunt inprobis, fugiendam inprobitatem putamus, sed multo etiam magis,
  quod, cuius in animo versatur, numquam sinit eum respirare, numquam
  adquiescere.}
 %{60}
\NewLipsumPar{Quodsi ne ipsarum quidem virtutum laus, in qua maxime ceterorum
  philosophorum exultat oratio, reperire exitum potest, nisi derigatur ad
  voluptatem, voluptas autem est sola, quae nos vocet ad se et alliciat suapte
  natura, non potest esse dubium, quin id sit summum atque extremum bonorum
  omnium, beateque vivere nihil aliud sit nisi cum voluptate vivere.}
 %{61}
\NewLipsumPar{Huic certae stabilique sententiae quae sint coniuncta explicabo
  brevi. Nullus in ipsis error est finibus bonorum et malorum, id est in
  voluptate aut in dolore, sed in his rebus peccant, cum e quibus haec
  efficiantur ignorant. Animi autem voluptates et dolores nasci fatemur e
  corporis voluptatibus et doloribus \textendash\space itaque concedo, quod modo
  dicebas, cadere causa, si qui e nostris aliter existimant, quos quidem video
  esse multos, sed imperitos \textendash, quamquam autem et laetitiam nobis
  voluptas animi et molestiam dolor afferat, eorum tamen utrumque et ortum esse
  e corpore et ad corpus referri, nec ob eam causam non multo maiores esse et
  voluptates et dolores animi quam corporis. Nam corpore nihil nisi praesens et
  quod adest sentire possumus, animo autem et praeterita et futura. Ut enim
  aeque doleamus animo, cum corpore dolemus, fieri tamen permagna accessio
  potest, si aliquod aeternum et infinitum impendere malum nobis opinemur. Quod
  idem licet transferre in voluptatem, ut ea maior sit, si nihil tale metuamus.}
 %{62}
\NewLipsumPar{Iam illud quidem perspicuum est, maximam animi aut voluptatem
  aut molestiam plus aut ad beatam aut ad miseram vitam afferre momenti quam
  eorum utrumvis, si aeque diu sit in corpore. Non placet autem detracta
  voluptate aegritudinem statim consequi, nisi in voluptatis locum dolor forte
  successerit, at contra gaudere nosmet omittendis doloribus, etiamsi voluptas
  ea, quae sensum moveat, nulla successerit, eoque intellegi potest quanta
  voluptas sit non dolere.}
 %{63}
\NewLipsumPar{Sed ut iis bonis erigimur, quae expectamus, sic laetamur iis,
  quae recordamur. Stulti autem malorum memoria torquentur, sapientes bona
  praeterita grata recordatione renovata delectant. Est autem situm in nobis ut
  et adversa quasi perpetua oblivione obruamus et secunda iucunde ac suaviter
  meminerimus. Sed cum ea, quae praeterierunt, acri animo et attento intuemur,
  tum fit ut aegritudo sequatur, si illa mala sint, laetitia, si bona.}
 %{64}
\NewLipsumPar{O praeclaram beate vivendi et apertam et simplicem et directam
  viam! Cum enim certe nihil homini possit melius esse quam vacare omni dolore
  et molestia perfruique maximis et animi et corporis voluptatibus, videtisne
  quam nihil praetermittatur quod vitam adiuvet, quo facilius id, quod
  propositum est, summum bonum consequamur? Clamat Epicurus, is quem vos nimis
  voluptatibus esse deditum dicitis; non posse iucunde vivi, nisi sapienter,
  honeste iusteque vivatur, nec sapienter, honeste, iuste, nisi iucunde.}
 %{65}
\NewLipsumPar{Neque enim civitas in seditione beata esse potest nec in
  discordia dominorum domus; quo minus animus a se ipse dissidens secumque
  discordans gustare partem ullam liquidae voluptatis et liberae potest. Atqui
  pugnantibus et contrariis studiis consiliisque semper utens nihil quieti
  videre, nihil tranquilli potest.}
 %{66}
\NewLipsumPar{Quodsi corporis gravioribus morbis vitae iucunditas impeditur,
  quanto magis animi morbis impediri necesse est! Animi autem morbi sunt
  cupiditates inmensae et inanes divitiarum, gloriae, dominationis,
  libidinosarum etiam voluptatum. Accedunt aegritudines, molestiae, maerores,
  qui exedunt animos conficiuntque curis hominum non intellegentium nihil
  dolendum esse animo, quod sit a dolore corporis praesenti futurove seiunctum.
  Nec vero quisquam stultus non horum morborum aliquo laborat, nemo igitur est
  non miser.}
 %{67}
\NewLipsumPar{Accedit etiam mors, quae quasi saxum Tantalo semper impendet,
  tum superstitio, qua qui est imbutus quietus esse numquam potest. Praeterea
  bona praeterita non meminerunt, praesentibus non fruuntur, futura modo
  expectant, quae quia certa esse non possunt, conficiuntur et angore et metu
  maximeque cruciantur, cum sero sentiunt frustra se aut pecuniae studuisse aut
  imperiis aut opibus aut gloriae. Nullas enim consequuntur voluptates, quarum
  potiendi spe inflammati multos labores magnosque susceperant.}
 %{68}
\NewLipsumPar{Ecce autem alii minuti et angusti aut omnia semper desperantes
  aut malivoli, invidi, difficiles, lucifugi, maledici, monstruosi, alii autem
  etiam amatoriis levitatibus dediti, alii petulantes, alii audaces, protervi,
  idem intemperantes et ignavi, numquam in sententia permanentes, quas ob causas
  in eorum vita nulla est intercapedo molestiae. Igitur neque stultorum quisquam
  beatus neque sapientium non beatus. Multoque hoc melius nos veriusque quam
  Stoici. Illi enim negant esse bonum quicquam nisi nescio quam illam umbram,
  quod appellant honestum non tam solido quam splendido nomine, virtutem autem
  nixam hoc honesto nullam requirere voluptatem atque ad beate vivendum se ipsa
  esse contentam.}
 %{69}
\NewLipsumPar{Sed possunt haec quadam ratione dici non modo non repugnantibus,
  verum etiam approbantibus nobis. Sic enim ab Epicuro sapiens semper beatus
  inducitur: finitas habet cupiditates, neglegit mortem, de diis inmortalibus
  sine ullo metu vera sentit, non dubitat, si ita melius sit, migrare de vita.
  His rebus instructus semper est in voluptate. Neque enim tempus est ullum, quo
  non plus voluptatum habeat quam dolorum. Nam et praeterita grate meminit et
  praesentibus ita potitur, ut animadvertat quanta sint ea quamque iucunda,
  neque pendet ex futuris, sed expectat illa, fruitur praesentibus ab iisque
  vitiis, quae paulo ante collegi, abest plurimum et, cum stultorum vitam cum
  sua comparat, magna afficitur voluptate. Dolores autem si qui incurrunt,
  numquam vim tantam habent, ut non plus habeat sapiens, quod gaudeat, quam quod
  angatur.}
 %{70}
\NewLipsumPar{Optime vero Epicurus, quod exiguam dixit fortunam intervenire
  sapienti maximasque ab eo et gravissimas res consilio ipsius et ratione
  administrari neque maiorem voluptatem ex infinito tempore aetatis percipi
  posse, quam ex hoc percipiatur, quod videamus esse finitum. In dialectica
  autem vestra nullam existimavit esse nec ad melius vivendum nec ad commodius
  disserendum viam. In physicis plurimum posuit. Ea scientia et verborum vis et
  natura orationis et consequentium repugnantiumve ratio potest perspici. Omnium
  autem rerum natura cognita levamur superstitione, liberamur mortis metu, non
  conturbamur ignoratione rerum, e qua ipsa horribiles existunt saepe
  formidines. Denique etiam morati melius erimus, cum didicerimus quid natura
  desideret. Tum vero, si stabilem scientiam rerum tenebimus, servata illa, quae
  quasi delapsa de caelo est ad cognitionem omnium, regula, ad quam omnia
  iudicia rerum dirigentur, numquam ullius oratione victi sententia desistemus.}
 %{71}
\NewLipsumPar{Nisi autem rerum natura perspecta erit, nullo modo poterimus
  sensuum iudicia defendere. Quicquid porro animo cernimus, id omne oritur a
  sensibus; qui si omnes veri erunt, ut Epicuri ratio docet, tum denique poterit
  aliquid cognosci et percipi. Quos qui tollunt et nihil posse percipi dicunt,
  ii remotis sensibus ne id ipsum quidem expedire possunt, quod disserunt.
  Praeterea sublata cognitione et scientia tollitur omnis ratio et vitae
  degendae et rerum gerendarum. Sic e physicis et fortitudo sumitur contra
  mortis timorem et constantia contra metum religionis et sedatio animi omnium
  rerum occultarum ignoratione sublata et moderatio natura cupiditatum
  generibusque earum explicatis, et, ut modo docui, cognitionis regula et
  iudicio ab eadem illa constituto veri a falso distinctio traditur.}
 %{72}
\NewLipsumPar{Restat locus huic disputationi vel maxime necessarius de
  amicitia, quam, si voluptas summum sit bonum, affirmatis nullam omnino fore.
  De qua Epicurus quidem ita dicit, omnium rerum, quas ad beate vivendum
  sapientia comparaverit, nihil esse maius amicitia, nihil uberius, nihil
  iucundius. Nec vero hoc oratione solum, sed multo magis vita et factis et
  moribus comprobavit. Quod quam magnum sit fictae veterum fabulae declarant, in
  quibus tam multis tamque variis ab ultima antiquitate repetitis tria vix
  amicorum paria reperiuntur, ut ad Orestem pervenias profectus a Theseo. At
  vero Epicurus una in domo, et ea quidem angusta, quam magnos quantaque amoris
  conspiratione consentientis tenuit amicorum greges! Quod fit etiam nunc ab
  Epicureis. Sed ad rem redeamus; de hominibus dici non necesse est.}
 %{73}
\NewLipsumPar{Tribus igitur modis video esse a nostris de amicitia disputatum.
  Alii cum eas voluptates, quae ad amicos pertinerent, negarent esse per se
  ipsas tam expetendas, quam nostras expeteremus, quo loco videtur quibusdam
  stabilitas amicitiae vacillare, tuentur tamen eum locum seque facile, ut mihi
  videtur, expediunt. Ut enim virtutes, de quibus ante dictum est, sic amicitiam
  negant posse a voluptate discedere. Nam cum solitudo et vita sine amicis
  insidiarum et metus plena sit, ratio ipsa monet amicitias comparare, quibus
  partis confirmatur animus et a spe pariendarum voluptatum seiungi non potest.}
 %{74}
\NewLipsumPar{Atque ut odia, invidiae, despicationes adversantur voluptatibus,
  sic amicitiae non modo fautrices fidelissimae, sed etiam effectrices sunt
  voluptatum tam amicis quam sibi, quibus non solum praesentibus fruuntur, sed
  etiam spe eriguntur consequentis ac posteri temporis. Quod quia nullo modo
  sine amicitia firmam et perpetuam iucunditatem vitae tenere possumus neque
  vero ipsam amicitiam tueri, nisi aeque amicos et nosmet ipsos diligamus,
  idcirco et hoc ipsum efficitur in amicitia, et amicitia cum voluptate
  conectitur. Nam et laetamur amicorum laetitia aeque atque nostra et pariter
  dolemus angoribus.}
 %{75}
\NewLipsumPar{Quocirca eodem modo sapiens erit affectus erga amicum, quo in se
  ipsum, quosque labores propter suam voluptatem susciperet, eosdem suscipiet
  propter amici voluptatem. Quaeque de virtutibus dicta sunt, quem ad modum eae
  semper voluptatibus inhaererent, eadem de amicitia dicenda sunt. Praeclare
  enim Epicurus his paene verbis: \textquoteleft Eadem\textquoteright, inquit,
  \textquoteleft scientia confirmavit animum, ne quod aut sempiternum aut
  diuturnum timeret malum, quae perspexit in hoc ipso vitae spatio amicitiae
  praesidium esse firmissimum.\textquoteright}
 %{76}
\NewLipsumPar{Sunt autem quidam Epicurei timidiores paulo contra vestra
  convicia, sed tamen satis acuti, qui verentur ne, si amicitiam propter nostram
  voluptatem expetendam putemus, tota amicitia quasi claudicare videatur. Itaque
  primos congressus copulationesque et consuetudinum instituendarum voluntates
  fieri propter voluptatem; cum autem usus progrediens familiaritatem effecerit,
  tum amorem efflorescere tantum, ut, etiamsi nulla sit utilitas ex amicitia,
  tamen ipsi amici propter se ipsos amentur. Etenim si loca, si fana, si urbes,
  si gymnasia, si campum, si canes, si equos, si ludicra exercendi aut venandi
  consuetudine adamare solemus, quanto id in hominum consuetudine facilius fieri
  poterit et iustius?}
 %{77}
\NewLipsumPar{Sunt autem, qui dicant foedus esse quoddam sapientium, ut ne
  minus amicos quam se ipsos diligant. Quod et posse fieri intellegimus et saepe
  etiam videmus, et perspicuum est nihil ad iucunde vivendum reperiri posse,
  quod coniunctione tali sit aptius. Quibus ex omnibus iudicari potest non modo
  non impediri rationem amicitiae, si summum bonum in voluptate ponatur, sed
  sine hoc institutionem omnino amicitiae non posse reperiri.}
 %{78}
\NewLipsumPar{Quapropter si ea, quae dixi, sole ipso illustriora et clariora
  sunt, si omnia dixi hausta e fonte naturae, si tota oratio nostra omnem sibi
  fidem sensibus confirmat, id est incorruptis atque integris testibus, si
  infantes pueri, mutae etiam bestiae paene loquuntur magistra ac duce natura
  nihil esse prosperum nisi voluptatem, nihil asperum nisi dolorem, de quibus
  neque depravate iudicant neque corrupte, nonne ei maximam gratiam habere
  debemus, qui hac exaudita quasi voce naturae sic eam firme graviterque
  comprehenderit, ut omnes bene sanos in viam placatae, tranquillae, quietae,
  beatae vitae deduceret? Qui quod tibi parum videtur eruditus, ea causa est,
  quod nullam eruditionem esse duxit, nisi quae beatae vitae disciplinam
  iuvaret.}
 %{79}
\NewLipsumPar{An ille tempus aut in poetis evolvendis, ut ego et Triarius te
  hortatore facimus, consumeret, in quibus nulla solida utilitas omnisque
  puerilis est delectatio, aut se, ut Plato, in musicis, geometria, numeris,
  astris contereret, quae et a falsis initiis profecta vera esse non possunt et,
  si essent vera, nihil afferrent, quo iucundius, id est quo melius viveremus,
  eas ergo artes persequeretur, vivendi artem tantam tamque et operosam et
  perinde fructuosam relinqueret? Non ergo Epicurus ineruditus, sed ii indocti,
  qui, quae pueros non didicisse turpe est, ea putant usque ad senectutem esse
  discenda.}
 %{80}
\NewLipsumPar{Quae cum dixisset, Explicavi, inquit, sententiam meam, et eo
  quidem consilio, tuum iudicium ut cognoscerem, quoniam mihi ea facultas, ut id
  meo arbitratu facerem, ante hoc tempus numquam est data.}
 %{81}
\NewLipsumPar{Hic cum uterque me intueretur seseque ad audiendum significarent
  paratos, Primum, inquam, deprecor, ne me tamquam philosophum putetis scholam
  vobis aliquam explicaturum, quod ne in ipsis quidem philosophis magnopere
  umquam probavi. Quando enim Socrates, qui parens philosophiae iure dici
  potest, quicquam tale fecit? Eorum erat iste mos qui tum sophistae
  nominabantur, quorum e numero primus est ausus Leontinus Gorgias in conventu
  poscere quaestionem, id est iubere dicere, qua de re quis vellet audire. Audax
  negotium, dicerem impudens, nisi hoc institutum postea translatum ad
  philosophos nostros esset.}
 %{82}
\NewLipsumPar{Sed et illum, quem nominavi, et ceteros sophistas, ut e Platone
  intellegi potest, lusos videmus a Socrate. Is enim percontando atque
  interrogando elicere solebat eorum opiniones, quibuscum disserebat, ut ad ea,
  quae ii respondissent, si quid videretur, diceret. Qui mos cum a posterioribus
  non esset retentus, Arcesilas eum revocavit instituitque ut ii, qui se audire
  vellent, non de se quaererent, sed ipsi dicerent, quid sentirent; quod cum
  dixissent, ille contra. Sed eum qui audiebant, quoad poterant, defendebant
  sententiam suam. Apud ceteros autem philosophos, qui quaesivit aliquid, tacet;
  quod quidem iam fit etiam in Academia. Ubi enim is, qui audire vult, ita
  dixit: \textquoteleft Voluptas mihi videtur esse summum bonum\textquoteright,
  perpetua oratione contra disputatur, ut facile intellegi possit eos, qui
  aliquid sibi videri dicant, non ipsos in ea sententia esse, sed audire velle
  contraria.}
 %{83}
\NewLipsumPar{Nos commodius agimus. Non enim solum Torquatus dixit quid
  sentiret, sed etiam cur. Ego autem arbitror, quamquam admodum delectatus sum
  eius oratione perpetua, tamen commodius, cum in rebus singulis insistas et
  intellegas quid quisque concedat, quid abnuat, ex rebus concessis concludi
  quod velis et ad exitum perveniri. Cum enim fertur quasi torrens oratio,
  quamvis multa cuiusque modi rapiat, nihil tamen teneas, nihil apprehendas,
  nusquam orationem rapidam coerceas. Omnis autem in quaerendo, quae via quadam
  et ratione habetur, oratio praescribere primum debet ut quibusdam in formulis
  ea res agetur, ut, inter quos disseritur, conveniat quid sit id, de quo
  disseratur.}
 %{84}
\NewLipsumPar{Hoc positum in Phaedro a Platone probavit Epicurus sensitque in
  omni disputatione id fieri oportere. Sed quod proximum fuit non vidit. Negat
  enim definiri rem placere, sine quo fieri interdum non potest, ut inter eos,
  qui ambigunt, conveniat quid sit id, de quo agatur, velut in hoc ipso, de quo
  nunc disputamus. Quaerimus enim finem bonorum. Possumusne hic scire qualis
  sit, nisi contulerimus inter nos, cum finem bonorum dixerimus, quid finis,
  quid etiam sit ipsum bonum?}
 %{85}
\NewLipsumPar{Atqui haec patefactio quasi rerum opertarum, cum quid quidque
  sit aperitur, definitio est. Qua tu etiam inprudens utebare non numquam. Nam
  hunc ipsum sive finem sive extremum sive ultimum definiebas id esse, quo
  omnia, quae recte fierent, referrentur neque id ipsum usquam referretur.
  Praeclare hoc quidem. Bonum ipsum etiam quid esset, fortasse, si opus fuisset,
  definisses aut quod esset natura adpetendum aut quod prodesset aut quod
  iuvaret aut quod liberet modo. Nunc idem, nisi molestum est, quoniam tibi non
  omnino displicet definire et id facis, cum vis, velim definias quid sit
  voluptas, de quo omnis haec quaestio est.}
 %{86}
\NewLipsumPar{Quis, quaeso, inquit, est, qui quid sit voluptas nesciat, aut
  qui, quo magis id intellegat, definitionem aliquam desideret? Me ipsum esse
  dicerem, inquam, nisi mihi viderer habere bene cognitam voluptatem et satis
  firme conceptam animo atque comprehensam. Nunc autem dico ipsum Epicurum
  nescire et in eo nutare eumque, qui crebro dicat diligenter oportere exprimi
  quae vis subiecta sit vocibus, non intellegere interdum, quid sonet haec vox
  voluptatis, id est quae res huic voci subiciatur. Tum ille ridens: Hoc vero,
  inquit, optimum, ut is, qui finem rerum expetendarum voluptatem esse dicat, id
  extremum, id ultimum bonorum, id ipsum quid et quale sit, nesciat! Atqui,
  inquam, aut Epicurus quid sit voluptas aut omnes mortales, qui ubique sunt,
  nesciunt. Quonam, inquit, modo? Quia voluptatem hanc esse sentiunt omnes, quam
  sensus accipiens movetur et iucunditate quadam perfunditur.}
 %{87}
\NewLipsumPar{Quid ergo? Istam voluptatem, inquit, Epicurus ignorat? Non
  semper, inquam; nam interdum nimis etiam novit, quippe qui testificetur ne
  intellegere quidem se posse ubi sit aut quod sit ullum bonum praeter illud,
  quod cibo et potione et aurium delectatione et obscena voluptate capiatur. An
  haec ab eo non dicuntur? Quasi vero me pudeat, inquit, istorum, aut non possim
  quem ad modum ea dicantur ostendere! Ego vero non dubito, inquam, quin facile
  possis, nec est quod te pudeat sapienti adsentiri, qui se unus, quod sciam,
  sapientem profiteri sit ausus. Nam Metrodorum non puto ipsum professum, sed,
  cum appellaretur ab Epicuro, repudiare tantum beneficium noluisse; septem
  autem illi non suo, sed populorum suffragio omnium nominati sunt.}
 %{88}
\NewLipsumPar{Verum hoc loco sumo verbis his eandem certe vim voluptatis
  Epicurum nosse quam ceteros. Omnes enim iucundum motum, quo sensus hilaretur.
  Graece [\ldots], Latine voluptatem vocant. Quid est igitur, inquit, quod
  requiras? Dicam, inquam, et quidem discendi causa magis, quam quo te aut
  Epicurum reprehensum velim. Ego quoque, inquit, didicerim libentius si quid
  attuleris, quam te reprehenderim. Tenesne igitur, inquam, Hieronymus Rhodius
  quid dicat esse summum bonum, quo putet omnia referri oportere? Teneo, inquit,
  finem illi videri nihil dolere. Quid? Idem iste, inquam, de voluptate quid
  sentit?}
 %{89}
\NewLipsumPar{Negat esse eam, inquit, propter se expetendam. Aliud igitur esse
  censet gaudere, aliud non dolere. Et quidem, inquit, vehementer errat; nam, ut
  paulo ante docui, augendae voluptatis finis est doloris omnis amotio. Non
  dolere, inquam, istud quam vim habeat postea videro; aliam vero vim voluptatis
  esse, aliam nihil dolendi, nisi valde pertinax fueris, concedas necesse est.
  Atqui reperies, inquit, in hoc quidem pertinacem; dici enim nihil potest
  verius. Estne, quaeso, inquam, sitienti in bibendo voluptas? Quis istud
  possit, inquit, negare? Eademne, quae restincta siti? Immo alio genere;
  restincta enim sitis stabilitatem voluptatis habet, inquit, illa autem
  voluptas ipsius restinctionis in motu est. Cur igitur, inquam, res tam
  dissimiles eodem nomine appellas?}
 %{90}
\NewLipsumPar{Quid paulo ante, inquit, dixerim nonne meministi, cum omnis
  dolor detractus esset, variari, non augeri voluptatem? Memini vero, inquam;
  sed tu istuc dixti bene Latine, parum plane. Varietas enim Latinum verbum est,
  idque proprie quidem in disparibus coloribus dicitur, sed transfertur in multa
  disparia: varium poema, varia oratio, varii mores, varia fortuna, voluptas
  etiam varia dici solet, cum percipitur e multis dissimilibus rebus dissimilis
  efficientibus voluptates. Eam si varietatem diceres, intellegerem, ut etiam
  non dicente te intellego; ista varietas quae sit non satis perspicio, quod
  ais, cum dolore careamus, tum in summa voluptate nos esse, cum autem vescamur
  iis rebus, quae dulcem motum afferant sensibus, tum esse in motu voluptatem,
  qui faciat varietatem voluptatum, sed non augeri illam non dolendi voluptatem,
  quam cur voluptatem appelles nescio.}
 %{91}
\NewLipsumPar{An potest, inquit ille, quicquam esse suavius quam nihil dolere?
  Immo sit sane nihil melius, inquam\textendash nondum enim id
  quaero\textendash, num propterea idem voluptas est, quod, ut ita dicam,
  indolentia? Plane idem, inquit, et maxima quidem, qua fieri nulla maior
  potest. Quid dubitas igitur, inquam, summo bono a te ita constituto, ut id
  totum in non dolendo sit, id tenere unum, id tueri, id defendere?}
 %{92}
\NewLipsumPar{Quid enim necesse est, tamquam meretricem in matronarum coetum,
  sic voluptatem in virtutum concilium adducere? Invidiosum nomen est, infame,
  suspectum. Itaque hoc frequenter dici solet a vobis, non intellegere nos, quam
  dicat Epicurus voluptatem. Quod quidem mihi si quando dictum est\textendash
  est autem dictum non parum saepe\textendash, etsi satis clemens sum in
  disputando, tamen interdum soleo subirasci. Egone non intellego, quid sit
  [\ldots] Graece, Latine voluptas? Utram tandem linguam nescio? Deinde qui fit,
  ut ego nesciam, sciant omnes, quicumque Epicurei esse voluerunt? Quod vestri
  quidem vel optime disputant, nihil opus esse eum, qui philosophus futurus sit,
  scire litteras. Itaque ut maiores nostri ab aratro adduxerunt Cincinnatum
  illum, ut dictator esset, sic vos de pagis omnibus colligitis bonos illos
  quidem viros, sed certe non pereruditos.}
 %{93}
\NewLipsumPar{Ergo illi intellegunt quid Epicurus dicat, ego non intellego? Ut
  scias me intellegere, primum idem esse dico voluptatem, quod ille [\ldots]. Et
  quidem saepe quaerimus verbum Latinum par Graeco et quod idem valeat; hic
  nihil fuit, quod quaereremus. Nullum inveniri verbum potest quod magis idem
  declaret Latine, quod Graece, quam declarat voluptas. Huic verbo omnes, qui
  ubique sunt, qui Latine sciunt, duas res subiciunt, laetitiam in animo,
  commotionem suavem iucunditatis in corpore. Nam et ille apud Trabeam
  \textquoteleft voluptatem animi nimiam\textquoteright\space laetitiam dicit
  eandem, quam ille Caecilianus, qui \textquoteleft omnibus laetitiis
  laetum\textquoteright\space esse se narrat. Sed hoc interest, quod voluptas
  dicitur etiam in animo\textendash vitiosa res, ut Stoici putant, qui eam sic
  definiunt: sublationem animi sine ratione opinantis se magno bono
  frui\textendash, non dicitur laetitia nec gaudium in corpore.}
 %{94}
\NewLipsumPar{In eo autem voluptas omnium Latine loquentium more ponitur, cum
  percipitur ea, quae sensum aliquem moveat, iucunditas. Hanc quoque
  iucunditatem, si vis, transfer in animum; iuvare enim in utroque dicitur, ex
  eoque iucundum, modo intellegas inter illum, qui dicat: \textquoteleft Tanta
  laetitia auctus sum, ut nihil constet\textquoteright, et eum, qui:
  \textquoteleft Nunc demum mihi animus ardet\textquoteright, quorum alter
  laetitia gestiat, alter dolore crucietur, esse illum medium: \textquoteleft
  Quamquam haec inter nos nuper notitia admodum est\textquoteright, qui nec
  laetetur nec angatur, itemque inter eum, qui potiatur corporis expetitis
  voluptatibus, et eum, qui crucietur summis doloribus, esse eum, qui utroque
  careat.}
 %{95}
\NewLipsumPar{Satisne igitur videor vim verborum tenere, an sum etiam nunc vel
  Graece loqui vel Latine docendus? Et tamen vide, ne, si ego non intellegam
  quid Epicurus loquatur, cum Graece, ut videor, luculenter sciam, sit aliqua
  culpa eius, qui ita loquatur, ut non intellegatur. Quod duobus modis sine
  reprehensione fit, si aut de industria facias, ut Heraclitus, \textquoteleft
  cognomento qui [\ldots] perhibetur, quia de natura nimis obscure
  memoravit\textquoteright, aut cum rerum obscuritas, non verborum, facit ut non
  intellegatur oratio, qualis est in Timaeo Platonis. Epicurus autem, ut opinor,
  nec non vult, si possit, plane et aperte loqui, nec de re obscura, ut physici,
  aut artificiosa, ut mathematici, sed de illustri et facili et iam in vulgus
  pervagata loquitur. Quamquam non negatis nos intellegere quid sit voluptas,
  sed quid ille dicat. E quo efficitur, non ut nos non intellegamus quae vis sit
  istius verbi, sed ut ille suo more loquatur, nostrum neglegat.}
 %{96}
\NewLipsumPar{Si enim idem dicit, quod Hieronymus, qui censet summum bonum
  esse sine ulla molestia vivere, cur mavult dicere voluptatem quam vacuitatem
  doloris, ut ille facit, qui quid dicat intellegit? Sin autem voluptatem putat
  adiungendam eam, quae sit in motu\textendash sic enim appellat hanc dulcem:
  \textquoteleft in motu\textquoteright, illam nihil dolentis \textquoteleft in
  stabilitate\textquoteright\textendash, quid tendit? Cum efficere non possit ut
  cuiquam, qui ipse sibi notus sit, hoc est qui suam naturam sensumque
  perspexerit, vacuitas doloris et voluptas idem esse videatur. Hoc est vim
  afferre, Torquate, sensibus, extorquere ex animis cognitiones verborum, quibus
  inbuti sumus. Quis enim est, qui non videat haec esse in natura rerum tria?
  Unum, cum in voluptate sumus, alterum, cum in dolore, tertium hoc, in quo nunc
  equidem sum, credo item vos, nec in dolore nec in voluptate; ut in voluptate
  sit, qui epuletur, in dolore, qui torqueatur. Tu autem inter haec tantam
  multitudinem hominum interiectam non vides nec laetantium nec dolentium?}
 %{97}
\NewLipsumPar{Non prorsus, inquit, omnisque, qui sine dolore sint, in
  voluptate, et ea quidem summa, esse dico. Ergo in eadem voluptate eum, qui
  alteri misceat mulsum ipse non sitiens, et eum, qui illud sitiens bibat? Tum
  ille: Finem, inquit, interrogandi, si videtur, quod quidem ego a principio ita
  me malle dixeram hoc ipsum providens, dialecticas captiones. Rhetorice igitur,
  inquam, nos mavis quam dialectice disputare? Quasi vero, inquit, perpetua
  oratio rhetorum solum, non etiam philosophorum sit. Zenonis est, inquam, hoc
  Stoici. Omnem vim loquendi, ut iam ante Aristoteles, in duas tributam esse
  partes, rhetoricam palmae, dialecticam pugni similem esse dicebat, quod latius
  loquerentur rhetores, dialectici autem compressius. Obsequar igitur voluntati
  tuae dicamque, si potero, rhetorice, sed hac rhetorica philosophorum, non
  nostra illa forensi, quam necesse est, cum populariter loquatur, esse interdum
  paulo hebetiorem.}
 %{98}
\NewLipsumPar{Sed dum dialecticam, Torquate, contemnit Epicurus, quae una
  continet omnem et perspiciendi quid in quaque re sit scientiam et iudicandi
  quale quidque sit et ratione ac via disputandi, ruit in dicendo, ut mihi
  quidem videtur, nec ea, quae docere vult, ulla arte distinguit, ut haec ipsa,
  quae modo loquebamur. Summum a vobis bonum voluptas dicitur. Aperiendum est
  igitur, quid sit voluptas; aliter enim explicari, quod quaeritur, non potest.
  Quam si explicavisset, non tam haesitaret. Aut enim eam voluptatem tueretur,
  quam Aristippus, id est, qua sensus dulciter ac iucunde movetur, quam etiam
  pecudes, si loqui possent, appellarent voluptatem, aut, si magis placeret suo
  more loqui, quam ut Omnes Danai atque Mycenenses. Attica pubes reliquique
  Graeci, qui hoc anapaesto citantur, hoc non dolere solum voluptatis nomine
  appellaret, illud Aristippeum contemneret, aut, si utrumque probaret, ut
  probat, coniungeret doloris vacuitatem cum voluptate et duobus ultimis
  uteretur.}
 %{99}
\NewLipsumPar{Multi enim et magni philosophi haec ultima bonorum iuncta
  fecerunt, ut Aristoteles virtutis usum cum vitae perfectae prosperitate
  coniunxit, Callipho adiunxit ad honestatem voluptatem, Diodorus ad eandem
  honestatem addidit vacuitatem doloris. Idem fecisset Epicurus, si sententiam
  hanc, quae nunc Hieronymi est, coniunxisset cum Aristippi vetere sententia.
  Illi enim inter se dissentiunt. Propterea singulis finibus utuntur et, cum
  uterque Graece egregie loquatur, nec Aristippus, qui voluptatem summum bonum
  dicit, in voluptate ponit non dolere, neque Hieronymus, qui summum bonum
  statuit non dolere, voluptatis nomine umquam utitur pro illa indolentia,
  quippe qui ne in expetendis quidem rebus numeret voluptatem.}
 %{100}
\NewLipsumPar{Duae sunt enim res quoque, ne tu verba solum putes. Unum est
  sine dolore esse, alterum cum voluptate. Vos ex his tam dissimilibus rebus non
  modo nomen unum \textendash nam id facilius paterer\textendash, sed etiam rem
  unam ex duabus facere conamini, quod fieri nullo modo potest. Hic, qui
  utrumque probat, ambobus debuit uti, sicut facit re, neque tamen dividit
  verbis. Cum enim eam ipsam voluptatem, quam eodem nomine omnes appellamus,
  laudat locis plurimis, audet dicere ne suspicari quidem se ullum bonum
  seiunctum ab illo Aristippeo genere voluptatis, atque ibi hoc dicit, ubi omnis
  eius est oratio de summo bono. In alio vero libro, in quo breviter
  comprehensis gravissimis sententiis quasi oracula edidisse sapientiae dicitur,
  scribit his verbis, quae nota tibi profecto, Torquate, sunt\textendash quis
  enim vestrum non edidicit Epicuri [\ldots], id est quasi maxime ratas, quia
  gravissimae sint ad beate vivendum breviter enuntiatae sententiae?\textendash
  animadverte igitur rectene hanc sententiam interpreter:}
 %{101}
\NewLipsumPar{\textquoteleft Si ea, quae sunt luxuriosis efficientia
  voluptatum, liberarent eos deorum et mortis et doloris metu docerentque qui
  essent fines cupiditatum, nihil haberemus (quod reprehenderemus), cum undique
  complerentur voluptatibus nec haberent ulla ex parte aliquid aut dolens aut
  aegrum, id est autem malum.\textquoteright\space Hoc loco tenere se Triarius
  non potuit. Obsecro, inquit, Torquate, haec dicit Epicurus? Quod mihi quidem
  visus est, cum sciret, velle tamen confitentem audire Torquatum. At ille non
  pertimuit saneque fidenter: Istis quidem ipsis verbis, inquit; sed quid
  sentiat, non videtis. Si alia sentit, inquam, alia loquitur, numquam
  intellegam quid sentiat; sed plane dicit quod intellegit. Idque si ita dicit,
  non esse reprehendendos luxuriosos, si sapientes sint, dicit absurde,
  similiter et si dicat non reprehendendos parricidas, si nec cupidi sint nec
  deos metuant nec mortem nec dolorem. Et tamen quid attinet luxuriosis ullam
  exceptionem dari aut fingere aliquos, qui, cum luxuriose viverent, a summo
  philosopho non reprehenderentur eo nomine dumtaxat, cetera caverent?}
 %{102}
\NewLipsumPar{Sed tamen nonne reprehenderes, Epicure, luxuriosos ob eam ipsam
  causam, quod ita viverent, ut persequerentur cuiusque modi voluptates, cum
  esset praesertim, ut ais tu, summa voluptas nihil dolere? Atqui reperiemus
  asotos primum ita non religiosos, ut edint de patella, deinde ita mortem non
  timentes, ut illud in ore habeant ex Hymnide: \textquoteleft Mihi sex menses
  satis sunt vitae, septimum Orco spondeo\textquoteright. Iam doloris
  medicamenta illa Epicurea tamquam de narthecio proment: \textquoteleft Si
  gravis, brevis; si longus, levis.\textquoteright\space Unum nescio, quo modo
  possit, si luxuriosus sit, finitas cupiditates habere.}
 %{103}
\NewLipsumPar{Quid ergo attinet dicere: \textquoteleft Nihil haberem, quod
  reprehenderem, si finitas cupiditates haberent\textquoteright? Hoc est dicere:
  \textquoteleft Non reprehenderem asotos, si non essent
  asoti.\textquoteright\space isto modo ne improbos quidem, si essent boni viri.
  Hic homo severus luxuriam ipsam per se reprehendendam non putat, et hercule,
  Torquate, ut verum loquamur, si summum bonum voluptas est, rectissime non
  putat. Noli enim mihi fingere asotos, ut soletis, qui in mensam vomant, et qui
  de conviviis auferantur crudique postridie se rursus ingurgitent, qui solem,
  ut aiunt, nec occidentem umquam viderint nec orientem, qui consumptis
  patrimoniis egeant. Nemo nostrum istius generis asotos iucunde putat vivere.
  Mundos, elegantis, optimis cocis, pistoribus, piscatu, aucupio, venatione, his
  omnibus exquisitis, vitantes cruditatem, quibus vinum defusum e pleno sit
  chrysizon, ut ait Lucilius, cui nihildum situlus et sacculus abstulerit,
  adhibentis ludos et quae sequuntur, illa, quibus detractis clamat Epicurus se
  nescire quid sit bonum; adsint etiam formosi pueri, qui ministrent, respondeat
  his vestis, argentum, Corinthium, locus ipse, aedificium\textendash hos ergo
  asotos bene quidem vivere aut beate numquam dixerim.}
 %{104}
\NewLipsumPar{Ex quo efficitur, non ut voluptas ne sit voluptas, sed ut
  voluptas non sit summum bonum. Nec ille, qui Diogenem Stoicum adolescens, post
  autem Panaetium audierat, Laelius, eo dictus est sapiens, quod non
  intellegeret quid suavissimum esset\textendash nec enim sequitur, ut, cui cor
  sapiat, ei non sapiat palatus\textendash, sed quia parvi id duceret. O
  lapathe, ut iactare, nec es satis cognitu\textquoteright\space qui sis! In quo
  [cognitu] Laelius clamores [\ldots] ille so lebat Edere compellans gumias ex
  ordine nostros. Praeclare Laelius, et recte [\ldots], illudque vere: O Publi,
  o gurges, Galloni! Es homo miser, inquit. Cenasti in vita numquam bene, cum
  omnia in ista Consumis squilla atque acupensere cum decimano. Is haec
  loquitur, qui in voluptate nihil ponens negat eum bene cenare, qui omnia ponat
  in voluptate, et tamen non negat libenter cenasse umquam
  Gallonium\textendash\space mentiretur enim\textendash, sed bene. Ita graviter
  et severe voluptatem secrevit a bono. Ex quo illud efficitur, qui bene cenent
  omnis libenter cenare, qui libenter, non continuo bene.}
 %{105}
\NewLipsumPar{Semper Laelius bene. Quid bene? Dicet Lucilius: \textquoteleft
  cocto, condito\textquoteright, sed cedo caput cenae: \textquoteleft sermone
  bono\textquoteright, quid ex eo? \textquoteleft si quaeris,
  libenter\textquoteright; veniebat enim ad cenam, ut animo quieto satiaret
  desideria naturae. Recte ergo is negat umquam bene cenasse Gallonium, recte
  miserum, cum praesertim in eo omne studium consumeret. Quem libenter cenasse
  nemo negat. Cur igitur non bene? Quia, quod bene, id recte, frugaliter,
  honeste; ille porro [male] prave, nequiter, turpiter cenabat; non igitur
  (bene). Nec lapathi suavitatem acupenseri Galloni Laelius anteponebat, sed
  suavitatem ipsam neglegebat; quod non faceret, si in voluptate summum bonum
  poneret.}
 %{106}
\NewLipsumPar{Semovenda est igitur voluptas, non solum ut recta sequamini, sed
  etiam ut loqui deceat frugaliter. Possumusne ergo in vita summum bonum dicere,
  cum id ne in cena quidem posse videamur? Quo modo autem philosophus loquitur?
  \textquoteleft Tria genera cupiditatum, naturales et necessariae, naturales et
  non necessariae, nec naturales nec necessariae.\textquoteright\space primum
  divisit ineleganter; duo enim genera quae erant, fecit tria. Hoc est non
  dividere, sed frangere. Qui haec didicerunt, quae ille contemnit, sic solent:
  \textquoteleft Duo genera cupiditatum, naturales et inanes, naturalium duo,
  necessariae et non necessariae.\textquoteright\space confecta res esset.
  Vitiosum est enim in dividendo partem in genere numerare.}
 %{107}
\NewLipsumPar{Sed hoc sane concedamus. Contemnit enim disserendi elegantiam,
  confuse loquitur. Gerendus est mos, modo recte sentiat. Et quidem illud ipsum
  non nimium probo et tantum patior, philosophum loqui de cupiditatibus
  finiendis. An potest cupiditas finiri? Tollenda est atque extrahenda
  radicitus. Quis est enim, in quo sit cupiditas, quin recte cupidus dici
  possit? Ergo et avarus erit, sed finite, et adulter, verum habebit modum, et
  luxuriosus eodem modo. Qualis ista philosophia est, quae non interitum afferat
  pravitatis, sed sit contenta mediocritate vitiorum? Quamquam in hac divisione
  rem ipsam prorsus probo, elegantiam desidero. Appellet haec desideria naturae,
  cupiditatis nomen servet alio, ut eam, cum de avaritia, cum de intemperantia,
  cum de maximis vitiis loquetur, tamquam capitis accuset.}
 %{108}
\NewLipsumPar{Sed haec quidem liberius ab eo dicuntur et saepius. Quod equidem
  non reprehendo; est enim tanti philosophi tamque nobilis audacter sua decreta
  defendere. Sed tamen ex eo, quod eam voluptatem, quam omnes gentes hoc nomine
  appellant, videtur amplexari saepe vehementius, in magnis interdum versatur
  angustiis, ut hominum conscientia remota nihil tam turpe sit, quod voluptatis
  causa non videatur esse facturus. Deinde, ubi erubuit\textendash vis enim est
  permagna naturae\textendash, confugit illuc, ut neget accedere quicquam posse
  ad voluptatem nihil dolentis. At iste non dolendi status non vocatur voluptas.
  \textquoteleft Non laboro\textquoteright, inquit, \textquoteleft de
  nomine\textquoteright. Quid, quod res alia tota est? \textquoteleft Reperiam
  multos, vel innumerabilis potius, non tam curiosos nec tam molestos, quam vos
  estis, quibus, quid velim, facile persuadeam.\textquoteright\space quid ergo
  dubitamus, quin, si non dolere voluptas sit summa, non esse in voluptate dolor
  sit maximus? Cur id non ita fit? \textquoteleft Quia dolori non voluptas
  contraria est, sed doloris privatio.\textquoteright}
 %{109}
\NewLipsumPar{Hoc vero non videre, maximo argumento esse voluptatem illam, qua
  sublata neget se intellegere omnino quid sit bonum\textendash eam autem ita
  persequitur: quae palato percipiatur, quae auribus; cetera addit, quae si
  appelles, honos praefandus sit\textendash hoc igitur, quod solum bonum severus
  et gravis philosophus novit, idem non videt ne expetendum quidem esse, quod
  eam voluptatem hoc eodem auctore non desideremus, cum dolore careamus.}
 %{110}
\NewLipsumPar{Quam haec sunt contraria! Hic si definire, si dividere
  didicisset, si loquendi vim, si denique consuetudinem verborum teneret,
  numquam in tantas salebras incidisset. Nunc vides, quid faciat. Quam nemo
  umquam voluptatem appellavit, appellat; quae duo sunt, unum facit. Hanc in
  motu voluptatem \textendash sic enim has suaves et quasi dulces voluptates
  appellat\textendash interdum ita extenuat, ut M\textquoteright. Curium putes
  loqui, interdum ita laudat, ut quid praeterea sit bonum neget se posse ne
  suspicari quidem. Quae iam oratio non a philosopho aliquo, sed a censore
  opprimenda est. Non est enim vitium in oratione solum, sed etiam in moribus.
  Luxuriam non reprehendit, modo sit vacua infinita cupiditate et timore. Hoc
  loco discipulos quaerere videtur, ut, qui asoti esse velint, philosophi ante
  fiant.}
 %{111}
\NewLipsumPar{A primo, ut opinor, animantium ortu petitur origo summi boni.
  \textquoteleft Simul atque natum animal est, gaudet voluptate et eam appetit
  ut bonum, aspernatur dolorem ut malum.\textquoteright\space De malis autem et
  bonis ab iis animalibus, quae nondum depravata sint, ait optime iudicari. Haec
  et tu ita posuisti, et verba vestra sunt. Quam multa vitiosa! Summum enim
  bonum et malum vagiens puer utra voluptate diiudicabit, stante an movente?
  Quoniam, si dis placet, ab Epicuro loqui discimus. Si stante, hoc natura
  videlicet vult, salvam esse se, quod concedimus; si movente, quod tamen
  dicitis, nulla turpis voluptas erit, quae praetermittenda sit, et simul non
  proficiscitur animal illud modo natum a summa voluptate, quae est a te posita
  in non dolendo.}
 %{112}
\NewLipsumPar{Nec tamen argumentum hoc Epicurus a parvis petivit aut etiam a
  bestiis, quae putat esse specula naturae, ut diceret ab iis duce natura hanc
  voluptatem expeti nihil dolendi. Nec enim haec movere potest appetitum animi,
  nec ullum habet ictum, quo pellat animum, status hic non dolendi, itaque in
  hoc eodem peccat Hieronymus. At ille pellit, qui permulcet sensum voluptate.
  Itaque Epicurus semper hoc utitur, ut probet voluptatem natura expeti, quod ea
  voluptas, quae in motu sit, et parvos ad se alliciat et bestias, non illa
  stabilis, in qua tantum inest nihil dolere. Qui igitur convenit ab alia
  voluptate dicere naturam proficisci, in alia summum bonum ponere?}
 %{113}
\NewLipsumPar{Bestiarum vero nullum iudicium puto. Quamvis enim depravatae non
  sint, pravae tamen esse possunt. Ut bacillum aliud est inflexum et incurvatum
  de industria, aliud ita natum, sic ferarum natura non est illa quidem
  depravata mala disciplina, sed natura sua. Nec vero ut voluptatem expetat,
  natura movet infantem, sed tantum ut se ipse diligat, ut integrum se salvumque
  velit. Omne enim animal, simul et ortum est, se ipsum et omnes partes suas
  diligit duasque, quae maximae sunt, in primis amplectitur, animum et corpus,
  deinde utriusque partes. Nam sunt et in animo praecipua quaedam et in corpore,
  quae cum leviter agnovit, tum discernere incipit, ut ea, quae prima data sunt
  natura, appetat asperneturque contraria.}
 %{114}
\NewLipsumPar{In his primis naturalibus voluptas insit necne, magna quaestio
  est. Nihil vero putare esse praeter voluptatem, non membra, non sensus, non
  ingenii motum, non integritatem corporis, non valitudinem [corporis], summae
  mihi videtur inscitiae. Atque ab isto capite fluere necesse est omnem rationem
  bonorum et malorum. Polemoni et iam ante Aristoteli ea prima visa sunt, quae
  paulo ante dixi. Ergo nata est sententia veterum Academicorum et
  Peripateticorum, ut finem bonorum dicerent secundum naturam vivere, id est
  virtute adhibita frui primis a natura datis. Callipho ad virtutem nihil
  adiunxit nisi voluptatem, Diodorus vacuitatem doloris. ** his omnibus, quos
  dixi, consequentes fines sunt bonorum, Aristippo simplex voluptas, Stoicis
  consentire naturae, quod esse volunt e virtute, id est honeste, vivere, quod
  ita interpretantur: vivere cum intellegentia rerum earum, quae natura
  evenirent, eligentem ea, quae essent secundum naturam, reicientemque
  contraria.}
 %{115}
\NewLipsumPar{Ita tres sunt fines expertes honestatis, unus Aristippi vel
  Epicuri, alter Hieronymi, Carneadi tertius, tres, in quibus honestas cum
  aliqua accessione, Polemonis, Calliphontis, Diodori, una simplex, cuius Zeno
  auctor, posita in decore tota, id est in honestate; nam Pyrrho, Aristo,
  Erillus iam diu abiecti. Reliqui sibi constiterunt, ut extrema cum initiis
  convenirent, ut Aristippo voluptas, Hieronymo doloris vacuitas, Carneadi frui
  principiis naturalibus esset extremum. Epicurus autem cum in prima
  commendatione voluptatem dixisset, si eam, quam Aristippus, idem tenere debuit
  ultimum bonorum, quod ille; sin eam, quam Hieronymus, (ne) fecisset idem, ut
  voluptatem illam Aristippi in prima commendatione poneret.}
 %{116}
\NewLipsumPar{Nam quod ait sensibus ipsis iudicari voluptatem bonum esse,
  dolorem malum, plus tribuit sensibus, quam nobis leges permittunt, (cum)
  privatarum litium iudices sumus. Nihil enim possumus iudicare, nisi quod est
  nostri iudicii\textendash in quo frustra iudices solent, cum sententiam
  pronuntiant, addere: \textquoteleft si quid mei iudicii est\textquoteright; si
  enim non fuit eorum iudicii, nihilo magis hoc non addito illud est
  iudicatum\textendash. Quid iudicant sensus? Dulce amarum, leve asperum, prope
  longe, stare movere, quadratum rotundum.}
 %{117}
\NewLipsumPar{Aequam igitur pronuntiabit sententiam ratio adhibita primum
  divinarum humanarumque rerum scientia, quae potest appellari rite sapientia,
  deinde adiunctis virtutibus, quas ratio rerum omnium dominas, tu voluptatum
  satellites et ministras esse voluisti. Quarum adeo omnium sententia
  pronuntiabit primum de voluptate nihil esse ei loci, non modo ut sola ponatur
  in summi boni sede, quam quaerimus, sed ne illo quidem modo, ut ad honestatem
  applicetur. De vacuitate doloris eadem sententia erit.}
 %{118}
\NewLipsumPar{Reicietur etiam Carneades, nec ulla de summo bono ratio aut
  voluptatis non dolendive particeps aut honestatis expers probabitur. Ita
  relinquet duas, de quibus etiam atque etiam consideret. Aut enim statuet nihil
  esse bonum nisi honestum, nihil malum nisi turpe, cetera aut omnino nihil
  habere momenti aut tantum, ut nec expetenda nec fugienda, sed eligenda modo
  aut reicienda sint, aut anteponet eam, quam cum honestate ornatissimam, tum
  etiam ipsis initiis naturae et totius perfectione vitae locupletatam videbit.
  Quod eo liquidius faciet, si perspexerit rerum inter eas verborumne sit
  controversia.}
 %{119}
\NewLipsumPar{Huius ego nunc auctoritatem sequens idem faciam. Quantum enim
  potero, minuam contentiones omnesque simplices sententias eorum, in quibus
  nulla inest virtutis adiunctio, omnino a philosophia semovendas putabo, primum
  Aristippi Cyrenaicorumque omnium, quos non est veritum in ea voluptate, quae
  maxima dulcedine sensum moveret, summum bonum ponere contemnentis istam
  vacuitatem doloris.}
 %{120}
\NewLipsumPar{Hi non viderunt, ut ad cursum equum, ad arandum bovem, ad
  indagandum canem, sic hominem ad duas res, ut ait Aristoteles, ad
  intellegendum et agendum, esse natum quasi mortalem deum, contraque ut tardam
  aliquam et languidam pecudem ad pastum et ad procreandi voluptatem hoc divinum
  animal ortum esse voluerunt, quo nihil mihi videtur absurdius.}
 %{121}
\NewLipsumPar{Atque haec contra Aristippum, qui eam voluptatem non modo
  summam, sed solam etiam ducit, quam omnes unam appellamus voluptatem. Aliter
  autem vobis placet. Sed ille, ut dixi, vitiose. Nec enim figura corporis nec
  ratio excellens ingenii humani significat ad unam hanc rem natum hominem, ut
  frueretur voluptatibus. Nec vero audiendus Hieronymus, cui summum bonum est
  idem, quod vos interdum vel potius nimium saepe dicitis, nihil dolere. Non
  enim, si malum est dolor, carere eo malo satis est ad bene vivendum. Hoc
  dixerit potius Ennius: \textquoteleft Nimium boni est, cui nihil est
  mali\textquoteright. Nos beatam vitam non depulsione mali, sed adeptione boni
  iudicemus, nec eam cessando, sive gaudentem, ut Aristippus, sive non dolentem,
  ut hic, sed agendo aliquid considerandove quaeramus.}
 %{122}
\NewLipsumPar{Quae possunt eadem contra Carneadeum illud summum bonum dici,
  quod is non tam, ut probaret, protulit, quam ut Stoicis, quibuscum bellum
  gerebat, opponeret. Id autem eius modi est, ut additum ad virtutem
  auctoritatem videatur habiturum et expleturum cumulate vitam beatam, de quo
  omnis haec quaestio est. Nam qui ad virtutem adiungunt vel voluptatem, quam
  unam virtus minimi facit, vel vacuitatem doloris, quae etiamsi malo caret,
  tamen non est summum bonum, accessione utuntur non ita probabili, nec tamen,
  cur id tam parce tamque restricte faciant, intellego. Quasi enim emendum eis
  sit, quod addant ad virtutem, primum vilissimas res addunt, dein singulas
  potius, quam omnia, quae prima natura approbavisset, ea cum honestate
  coniungerent.}
 %{123}
\NewLipsumPar{Quae quod Aristoni et Pyrrhoni omnino visa sunt pro nihilo, ut
  inter optime valere et gravissime aegrotare nihil prorsus dicerent interesse,
  recte iam pridem contra eos desitum est disputari. Dum enim in una virtute sic
  omnia esse voluerunt, ut eam rerum selectione expoliarent nec ei quicquam, aut
  unde oriretur, darent, aut ubi niteretur, virtutem ipsam, quam amplexabantur,
  sustulerunt. Erillus autem ad scientiam omnia revocans unum quoddam bonum
  vidit, sed nec optimum nec quo vita gubernari possit. Itaque hic ipse iam
  pridem est reiectus; post enim Chrysippum (eum) non sane est disputatum.
  Restatis igitur vos; nam cum Academicis incerta luctatio est, qui nihil
  affirmant et quasi desperata cognitione certi id sequi volunt, quodcumque veri
  simile videatur.}
 %{124}
\NewLipsumPar{Cum Epicuro autem hoc plus est negotii, quod e duplici genere
  voluptatis coniunctus est, quodque et ipse et amici eius et multi postea
  defensores eius sententiae fuerunt, et nescio quo modo, is qui auctoritatem
  minimam habet, maximam vim, populus cum illis facit. Quos nisi redarguimus,
  omnis virtus, omne decus, omnis vera laus deserenda est. Ita ceterorum
  sententiis semotis relinquitur non mihi cum Torquato, sed virtuti cum
  voluptate certatio. Quam quidem certationem homo et acutus et diligens,
  Chrysippus, non contemnit totumque discrimen summi boni in earum comparatione
  positum putat. Ego autem existimo, si honestum esse aliquid ostendero, quod
  sit ipsum vi sua propter seque expetendum, iacere vestra omnia. Itaque eo,
  quale sit, breviter, ut tempus postulat, constituto accedam ad omnia tua,
  Torquate, nisi memoria forte defecerit.}
 %{125}
\NewLipsumPar{Honestum igitur id intellegimus, quod tale est, ut detracta omni
  utilitate sine ullis praemiis fructibusve per se ipsum possit iure laudari.
  Quod quale sit, non tam definitione, qua sum usus, intellegi potest, quamquam
  aliquantum potest, quam communi omnium iudicio et optimi cuiusque studiis
  atque factis, qui permulta ob eam unam causam faciunt, quia decet, quia
  rectum, quia honestum est, etsi nullum consecuturum emolumentum vident.
  Homines enim, etsi aliis multis, tamen hoc uno plurimum a bestiis differunt,
  quod rationem habent a natura datam mentemque acrem et vigentem celerrimeque
  multa simul agitantem et, ut ita dicam, sagacem, quae et causas rerum et
  consecutiones videat et similitudines transferat et disiuncta coniungat et cum
  praesentibus futura copulet omnemque complectatur vitae consequentis statum.
  Eademque ratio fecit hominem hominum adpetentem cumque iis natura et sermone
  et usu congruentem, ut profectus a caritate domesticorum ac suorum serpat
  longius et se implicet primum civium, deinde omnium mortalium societate atque,
  ut ad Archytam scripsit Plato, non sibi se soli natum meminerit, sed patriae,
  sed suis, ut perexigua pars ipsi relinquatur.}
 %{126}
\NewLipsumPar{Et quoniam eadem natura cupiditatem ingenuit homini veri
  videndi, quod facillime apparet, cum vacui curis etiam quid in caelo fiat
  scire avemus, his initiis inducti omnia vera diligimus, id est fidelia,
  simplicia, constantia, tum vana, falsa, fallentia odimus, ut fraudem,
  periurium, malitiam, iniuriam. Eadem ratio habet in se quiddam amplum atque
  magnificum, ad imperandum magis quam ad parendum accommodatum, omnia humana
  non tolerabilia solum, sed etiam levia ducens, altum quiddam et excelsum,
  nihil timens, nemini cedens, semper invictum.}
 %{127}
\NewLipsumPar{Atque his tribus generibus honestorum notatis quartum sequitur
  et in eadem pulchritudine et aptum ex illis tribus, in quo inest ordo et
  moderatio. Cuius similitudine perspecta in formarum specie ac dignitate
  transitum est ad honestatem dictorum atque factorum. Nam ex his tribus
  laudibus, quas ante dixi, et temeritatem reformidat et non audet cuiquam aut
  dicto protervo aut facto nocere vereturque quicquam aut facere aut eloqui,
  quod parum virile videatur.}
 %{128}
\NewLipsumPar{Habes undique expletam et perfectam, Torquate, formam
  honestatis, quae tota quattuor his virtutibus, quae a te quoque commemoratae
  sunt, continetur. Hanc se tuus Epicurus omnino ignorare dicit quam aut qualem
  esse velint qui honestate summum bonum metiantur. Si enim ad honestatem omnia
  referant neque in ea voluptatem dicant inesse, ait eos voce inani
  sonare\textendash\space his enim ipsis verbis utitur\textendash neque
  intellegere nec videre sub hanc vocem honestatis quae sit subicienda
  sententia. Ut enim consuetudo loquitur, id solum dicitur honestum, quod est
  populari fama gloriosum. \textquoteleft Quod\textquoteright, inquit,
  \textquoteleft quamquam voluptatibus quibusdam est saepe iucundius, tamen
  expetitur propter voluptatem.\textquoteright}
 %{129}
\NewLipsumPar{Videsne quam sit magna dissensio? Philosophus nobilis, a quo non
  solum Graecia et Italia, sed etiam omnis barbaria commota est, honestum quid
  sit, si id non sit in voluptate, negat se intellegere, nisi forte illud, quod
  multitudinis rumore laudetur. Ego autem hoc etiam turpe esse saepe iudico et,
  si quando turpe non sit, tum esse non turpe, cum id a multitudine laudetur,
  quod sit ipsum per se rectum atque laudabile, non ob eam causam tamen illud
  dici esse honestum, quia laudetur a multis, sed quia tale sit, ut, vel si
  ignorarent id homines, vel si obmutuissent, sua tamen pulchritudine esset
  specieque laudabile. Itaque idem natura victus, cui obsisti non potest, dicit
  alio loco id, quod a te etiam paulo ante dictum est, non posse iucunde vivi
  nisi etiam honeste.}
 %{130}
\NewLipsumPar{Quid nunc \textquoteleft honeste\textquoteright\space dicit?
  Idemne, quod iucunde? Ergo ita: non posse honeste vivi, nisi honeste vivatur?
  An nisi populari fama? Sine ea igitur iucunde negat posse (se) vivere? Quid
  turpius quam sapientis vitam ex insipientium sermone pendere? Quid ergo hoc
  loco intellegit honestum? Certe nihil nisi quod possit ipsum propter se iure
  laudari. Nam si propter voluptatem, quae est ista laus, quae possit e macello
  peti? Non is vir est, ut, cum honestatem eo loco habeat, ut sine ea iucunde
  neget posse vivi, illud honestum, quod populare sit, sentiat et sine eo neget
  iucunde vivi posse, aut quicquam aliud honestum intellegat, nisi quod sit
  rectum ipsumque per se sua vi, sua natura, sua sponte laudabile.}
 %{131}
\NewLipsumPar{Itaque, Torquate, cum diceres clamare Epicurum non posse iucunde
  vivi, nisi honeste et sapienter et iuste viveretur, tu ipse mihi gloriari
  videbare. Tanta vis inerat in verbis propter earum rerum, quae significabantur
  his verbis, dignitatem, ut altior fieres, ut interdum insisteres, ut nos
  intuens quasi testificarere laudari honestatem et iustitiam aliquando ab
  Epicuro. Quam te decebat iis verbis uti, quibus si philosophi non uterentur,
  philosophia omnino non egeremus! Istorum enim verborum amore, quae perraro
  appellantur ab Epicuro, sapientiae, fortitudinis, iustitiae, temperantiae,
  praestantissimis ingeniis homines se ad philosophiae studium contulerunt.}
 %{132}
\NewLipsumPar{\textquoteleft Oculorum\textquoteright, inquit Plato,
  \textquoteleft est in nobis sensus acerrimus, quibus sapientiam non cernimus.
  Quam illa ardentis amores excitaret sui!\textquoteright\space Cur tandem? An
  quod ita callida est, ut optime possit architectari voluptates? Cur iustitia
  laudatur? aut unde est hoc contritum vetustate proverbium: \textquoteleft
  quicum in tenebris\textquoteright? Hoc dictum in una re latissime patet, ut in
  omnibus factis re, non teste moveamur.}
 %{133}
\NewLipsumPar{Sunt enim levia et perinfirma, quae dicebantur a te, animi
  conscientia improbos excruciari, tum etiam poenae timore, qua aut afficiantur
  aut semper sint in metu ne afficiantur aliquando. Non oportet timidum aut
  inbecillo animo fingi non bonum illum virum, qui, quicquid fecerit, ipse se
  cruciet omniaque formidet, sed omnia callide referentem ad utilitatem, acutum,
  versutum, veteratorem, facile ut excogitet quo modo occulte, sine teste, sine
  ullo conscio fallat.}
 %{134}
\NewLipsumPar{An tu me de L. Tubulo putas dicere? Qui cum praetor quaestionem
  inter sicarios exercuisset, ita aperte cepit pecunias ob rem iudicandam, ut
  anno proximo P. Scaevola tribunus plebis ferret ad plebem vellentne de ea re
  quaeri. Quo plebiscito decreta a senatu est consuli quaestio Cn.\ Caepioni.
  Profectus in exilium Tubulus statim nec respondere ausus; erat enim res
  aperta. Non igitur de improbo, sed (de) callido improbo quaerimus, qualis Q.
  Pompeius in foedere Numantino infitiando fuit, nec vero omnia timente, sed
  primum qui animi conscientiam non curet, quam scilicet comprimere nihil est
  negotii. Is enim, qui occultus et tectus dicitur, tantum abest ut se indicet,
  perficiet etiam ut dolere alterius improbe facto videatur. Quid est enim aliud
  esse versutum?}
 %{135}
\NewLipsumPar{Memini me adesse P. Sextilio Rufo, cum is rem ad amicos ita
  deferret, se esse heredem Q. Fadio Gallo, cuius in testamento scriptum esset
  se ab eo rogatum ut omnis hereditas ad filiam perveniret. Id Sextilius factum
  negabat. Poterat autem inpune; quis enim redargueret? Nemo nostrum credebat,
  eratque veri similius hunc mentiri, cuius interesset, quam illum, qui id se
  rogasse scripsisset, quod debuisset rogare. Addebat etiam se in legem Voconiam
  iuratum contra eam facere non audere, nisi aliter amicis videretur. Aderamus
  nos quidem adolescentes, sed multi amplissimi viri, quorum nemo censuit plus
  Fadiae dandum, quam posset ad eam lege Voconia pervenire. Tenuit permagnam
  Sextilius hereditatem, unde, si secutus esset eorum sententiam, qui honesta et
  recta emolumentis omnibus et commodis anteponerent, nummum nullum attigisset.
  Num igitur eum postea censes anxio animo aut sollicito fuisse? Nihil minus,
  contraque illa hereditate dives ob eamque rem laetus. Magni enim aestimabat
  pecuniam non modo non contra leges, sed etiam legibus partam. Quae quidem vel
  cum periculo est quaerenda vobis; est enim effectrix multarum et magnarum
  voluptatum.}
 %{136}
\NewLipsumPar{Ut igitur illis, qui, recta et honesta quae sunt, ea statuunt
  per se expetenda, adeunda sunt saepe pericula decoris honestatisque causa, sic
  vestris, qui omnia voluptate metiuntur, pericula adeunda sunt, ut adipiscantur
  magnas voluptates. Si magna res, magna hereditas agetur, cum pecunia
  voluptates pariantur plurimae, idem erit Epicuro vestro faciendum, si suum
  finem bonorum sequi volet, quod Scipioni magna gloria proposita, si Hannibalem
  in Africam retraxisset. Itaque quantum adiit periculum! Ad honestatem enim
  illum omnem conatum suum referebat, non ad voluptatem. Sic vester sapiens
  magno aliquo emolumento commotus cicuta, si opus erit, dimicabit.}
 %{137}
\NewLipsumPar{Occultum facinus esse potuerit, gaudebit; deprehensus omnem
  poenam contemnet. Erit enim instructus ad mortem contemnendam, ad exilium, ad
  ipsum etiam dolorem. Quem quidem vos, cum improbis poenam proponitis,
  inpetibilem facitis, cum sapientem semper boni plus habere vultis,
  tolerabilem. Sed finge non solum callidum eum, qui aliquid improbe faciat,
  verum etiam praepotentem, ut M. Crassus fuit, qui tamen solebat uti suo bono,
  ut hodie est noster Pompeius, cui recte facienti gratia est habenda; esse enim
  quam vellet iniquus iustus poterat inpune. Quam multa vero iniuste fieri
  possunt, quae nemo possit reprehendere!}
 %{138}
\NewLipsumPar{Si te amicus tuus moriens rogaverit, ut hereditatem reddas suae
  filiae, nec usquam id scripserit, ut scripsit Fadius, nec cuiquam dixerit,
  quid facies? Tu quidem reddes; ipse Epicurus fortasse redderet, ut Sextus
  Peducaeus, Sex. F., is qui hunc nostrum reliquit effigiem et humanitatis et
  probitatis suae filium, cum doctus, tum omnium vir optimus et iustissimus, cum
  sciret nemo eum rogatum a Caio Plotio, equite Romano splendido, Nursino, ultro
  ad mulierem venit eique nihil opinanti viri mandatum euit hereditatemque
  reddidit. Sed ego ex te quaero, quoniam idem tu certe fecisses, nonne
  intellegas eo maiorem vim esse naturae, quod ipsi vos, qui omnia ad vestrum
  commodum et, ut ipsi dicitis, ad voluptatem referatis, tamen ea faciatis, e
  quibus appareat non voluptatem vos, sed officium sequi, plusque rectam naturam
  quam rationem pravam valere.}
 %{139}
\NewLipsumPar{Si scieris, inquit Carneades, aspidem occulte latere uspiam, et
  velle aliquem inprudentem super eam assidere, cuius mors tibi emolumentum
  futura sit, improbe feceris, nisi monueris ne assidat, sed inpunite tamen;
  scisse enim te quis coarguere possit? Sed nimis multa. Perspicuum est enim,
  nisi aequitas, fides, iustitia proficiscantur a natura, et si omnia haec ad
  utilitatem referantur, virum bonum non posse reperiri; deque his rebus satis
  multa in nostris de re publica libris sunt dicta a Laelio.}
 %{140}
\NewLipsumPar{Transfer idem ad modestiam vel temperantiam, quae est moderatio
  cupiditatum rationi oboediens. Satisne ergo pudori consulat, si quis sine
  teste libidini pareat? An est aliquid per se ipsum flagitiosum, etiamsi nulla
  comitetur infamia? Quid? Fortes viri voluptatumne calculis subductis proelium
  ineunt, sanguinem pro patria profundunt, an quodam animi ardore atque impetu
  concitati? utrum tandem censes, Torquate, Imperiosum illum, si nostra verba
  audiret, tuamne de se orationem libentius auditurum fuisse an meam, cum ego
  dicerem nihil eum fecisse sua causa omniaque rei publicae, tu contra nihil
  nisi sua? Si vero id etiam explanare velles apertiusque diceres nihil eum
  fecisse nisi voluptatis causa, quo modo eum tandem laturum fuisse existimas?}
 %{141}
\NewLipsumPar{Esto, fecerit, si ita vis, Torquatus propter suas
  utilitates\textendash\space malo enim dicere quam voluptates, in tanto
  praesertim viro\textendash, num etiam eius collega P. Decius, princeps in ea
  familia consulatus, cum se devoverat et equo admisso in mediam aciem Latinorum
  irruebat, aliquid de voluptatibus suis cogitabat? Ubi ut eam caperet aut
  quando? Cum sciret confestim esse moriendum eamque mortem ardentiore studio
  peteret, quam Epicurus voluptatem petendam putat. Quod quidem eius factum nisi
  esset iure laudatum, non esset imitatus quarto consulatu suo filius, neque
  porro ex eo natus cum Pyrrho bellum gerens consul cecidisset in proelio seque
  e continenti genere tertiam victimam rei publicae praebuisset.}
 %{142}
\NewLipsumPar{Contineo me ab exemplis. Graecis hoc modicum est: Leonidas,
  Epaminondas, tres aliqui aut quattuor; ego si nostros colligere coepero,
  perficiam illud quidem, ut se virtuti tradat constringendam voluptas, sed dies
  me deficiet, et, ut Aulus Varius, qui est habitus iudex durior, dicere
  consessori solebat, cum datis testibus alii tamen citarentur: \textquoteleft
  Aut hoc testium satis est, aut nescio, quid satis sit,\textquoteright\space
  sic a me satis datum est testium. Quid enim? Te ipsum, dignissimum maioribus
  tuis, voluptasne induxit, ut adolescentulus eriperes P. Sullae consulatum?
  Quem cum ad patrem tuum rettulisses, fortissimum virum, qualis ille vel consul
  vel civis cum semper, tum post consulatum fuit! Quo quidem auctore nos ipsi ea
  gessimus, ut omnibus potius quam ipsis nobis consuluerimus.}
 %{143}
\NewLipsumPar{At quam pulchre dicere videbare, cum ex altera parte ponebas
  cumulatum aliquem plurimis et maximis voluptatibus nullo nec praesenti nec
  futuro dolore, ex altera autem cruciatibus maximis toto corpore nulla nec
  adiuncta nec sperata voluptate, et quaerebas, quis aut hoc miserior aut
  superiore illo beatior; deinde concludebas summum malum esse dolorem, summum
  bonum voluptatem! Lucius Thorius Balbus fuit, Lanuvinus, quem meminisse tu non
  potes. Is ita vivebat, ut nulla tam exquisita posset inveniri voluptas, qua
  non abundaret. Erat et cupidus voluptatum et eius generis intellegens et
  copiosus, ita non superstitiosus, ut illa plurima in sua patria sacrificia et
  fana contemneret, ita non timidus ad mortem, ut in acie sit ob rem publicam
  interfectus.}
 %{144}
\NewLipsumPar{Cupiditates non Epicuri divisione finiebat, sed sua satietate.
  Habebat tamen rationem valitudinis: utebatur iis exercitationibus, ut ad cenam
  et sitiens et esuriens veniret, eo cibo, qui et suavissimus esset et idem
  facillimus ad concoquendum, vino et ad voluptatem et ne noceret. Cetera illa
  adhibebat, quibus demptis negat se Epicurus intellegere quid sit bonum. Aberat
  omnis dolor, qui si adesset, nec molliter ferret et tamen medicis plus quam
  philosophis uteretur. Color egregius, integra valitudo, summa gratia, vita
  denique conferta voluptatum omnium varietate.}
 %{145}
\NewLipsumPar{Hunc vos beatum; ratio quidem vestra sic cogit. At ego quem huic
  anteponam non audeo dicere; dicet pro me ipsa virtus nec dubitabit isti vestro
  beato M. Regulum anteponere, quem quidem, cum sua voluntate, nulla vi coactus
  praeter fidem, quam dederat hosti, ex patria Karthaginem revertisset, tum
  ipsum, cum vigiliis et fame cruciaretur, clamat virtus beatiorem fuisse quam
  potantem in rosa Thorium. Bella magna gesserat, bis consul fuerat, triumpharat
  nec tamen sua illa superiora tam magna neque tam praeclara ducebat quam illum
  ultimum casum, quem propter fidem constantiamque susceperat, qui nobis
  miserabilis videtur audientibus, illi perpetienti erat voluptarius. Non enim
  hilaritate nec lascivia nec risu aut ioco, comite levitatis, saepe etiam
  tristes firmitate et constantia sunt beati.}
 %{146}
\NewLipsumPar{Stuprata per vim Lucretia a regis filio testata civis se ipsa
  interemit. Hic dolor populi Romani duce et auctore Bruto causa civitati
  libertatis fuit, ob eiusque mulieris memoriam primo anno et vir et pater eius
  consul est factus. Tenuis Lucius Verginius unusque de multis sexagesimo anno
  post libertatem receptam virginem filiam sua manu occidit potius, quam ea
  Ap.\space Claudii libidini, qui tum erat summo (ne) imperio, dederetur.}
 %{147}
\NewLipsumPar{Aut haec tibi, Torquate, sunt vituperanda aut patrocinium
  voluptatis repudiandum. Quod autem patrocinium aut quae ista causa est
  voluptatis, quae nec testes ullos e claris viris nec laudatores poterit
  adhibere? Ut enim nos ex annalium monimentis testes excitamus eos, quorum
  omnis vita consumpta est in laboribus gloriosis, qui voluptatis nomen audire
  non possent, sic in vestris disputationibus historia muta est. Numquam audivi
  in Epicuri schola Lycurgum, Solonem, Miltiadem, Themistoclem, Epaminondam
  nominari, qui in ore sunt ceterorum omnium philosophorum. Nunc vero, quoniam
  haec nos etiam tractare coepimus, suppeditabit nobis Atticus noster e
  thesauris suis quos et quantos viros!}
 %{148}
\NewLipsumPar{Nonne melius est de his aliquid quam tantis voluminibus de
  Themista loqui? Sint ista Graecorum; quamquam ab iis philosophiam et omnes
  ingenuas disciplinas habemus; sed tamen est aliquid, quod nobis non liceat,
  liceat illis. Pugnant Stoici cum Peripateticis. Alteri negant quicquam esse
  bonum, nisi quod honestum sit, alteri plurimum se et longe longeque plurimum
  tribuere honestati, sed tamen et in corpore et extra esse quaedam bona. Et
  certamen honestum et disputatio splendida! Omnis est enim de virtutis
  dignitate contentio. At cum tuis cum disseras, multa sunt audienda etiam de
  obscenis voluptatibus, de quibus ab Epicuro saepissime dicitur.}
 %{149}
\NewLipsumPar{Non potes ergo ista tueri, Torquate, mihi crede, si te ipse et
  tuas cogitationes et studia perspexeris; pudebit te, inquam, illius tabulae,
  quam Cleanthes sane commode verbis depingere solebat. Iubebat eos, qui
  audiebant, secum ipsos cogitare pictam in tabula Voluptatem pulcherrimo
  vestitu et ornatu regali in solio sedentem, praesto esse Virtutes ut
  ancillulas, quae nihil aliud agerent, nullum suum officium ducerent, nisi ut
  Voluptati ministrarent et eam tantum ad aurem admonerent, si modo id pictura
  intellegi posset, ut caveret ne quid faceret inprudens, quod offenderet animos
  hominum, aut quicquam, e quo oriretur aliquis dolor. \textquoteleft Nos quidem
  Virtutes sic natae sumus, ut tibi serviremus, aliud negotii nihil
  habemus.\textquoteright}
 %{150}
\NewLipsumPar{At negat Epicurus\textendash hoc enim vestrum lumen
  est\textendash\space quemquam, qui honeste non vivat, iucunde posse vivere.
  Quasi ego id curem, quid ille aiat aut neget. Illud quaero, quid ei, qui in
  voluptate summum bonum ponat, consentaneum sit dicere. Quid affers, cur
  Thorius, cur Caius Postumius, cur omnium horum magister, Orata, non
  iucundissime vixerit? Ipse negat, ut ante dixi, luxuriosorum vitam
  reprehendendam, nisi plane fatui sint, id est nisi aut cupiant aut metuant.
  Quarum ambarum rerum cum medicinam pollicetur, luxuriae licentiam pollicetur.
  His enim rebus detractis negat se reperire in asotorum vita quod reprehendat.}
 %{151}
\NewLipsumPar{Non igitur potestis voluptate omnia dirigentes aut tueri aut
  retinere virtutem. Nam nec vir bonus ac iustus haberi debet qui, ne malum
  habeat, abstinet se ab iniuria. Nosti, credo, illud: \textquoteleft Nemo pius
  est, qui pietatem\textendash\textquoteright; cave putes quicquam esse verius.
  Nec enim, dum metuit, iustus est, et certe, si metuere destiterit, non erit;
  non metuet autem, sive celare poterit, sive opibus magnis quicquid fecerit
  optinere, certeque malet existimari bonus vir, ut non sit, quam esse, ut non
  putetur. Ita, quod certissimum est, pro vera certaque iustitia simulationem
  nobis iustitiae traditis praecipitisque quodam modo ut nostram stabilem
  conscientiam contemnamus, aliorum errantem opinionem aucupemur.}
 %{152}
\NewLipsumPar{Quae dici eadem de ceteris virtutibus possunt, quarum omnium
  fundamenta vos in voluptate tamquam in aqua ponitis. Quid enim? Fortemne
  possumus dicere eundem illum Torquatum?\textendash Delector enim, quamquam te
  non possum, ut ais, corrumpere, delector, inquam, et familia vestra et nomine.
  Et hercule mihi vir optimus nostrique amantissimus, Aulus Torquatus, versatur
  ante oculos, cuius quantum studium et quam insigne fuerit erga me temporibus
  illis, quae nota sunt omnibus, scire necesse est utrumque vestrum. Quae mihi
  ipsi, qui volo et esse et haberi gratus, grata non essent, nisi eum
  perspicerem mea causa mihi amicum fuisse, non sua, nisi hoc dicis sua, quod
  interest omnium recte facere. Si id dicis, vicimus. Id enim volumus, id
  contendimus, ut officii fructus sit ipsum officium.}
 %{153}
\NewLipsumPar{Hoc ille tuus non vult omnibusque ex rebus voluptatem quasi
  mercedem exigit. Sed ad illum redeo. Si voluptatis causa cum Gallo apud
  Anienem depugnavit provocatus et ex eius spoliis sibi et torquem et cognomen
  induit ullam aliam ob causam, nisi quod ei talia facta digna viro videbantur,
  fortem non puto. Iam si pudor, si modestia, si pudicitia, si uno verbo
  temperantia poenae aut infamiae metu coercebuntur, non sanctitate sua se
  tuebuntur, quod adulterium, quod stuprum, quae libido non se proripiet ac
  proiciet aut occultatione proposita aut inpunitate aut licentia? Quid?}
 %{154}
\NewLipsumPar{Illud, Torquate, quale tandem videtur, te isto nomine, ingenio,
  gloria, quae facis, quae cogitas, quae contendis quo referas, cuius rei causa
  perficere quae conaris velis, quid optimum denique in vita iudices non audere
  in conventu dicere? Quid enim mereri velis, iam cum magistratum inieris et in
  contionem ascenderis\textendash est enim tibi edicendum quae sis observaturus
  in iure dicendo, et fortasse etiam, si tibi erit visum, aliquid de maioribus
  tuis et de te ipso dices more maiorum\textendash, quid merearis igitur, ut
  dicas te in eo magistratu omnia voluptatis causa facturum esse, teque nihil
  fecisse in vita nisi voluptatis causa? \textquoteleft An me\textquoteright,
  inquis, \textquoteleft tam amentem putas, ut apud imperitos isto modo
  loquar?\textquoteright\space At tu eadem ista dic in iudicio aut, si coronam
  times, dic in senatu. Numquam facies. Cur, nisi quod turpis oratio est? Mene
  ergo et Triarium dignos existimas, apud quos turpiter loquare?}
 %{155}
\NewLipsumPar{Verum esto: verbum ipsum voluptatis non habet dignitatem, nec
  nos fortasse intellegimus. Hoc enim identidem dicitis, non intellegere nos
  quam dicatis voluptatem. Rem videlicet difficilem et obscuram! Individua cum
  dicitis et intermundia, quae nec sunt ulla nec possunt esse, intellegimus,
  voluptas, quae passeribus omnibus nota est, a nobis intellegi non potest?
  Quid, si efficio ut fateare me non modo quid sit voluptas scire\textendash est
  enim iucundus motus in sensu\textendash, sed etiam quid eam tu velis esse? Tum
  enim eam ipsam vis, quam modo ego dixi, et nomen inponis, in motu ut sit et
  faciat aliquam varietatem, tum aliam quandam summam voluptatem, quo addi nihil
  possit; eam tum adesse, cum dolor omnis absit; eam stabilem appellas. Sit sane
  ista voluptas.}
 %{156}
\NewLipsumPar{Dic in quovis conventu te omnia facere, ne doleas. Si ne hoc
  quidem satis ample, satis honeste dici putas, dic te omnia et in isto
  magistratu et in omni vita utilitatis tuae causa facturum, nihil nisi quod
  expediat, nihil denique nisi tua causa: quem clamorem contionis aut quam spem
  consulatus eius, qui tibi paratissimus est, futuram putas? Eamne rationem
  igitur sequere, qua tecum ipse et cum tuis utare, profiteri et in medium
  proferre non audeas? At vero illa, quae Peripatetici, quae Stoici dicunt,
  semper tibi in ore sunt in iudiciis, in senatu. Officium, aequitatem,
  dignitatem, fidem, recta, honesta, digna imperio, digna populo Romano, omnia
  pericula pro re publica, mori pro patria, haec cum loqueris, nos barones
  stupemus, tu videlicet tecum ipse rides.}
 %{157}
\NewLipsumPar{Nam inter ista tam magnifica verba tamque praeclara non habet
  ullum voluptas locum, non modo illa, quam in motu esse dicitis, quam omnes
  urbani rustici, omnes, inquam, qui Latine loquuntur, voluptatem vocant, sed ne
  haec quidem stabilis, quam praeter vos nemo appellat voluptatem. Vide igitur
  ne non debeas verbis nostris uti, sententiis tuis. Quodsi vultum tibi, si
  incessum fingeres, quo gravior viderere, non esses tui similis; verba tu
  fingas et ea dicas, quae non sentias? Aut etiam, ut vestitum, sic sententiam
  habeas aliam domesticam, aliam forensem, ut in fronte ostentatio sit, intus
  veritas occultetur? Vide, quaeso, rectumne sit. Mihi quidem eae verae videntur
  opiniones, quae honestae, quae laudabiles, quae gloriosae, quae in senatu,
  quae apud populum, quae in omni coetu concilioque profitendae sint, ne id non
  pudeat sentire, quod pudeat dicere.}
 %{158}
\NewLipsumPar{Amicitiae vero locus ubi esse potest aut quis amicus esse
  cuiquam, quem non ipsum amet propter ipsum? Quid autem est amare, e quo nomen
  ductum amicitiae est, nisi velle bonis aliquem affici quam maximis, etiamsi ad
  se ex iis nihil redundet? \textquoteleft Prodest\textquoteright, inquit,
  \textquoteleft mihi eo esse animo.\textquoteright\space Immo videri fortasse.
  Esse enim, nisi eris, non potes. Qui autem esse poteris, nisi te amor ipse
  ceperit? quod non subducta utilitatis ratione effici solet, sed ipsum a se
  oritur et sua sponte nascitur. \textquoteleft At enim sequor
  utilitatem.\textquoteright\space Manebit ergo amicitia tam diu, quam diu
  sequetur utilitas, et, si utilitas amicitiam constituet, tollet eadem.}
 %{159}
\NewLipsumPar{Sed quid ages tandem, si utilitas ab amicitia, ut fit saepe,
  defecerit? relinquesne? Quae ista amicitia est? Retinebis? Qui convenit? Quid
  enim de amicitia statueris utilitatis causa expetenda vides. \textquoteleft Ne
  in odium veniam, si amicum destitero tueri.\textquoteright\space Primum cur
  ista res digna odio est, nisi quod est turpis? Quodsi, ne quo incommodo
  afficiare, non relinques amicum, tamen, ne sine fructu alligatus sis, ut
  moriatur optabis. Quid, si non modo utilitatem tibi nullam afferet, sed
  iacturae rei familiaris erunt faciendae, labores suscipiendi, adeundum vitae
  periculum? Ne tum quidem te respicies et cogitabis sibi quemque natum esse et
  suis voluptatibus? Vadem te ad mortem tyranno dabis pro amico, ut Pythagoreus
  ille Siculo fecit tyranno? Aut, Pylades cum sis, dices te esse Orestem, ut
  moriare pro amico? Aut, si esses Orestes, Pyladem refelleres, te indicares et,
  si id non probares, quo minus ambo una necaremini non precarere?}
 %{160}
\NewLipsumPar{Faceres tu quidem, Torquate, haec omnia; nihil enim arbitror
  esse magna laude dignum, quod te praetermissurum credam aut mortis aut doloris
  metu. Non quaeritur autem quid naturae tuae consentaneum sit, sed quid
  disciplinae. Ratio ista, quam defendis, praecepta, quae didicisti, quae
  probas, funditus evertunt amicitiam, quamvis eam Epicurus, ut facit, in caelum
  efferat laudibus. At coluit ipse amicitias. Quis, quaeso, illum negat et bonum
  virum et comem et humanum fuisse? De ingenio eius in his disputationibus, non
  de moribus quaeritur. Sit ista in Graecorum levitate perversitas, qui
  maledictis insectantur eos, a quibus de veritate dissentiunt. Sed quamvis
  comis in amicis tuendis fuerit, tamen, si haec vera sunt\textendash nihil enim
  affirmo\textendash, non satis acutus fuit.}
 %{161}
\NewLipsumPar{At multis se probavit. Et quidem iure fortasse, sed tamen non
  gravissimum est testimonium multitudinis. In omni enim arte vel studio vel
  quavis scientia vel in ipsa virtute optimum quidque rarissimum est. Ac mihi
  quidem, quod et ipse bonus vir fuit et multi Epicurei et fuerunt et hodie sunt
  et in amicitiis fideles et in omni vita constantes et graves nec voluptate,
  sed officio consilia moderantes, hoc videtur maior vis honestatis et minor
  voluptatis. Ita enim vivunt quidam, ut eorum vita refellatur oratio. Atque ut
  ceteri dicere existimantur melius quam facere, sic hi mihi videntur facere
  melius quam dicere.}
 %{162}
\NewLipsumPar{Sed haec nihil sane ad rem; illa videamus, quae a te de amicitia
  dicta sunt. E quibus unum mihi videbar ab ipso Epicuro dictum cognoscere,
  amicitiam a voluptate non posse divelli ob eamque rem colendam esse, quod,
  (quoniam) sine ea tuto et sine metu vivi non posset, ne iucunde quidem posset.
  Satis est ad hoc responsum. Attulisti aliud humanius horum recentiorum,
  numquam dictum ab ipso illo, quod sciam, primo utilitatis causa amicum expeti,
  cum autem usus accessisset, tum ipsum amari per se etiam omissa spe
  voluptatis. Hoc etsi multimodis reprehendi potest, tamen accipio, quod dant.
  Mihi enim satis est, ipsis non satis. Nam aliquando posse recte fieri dicunt
  nulla expectata nec quaesita voluptate.}
 %{163}
\NewLipsumPar{Posuisti etiam dicere alios foedus quoddam inter se facere
  sapientis, ut, quem ad modum sint in se ipsos animati, eodem modo sint erga
  amicos; id et fieri posse et saepe esse factum et ad voluptates percipiendas
  maxime pertinere. Hoc foedus facere si potuerunt, faciant etiam illud, ut
  aequitatem, modestiam, virtutes omnes per se ipsas gratis diligant. An vero,
  si fructibus et emolumentis et utilitatibus amicitias colemus, si nulla
  caritas erit, quae faciat amicitiam ipsam sua sponte, vi sua, ex se et propter
  se expetendam, dubium est, quin fundos et insulas amicis anteponamus?}
 %{164}
\NewLipsumPar{Licet hic rursus ea commemores, quae optimis verbis ab Epicuro
  de laude amicitiae dicta sunt. Non quaero, quid dicat, sed quid convenienter
  possit rationi et sententiae suae dicere. \textquoteleft Utilitatis causa
  amicitia est quaesita.\textquoteright\space Num igitur utiliorem tibi hunc
  Triarium putas esse posse, quam si tua sint Puteolis granaria? Collige omnia,
  quae soletis: \textquoteleft Praesidium amicorum.\textquoteright\space Satis
  est tibi in te, satis in legibus, satis in mediocribus amicitiis praesidii.
  Iam contemni non poteris. Odium autem et invidiam facile vitabis. Ad eas enim
  res ab Epicuro praecepta dantur. Et tamen tantis vectigalibus ad liberalitatem
  utens etiam sine hac Pyladea amicitia multorum te benivolentia praeclare
  tuebere et munies.}
 %{165}
\NewLipsumPar{\textquoteleft At quicum ioca seria, ut dicitur, quicum arcana,
  quicum occulta omnia?\textquoteright\space Tecum optime, deinde etiam cum
  mediocri amico. Sed fac ista esse non inportuna; quid ad utilitatem tantae
  pecuniae? Vides igitur, si amicitiam sua caritate metiare, nihil esse
  praestantius, sin emolumento, summas familiaritates praediorum fructuosorum
  mercede superari. Me igitur ipsum ames oportet, non mea, si veri amici futuri
  sumus. Sed in rebus apertissimis nimium longi sumus. Perfecto enim et concluso
  neque virtutibus neque amicitiis usquam locum esse, si ad voluptatem omnia
  referantur, nihil praeterea est magnopere dicendum. Ac tamen, ne cui loco non
  videatur esse responsum, pauca etiam nunc dicam ad reliquam orationem tuam.}
 %{166}
\NewLipsumPar{Quoniam igitur omnis summa philosophiae ad beate vivendum
  refertur, idque unum expetentes homines se ad hoc studium contulerunt, beate
  autem vivere alii in alio, vos in voluptate ponitis, item contra miseriam
  omnem in dolore, id primum videamus, beate vivere vestrum quale sit. Atque hoc
  dabitis, ut opinor, si modo sit aliquid esse beatum, id oportere totum poni in
  potestate sapientis. Nam si amitti vita beata potest, beata esse non potest.
  Quis enim confidit semper sibi illud stabile et firmum permansurum, quod
  fragile et caducum sit? Qui autem diffidet perpetuitati bonorum suorum, timeat
  necesse est, ne aliquando amissis illis sit miser. Beatus autem esse in
  maximarum rerum timore nemo potest.}
 %{167}
\NewLipsumPar{Nemo igitur esse beatus potest. Neque enim in aliqua parte, sed
  in perpetuitate temporis vita beata dici solet, nec appellatur omnino vita,
  nisi confecta atque absoluta, nec potest quisquam alias beatus esse, alias
  miser; qui enim existimabit posse se miserum esse beatus non erit. Nam cum
  suscepta semel est beata vita, tam permanet quam ipsa illa effectrix beatae
  vitae sapientia neque expectat ultimum tempus aetatis, quod Croeso scribit
  Herodotus praeceptum a Solone. At enim, quem ad modum tute dicebas, negat
  Epicurus diuturnitatem quidem temporis ad beate vivendum aliquid afferre, nec
  minorem voluptatem percipi in brevitate temporis, quam si illa sit
  sempiterna.}
 %{168}
\NewLipsumPar{Haec dicuntur inconstantissime. Cum enim summum bonum in
  voluptate ponat, negat infinito tempore aetatis voluptatem fieri maiorem quam
  finito atque modico. Qui bonum omne in virtute ponit, is potest dicere perfici
  beatam vitam perfectione virtutis; negat enim summo bono afferre incrementum
  diem. Qui autem voluptate vitam effici beatam putabit, qui sibi is conveniet,
  si negabit voluptatem crescere longinquitate? Igitur ne dolorem quidem. An
  dolor longissimus quisque miserrimus, voluptatem non optabiliorem diuturnitas
  facit? Quid est igitur, cur ita semper deum appellet Epicurus beatum et
  aeternum? Dempta enim aeternitate nihilo beatior Iuppiter quam Epicurus;
  uterque enim summo bono fruitur, id est voluptate. \textquoteleft At enim hic
  etiam dolore.\textquoteright\space At eum nihili facit; ait enim se, si
  uratur, \textquoteleft Quam hoc suave!\textquoteright\space dicturum.}
 %{169}
\NewLipsumPar{Qua igitur re ab deo vincitur, si aeternitate non vincitur? In
  qua quid est boni praeter summam voluptatem, et eam sempiternam? Quid ergo
  attinet gloriose loqui, nisi constanter loquare? In voluptate
  corporis\textendash addam, si vis, \textquoteleft animi\textquoteright, dum ea
  ipsa, ut vultis, sit e corpore\textendash situm est vivere beate. Quid? Istam
  voluptatem perpetuam quis potest praestare sapienti? Nam quibus rebus
  efficiuntur voluptates, eae non sunt in potestate sapientis. Non enim in ipsa
  sapientia positum est beatum esse, sed in iis rebus, quas sapientia comparat
  ad voluptatem. Totum autem id externum est, et quod externum, id in casu est.
  Ita fit beatae vitae domina fortuna, quam Epicurus ait exiguam intervenire
  sapienti.}
 %{170}
\NewLipsumPar{Age, inquies, ista parva sunt. Sapientem locupletat ipsa natura,
  cuius divitias Epicurus parabiles esse docuit. Haec bene dicuntur, nec ego
  repugno, sed inter sese ipsa pugnant. Negat enim tenuissimo victu, id est
  contemptissimis escis et potionibus, minorem voluptatem percipi quam rebus
  exquisitissimis ad epulandum. Huic ego, si negaret quicquam interesse ad beate
  vivendum quali uteretur victu, concederem, laudarem etiam; verum enim diceret,
  idque Socratem, qui voluptatem nullo loco numerat, audio dicentem, cibi
  condimentum esse famem, potionis sitim. Sed qui ad voluptatem omnia referens
  vivit ut Gallonius, loquitur ut Frugi ille Piso, non audio nec eum, quod
  sentiat, dicere existimo.}
 %{171}
\NewLipsumPar{Naturales divitias dixit parabiles esse, quod parvo esset natura
  contenta. Certe, nisi voluptatem tanti aestimaretis. Non minor, inquit,
  voluptas percipitur ex vilissimis rebus quam ex pretiosissimis. Hoc est non
  modo cor non habere, sed ne palatum quidem. Qui enim voluptatem ipsam
  contemnunt, iis licet dicere se acupenserem maenae non anteponere. Cui vero in
  voluptate summum bonum est, huic omnia sensu, non ratione sunt iudicanda,
  eaque dicenda optima, quae sint suavissima.}
 %{172}
\NewLipsumPar{Verum esto; consequatur summas voluptates non modo parvo, sed
  per me nihilo, si potest; sit voluptas non minor in nasturcio illo, quo vesci
  Persas esse solitos scribit Xenophon, quam in Syracusanis mensis, quae a
  Platone graviter vituperantur; sit, inquam, tam facilis, quam vultis,
  comparatio voluptatis, quid de dolore dicemus? Cuius tanta tormenta sunt, ut
  in iis beata vita, si modo dolor summum malum est, esse non possit. Ipse enim
  Metrodorus, paene alter Epicurus, beatum esse describit his fere verbis:
  \textquoteleft cum corpus bene constitutum sit et sit exploratum ita
  futurum.\textquoteright\space an id exploratum cuiquam potest esse, quo modo
  se hoc habiturum sit corpus, non dico ad annum, sed ad vesperum? Dolor ergo,
  id est summum malum, metuetur semper, etiamsi non aderit; iam enim adesse
  poterit. Qui potest igitur habitare in beata vita summi mali metus?}
 %{173}
\NewLipsumPar{Traditur, inquit, ab Epicuro ratio neglegendi doloris. Iam id
  ipsum absurdum, maximum malum neglegi. Sed quae tandem ista ratio est? Maximus
  dolor, inquit, brevis est. Primum quid tu dicis breve? Deinde dolorem quem
  maximum? Quid enim? summus dolor plures dies manere non potest? Vide, ne etiam
  menses! Nisi forte eum dicis, qui, simul atque arripuit, interficit. Quis
  istum dolorem timet? illum mallem levares, quo optimum atque humanissimum
  virum, Cn.\space Octavium, Marci filium, familiarem meum, confici vidi, nec
  vero semel nec ad breve tempus, sed et saepe et plane diu. Quos ille, di
  inmortales, cum omnes artus ardere viderentur, cruciatus perferebat! Nec tamen
  miser esse, quia summum id malum non erat, tantum modo laboriosus videbatur;
  at miser, si in flagitiosa et vitiosa vita afflueret voluptatibus.}
 %{174}
\NewLipsumPar{Quod autem magnum dolorem brevem, longinquum levem esse dicitis,
  id non intellego quale sit. Video enim et magnos et eosdem bene longinquos
  dolores, quorum alia toleratio est verior, qua uti vos non potestis, qui
  honestatem ipsam per se non amatis. Fortitudinis quaedam praecepta sunt ac
  paene leges, quae effeminari virum vetant in dolore. Quam ob rem turpe
  putandum est, non dico dolere\textendash nam id quidem est interdum
  necesse\textendash, sed saxum illud Lemnium clamore Philocteteo funestare,
  Quod eiulatu, questu, gemitu, fremitibus Resonando mutum flebiles voces
  refert. Huic Epicurus praecentet, si potest, cui (e) viperino morsu venae
  viscerum Veneno inbutae taetros cruciatus cient! Sic Epicurus: \textquoteleft
  Philocteta, st! Brevis dolor.\textquoteright\space At iam decimum annum in
  spelunca iacet. \textquoteleft Si longus, levis; dat enim intervalla et
  relaxat.\textquoteright}
 %{175}
\NewLipsumPar{Primum non saepe, deinde quae est ista relaxatio, cum et
  praeteriti doloris memoria recens est et futuri atque inpendentis torquet
  timor? \textquoteleft Moriatur\textquoteright, inquit. Fortasse id optimum,
  sed ubi illud: \textquoteleft Plus semper voluptatis\textquoteright? Si enim
  ita est, vide ne facinus facias, cum mori suadeas. Potius ergo illa dicantur:
  turpe esse, viri non esse debilitari dolore, frangi, succumbere. Nam ista
  vestra: \textquoteleft Si gravis, brevis; si longus,
  levis\textquoteright\space dictata sunt. Virtutis, magnitudinis animi,
  patientiae, fortitudinis fomentis dolor mitigari solet.}
 %{176}
\NewLipsumPar{Audi, ne longe abeam, moriens quid dicat Epicurus, ut intellegas
  facta eius cum dictis discrepare: \textquoteleft Epicurus Hermarcho salutem.
  Cum ageremus\textquoteright, inquit, \textquoteleft vitae beatum et eundem
  supremum diem, scribebamus haec. Tanti autem aderant vesicae et torminum
  morbi, ut nihil ad eorum magnitudinem posset accedere.\textquoteright\space
  Miserum hominem! Si dolor summum malum est, dici aliter non potest. Sed
  audiamus ipsum: \textquoteleft Compensabatur\textquoteright, inquit,
  \textquoteleft tamen cum his omnibus animi laetitia, quam capiebam memoria
  rationum inventorumque nostrorum. Sed tu, ut dignum est tua erga me et
  philosophiam voluntate ab adolescentulo suscepta, fac ut Metrodori tueare
  liberos.\textquoteright}
 %{177}
\NewLipsumPar{Non ego iam Epaminondae, non Leonidae mortem huius morti
  antepono, quorum alter cum vicisset Lacedaemonios apud Mantineam atque ipse
  gravi vulnere exanimari se videret, ut primum dispexit, quaesivit salvusne
  esset clipeus. Cum salvum esse flentes sui respondissent, rogavit essentne
  fusi hostes. Cum id quoque, ut cupiebat, audivisset, evelli iussit eam, qua
  erat transfixus, hastam. Ita multo sanguine profuso in laetitia et in victoria
  est mortuus. Leonidas autem, rex Lacedaemoniorum, se in Thermopylis
  trecentosque eos, quos eduxerat Sparta, cum esset proposita aut fuga turpis
  aut gloriosa mors, opposuit hostibus. Praeclarae mortes sunt imperatoriae;
  philosophi autem in suis lectulis plerumque moriuntur. Refert tamen, quo modo.
  (beatus) sibi videtur esse moriens. Magna laus. \textquoteleft
  Compensabatur\textquoteright, inquit, \textquoteleft cum summis doloribus
  laetitia.\textquoteright}
 %{178}
\NewLipsumPar{Audio equidem philosophi vocem, Epicure, sed quid tibi dicendum
  sit oblitus es. Primum enim, si vera sunt ea, quorum recordatione te gaudere
  dicis, hoc est, si vera sunt tua scripta et inventa, gaudere non potes. Nihil
  enim iam habes, quod ad corpus referas; est autem a te semper dictum nec
  gaudere quemquam nisi propter corpus nec dolere. \textquoteleft
  Praeteritis\textquoteright, inquit, \textquoteleft
  gaudeo.\textquoteright\space Quibusnam praeteritis? Si ad corpus
  pertinentibus, rationes tuas te video compensare cum istis doloribus, non
  memoriam corpore perceptarum voluptatum; sin autem ad animum, falsum est, quod
  negas animi ullum esse gaudium, quod non referatur ad corpus. Cur deinde
  Metrodori liberos commendas? Quid (in) isto egregio tuo officio et tanta
  fide\textendash sic enim existimo\textendash ad corpus refers?}
 %{179}
\NewLipsumPar{Huc et illuc, Torquate, vos versetis licet, nihil in hac
  praeclara epistula scriptum ab Epicuro congruens et conveniens decretis eius
  reperietis. Ita redarguitur ipse a sese, convincunturque scripta eius
  probitate ipsius ac moribus. Nam ista commendatio puerorum, memoria et caritas
  amicitiae, summorum officiorum in extremo spiritu conservatio indicat innatam
  esse homini probitatem gratuitam, non invitatam voluptatibus nec praemiorum
  mercedibus evocatam. Quod enim testimonium maius quaerimus, quae honesta et
  recta sint, ipsa esse optabilia per sese, cum videamus tanta officia
  morientis?}
 %{180}
\NewLipsumPar{Sed ut epistulam laudandam arbitror eam, quam modo totidem fere
  verbis interpretatus sum, quamquam ea cum summa eius philosophia nullo modo
  congruebat, sic eiusdem testamentum non solum (a) philosophi gravitate, sed
  etiam ab ipsius sententia iudico discrepare. Scripsit enim et multis saepe
  verbis et breviter arteque in eo libro, quem modo nominavi, mortem nihil ad
  nos pertinere. Quod enim dissolutum sit, id esse sine sensu, quod autem sine
  sensu sit, id nihil ad nos pertinere omnino. Hoc ipsum elegantius poni
  meliusque potuit. Nam quod ita positum est, quod dissolutum sit, id esse sine
  sensu, id eius modi est, ut non satis plane dicat quid sit dissolutum.}
 %{181}
\NewLipsumPar{Sed tamen intellego quid velit. Quaero autem quid sit, quod, cum
  dissolutione, id est morte, sensus omnis extinguatur, et cum reliqui nihil sit
  omnino, quod pertineat ad nos, tam accurate tamque diligenter caveat et
  sanciat ut Amynomachus et Timocrates, heredes sui, de Hermarchi sententia dent
  quod satis sit ad diem agendum natalem suum quotannis mense Gamelione itemque
  omnibus mensibus vicesimo die lunae dent ad eorum epulas, qui una secum
  philosophati sint, ut et sui et Metrodori memoria colatur.}
 %{182}
\NewLipsumPar{Haec ego non possum dicere non esse hominis quamvis et belli et
  humani, sapientis vero nullo modo, physici praesertim, quem se ille esse vult,
  putare ullum esse cuiusquam diem natalem. Quid? Idemne potest esse dies
  saepius, qui semel fuit? Certe non potest. An eiusdem modi? Ne id quidem, nisi
  multa annorum intercesserint milia, ut omnium siderum eodem, unde profecta
  sint, fiat ad unum tempus reversio. Nullus est igitur cuiusquam dies natalis.
  \textquoteleft At habetur!\textquoteright\space Et ego id scilicet nesciebam!
  Sed ut sit, etiamne post mortem coletur? Idque testamento cavebit is, qui
  nobis quasi oraculum ediderit nihil post mortem ad nos pertinere? Haec non
  erant eius, qui innumerabilis mundos infinitasque regiones, quarum nulla esset
  ora, nulla extremitas, mente peragravisset. Num quid tale Democritus? Ut alios
  omittam, hunc appello, quem ille unum secutus est.}
 %{183}
\NewLipsumPar{Quodsi dies notandus fuit, eumne potius, quo natus, an eum, quo
  sapiens factus est? Non potuit, inquies, fieri sapiens, nisi natus esset. [et]
  Isto modo, ne si avia quidem eius nata non esset. Res tota, Torquate, non
  doctorum hominum, velle post mortem epulis celebrari memoriam sui nominis.
  Quos quidem dies quem ad modum agatis et in quantam hominum facetorum
  urbanitatem incurratis, non dico\textendash\space nihil opus est
  litibus\textendash; tantum dico, magis fuisse vestrum agere Epicuri diem
  natalem, quam illius testamento cavere ut ageretur.}
 %{184}
\NewLipsumPar{Sed ut ad propositum\textendash de dolore enim cum diceremus, ad
  istam epistulam delati sumus\textendash, nunc totum illud concludi sic licet:
  qui in summo malo est, is tum, cum in eo est, non est beatus; sapiens autem
  semper beatus est et est aliquando in dolore; non est igitur summum malum
  dolor. Iam illud quale tandem est, bona praeterita non effluere sapienti, mala
  meminisse non oportere? Primum in nostrane potestate est, quid meminerimus?
  Themistocles quidem, cum ei Simonides an quis alius artem memoriae
  polliceretur, \textquoteleft Oblivionis\textquoteright, inquit, \textquoteleft
  mallem. Nam memini etiam quae nolo, oblivisci non possum quae
  volo.\textquoteright}
 %{185}
\NewLipsumPar{Magno hic ingenio, sed res se tamen sic habet, ut nimis
  imperiosi philosophi sit vetare meminisse. Vide ne ista sint Manliana vestra
  aut maiora etiam, si imperes quod facere non possim. Quid, si etiam iucunda
  memoria est praeteritorum malorum? Ut proverbia non nulla veriora sint quam
  vestra dogmata. Vulgo enim dicitur: \textquoteleft Iucundi acti
  labores\textquoteright, nec male Euripides\textendash\space concludam, si
  potero, Latine; Graecum enim hunc versum nostis omnes\textendash:
  \textquoteleft Suavis laborum est praeteritorum memoria.\textquoteright\space
  Sed ad bona praeterita redeamus. Quae si a vobis talia dicerentur, qualibus
  Caius Marius uti poterat, ut expulsus, egens, in palude demersus tropaeorum
  recordatione levaret dolorem suum, audirem et plane probarem. Nec enim absolvi
  beata vita sapientis neque ad exitum perduci poterit, si prima quaeque bene ab
  eo consulta atque facta ipsius oblivione obruentur.}
 %{186}
\NewLipsumPar{Sed vobis voluptatum perceptarum recordatio vitam beatam facit,
  et quidem corpore perceptarum. Nam si quae sunt aliae, falsum est omnis animi
  voluptates esse e corporis societate. Corporis autem voluptas si etiam
  praeterita delectat, non intellego, cur Aristoteles Sardanapalli epigramma
  tantopere derideat, in quo ille rex Syriae glorietur se omnis secum libidinum
  voluptates abstulisse. Quod enim ne vivus quidem, inquit, diutius sentire
  poterat, quam dum fruebatur, quo modo id potuit mortuo permanere? Effluit
  igitur voluptas corporis et prima quaeque avolat saepiusque relinquit causam
  paenitendi quam recordandi. Itaque beatior Africanus cum patria illo modo
  loquens: \textquoteleft Desine, Roma, tuos hostes\textquoteright\space
  reliquaque praeclare: \textquoteleft Nam tibi moenimenta mei peperere
  labores.\textquoteright\space Laboribus hic praeteritis gaudet, tu iubes
  voluptatibus, et hic se ad ea revocat, e quibus nihil umquam rettulerit ad
  corpus, tu totus haeres in corpore.}
 %{187}
\NewLipsumPar{Illud autem ipsum qui optineri potest, quod dicitis, omnis animi
  et voluptates et dolores ad corporis voluptates ac dolores pertinere? Nihilne
  te delectat umquam \textendash video, quicum loquar\textendash, te igitur,
  Torquate, ipsum per se nihil delectat? Omitto dignitatem, honestatem, speciem
  ipsam virtutum, de quibus ante dictum est, haec leviora ponam: poema,
  orationem cum aut scribis aut legis, cum omnium factorum, cum regionum
  conquiris historiam, signum, tabula, locus amoenus, ludi, venatio, villa
  Luculli\textendash nam si \textquoteleft tuam\textquoteright\space dicerem,
  latebram haberes; ad corpus diceres pertinere\textendash, sed ea, quae dixi,
  ad corpusne refers? An est aliquid, quod te sua sponte delectet? Aut
  pertinacissimus fueris, si in eo perstiteris ad corpus ea, quae dixi, referri,
  aut deserueris totam Epicuri voluptatem, si negaveris.}
 %{188}
\NewLipsumPar{Quod vero a te disputatum est maiores esse voluptates et dolores
  animi quam corporis, quia trium temporum particeps animus sit, corpore autem
  praesentia solum sentiantur, qui id probari potest, ut is, qui propter me
  aliquid gaudeat, plus quam ego ipse gaudeat? [animo voluptas oritur propter
  voluptatem corporis, et maior est animi voluptas quam corporis. Ita fit, ut
  gratulator laetior sit quam is, cui gratulatur.] Sed dum efficere vultis
  beatum sapientem, cum maximas animo voluptates percipiat omnibusque partibus
  maiores quam corpore, quid occurrat non videtis. Animi enim quoque dolores
  percipiet omnibus partibus maiores quam corporis. Ita miser sit aliquando
  necesse est is, quem vos beatum semper vultis esse, nec vero id, dum omnia ad
  voluptatem doloremque referetis, efficietis umquam.}
 %{189}
\NewLipsumPar{Quare aliud aliquod, Torquate, hominis summum bonum reperiendum
  est, voluptatem bestiis concedamus, quibus vos de summo bono testibus uti
  soletis. Quid, si etiam bestiae multa faciunt duce sua quaeque natura partim
  indulgenter vel cum labore, ut in gignendo, in educando, perfacile appareat
  aliud quiddam iis propositum, non voluptatem? Partim cursu et peragratione
  laetantur, congregatione aliae coetum quodam modo civitatis imitantur;}
 %{190}
\NewLipsumPar{Videmus in quodam volucrium genere non nulla indicia pietatis,
  cognitionem, memoriam, in multis etiam desideria videmus. Ergo in bestiis
  erunt secreta e voluptate humanarum quaedam simulacra virtutum, in ipsis
  hominibus virtus nisi voluptatis causa nulla erit? Et homini, qui ceteris
  animantibus plurimum praestat, praecipue a natura nihil datum esse dicemus?}
 %{191}
\NewLipsumPar{Nos vero, siquidem in voluptate sunt omnia, longe multumque
  superamur a bestiis, quibus ipsa terra fundit ex sese pastus varios atque
  abundantes nihil laborantibus, nobis autem aut vix aut ne vix quidem suppetunt
  multo labore quaerentibus. Nec tamen ullo modo summum pecudis bonum et hominis
  idem mihi videri potest. Quid enim tanto opus est instrumento in optimis
  artibus comparandis? Quid tanto concursu honestissimorum studiorum, tanto
  virtutum comitatu, si ea nullam ad aliam rem nisi ad voluptatem conquiruntur?}
 %{192}
\NewLipsumPar{Ut, si Xerxes, cum tantis classibus tantisque equestribus et
  pedestribus copiis Hellesponto iuncto Athone perfosso mari ambulavisset terra
  navigavisset, si, cum tanto impetu in Graeciam venisset, causam quis ex eo
  quaereret tantarum copiarum tantique belli, mel se auferre ex Hymetto voluisse
  diceret, certe sine causa videretur tanta conatus, sic nos sapientem plurimis
  et gravissimis artibus atque virtutibus instructum et ornatum non, ut illum,
  maria pedibus peragrantem, classibus montes, sed omne caelum totamque cum
  universo mari terram mente complexum voluptatem petere si dicemus, mellis
  causa dicemus tanta molitum.}
 %{193}
\NewLipsumPar{Ad altiora quaedam et magnificentiora, mihi crede, Torquate,
  nati sumus, nec id ex animi solum partibus, in quibus inest memoria rerum
  innumerabilium, in te quidem infinita, inest coniectura consequentium non
  multum a divinatione differens, inest moderator cupiditatis pudor, inest ad
  humanam societatem iustitiae fida custodia, inest in perpetiendis laboribus
  adeundisque periculis firma et stabilis doloris mortisque
  contemptio\textendash ergo haec in animis, tu autem etiam membra ipsa
  sensusque considera, qui tibi, ut reliquae corporis partes, non comites solum
  virtutum, sed ministri etiam videbuntur.}
 %{194}
\NewLipsumPar{Quid? Si in ipso corpore multa voluptati praeponenda sunt, ut
  vires, valitudo, velocitas, pulchritudo, quid tandem in animis censes? In
  quibus doctissimi illi veteres inesse quiddam caeleste et divinum putaverunt.
  Quodsi esset in voluptate summum bonum, ut dicitis, optabile esset maxima in
  voluptate nullo intervallo interiecto dies noctesque versari, cum omnes sensus
  dulcedine omni quasi perfusi moverentur. Quis est autem dignus nomine hominis,
  qui unum diem totum velit esse in genere isto voluptatis? Cyrenaici quidem non
  recusant; vestri haec verecundius, illi fortasse constantius.}
 %{195}
\NewLipsumPar{Sed lustremus animo non has maximas artis, quibus qui carebant
  inertes a maioribus nominabantur, sed quaero num existimes, non dico Homerum,
  Archilochum, Pindarum, sed Phidian, Polyclitum, Zeuxim ad voluptatem artes
  suas direxisse. Ergo opifex plus sibi proponet ad formarum quam civis
  excellens ad factorum pulchritudinem? Quae autem est alia causa erroris tanti
  tam longe lateque diffusi, nisi quod is, qui voluptatem summum bonum esse
  decernit, non cum ea parte animi, qua inest ratio atque consilium, sed cum
  cupiditate, id est cum animi levissima parte, deliberat? Quaero enim de te, si
  sunt di, ut vos etiam putatis, qui possint esse beati, cum voluptates corpore
  percipere non possint, aut, si sine eo genere voluptatis beati sint, cur
  similem animi usum in sapiente esse nolitis.}
 %{196}
\NewLipsumPar{Lege laudationes, Torquate, non eorum, qui sunt ab Homero
  laudati, non Cyri, non Agesilai, non Aristidi aut Themistocli, non Philippi
  aut Alexandri, lege nostrorum hominum, lege vestrae familiae; neminem videbis
  ita laudatum, ut artifex callidus comparandarum voluptatum diceretur. Non
  elogia monimentorum id significant, velut hoc ad portam: \textquoteleft Hunc
  unum plurimae consentiunt gentes populi primarium fuisse
  virum.\textquoteright}
 %{197}
\NewLipsumPar{Idne consensisse de Calatino plurimas gentis arbitramur,
  primarium populi fuisse, quod praestantissimus fuisset in conficiendis
  voluptatibus? Ergo in iis adolescentibus bonam spem esse dicemus et magnam
  indolem, quos suis commodis inservituros et quicquid ipsis expediat facturos
  arbitrabimur? Nonne videmus quanta perturbatio rerum omnium consequatur,
  quanta confusio? Tollitur beneficium, tollitur gratia, quae sunt vincla
  concordiae. Nec enim, cum tua causa cui commodes, beneficium illud habendum
  est, sed faeneratio, nec gratia deberi videtur ei, qui sua causa commodaverit.
  Maximas vero virtutes iacere omnis necesse est voluptate dominante. Sunt etiam
  turpitudines plurimae, quae, nisi honestas natura plurimum valeat, cur non
  cadant in sapientem non est facile defendere.}
 %{198}
\NewLipsumPar{Ac ne plura complectar\textendash sunt enim
  innumerabilia\textendash, bene laudata virtus voluptatis aditus intercludat
  necesse est. Quod iam a me expectare noli. Tute introspice in mentem tuam ipse
  eamque omni cogitatione pertractans percontare ipse te perpetuisne malis
  voluptatibus perfruens in ea, quam saepe usurpabas, tranquillitate degere
  omnem aetatem sine dolore, adsumpto etiam illo, quod vos quidem adiungere
  soletis, sed fieri non potest, sine doloris metu, an, cum de omnibus gentibus
  optime mererere, cum opem indigentibus salutemque ferres, vel Herculis perpeti
  aerumnas. Sic enim maiores nostri labores non fugiendos tristissimo tamen
  verbo aerumnas etiam in deo nominaverunt.}
 %{199}
\NewLipsumPar{Elicerem ex te cogeremque, ut responderes, nisi vererer ne
  Herculem ipsum ea, quae pro salute gentium summo labore gessisset, voluptatis
  causa gessisse diceres. Quae cum dixissem, Habeo, inquit Torquatus, ad quos
  ista referam, et, quamquam aliquid ipse poteram, tamen invenire malo
  paratiores. Familiares nostros, credo, Sironem dicis et Philodemum, cum
  optimos viros, tum homines doctissimos. Recte, inquit, intellegis. Age sane,
  inquam. Sed erat aequius Triarium aliquid de dissensione nostra iudicare.
  Eiuro, inquit adridens, iniquum, hac quidem de re; tu enim ista lenius, hic
  Stoicorum more nos vexat. Tum Triarius: Posthac quidem, inquit, audacius. Nam
  haec ipsa mihi erunt in promptu, quae modo audivi, nec ante aggrediar, quam te
  ab istis, quos dicis, instructum videro. Quae cum essent dicta, finem fecimus
  et ambulandi et disputandi.}
 %{200}
\NewLipsumPar{Voluptatem quidem, Brute, si ipsa pro se loquatur nec tam
  pertinaces habeat patronos, concessuram arbitror convictam superiore libro
  dignitati. Etenim sit inpudens, si virtuti diutius repugnet, aut si honestis
  iucunda anteponat aut pluris esse contendat dulcedinem corporis ex eave natam
  laetitiam quam gravitatem animi atque constantiam. Quare illam quidem
  dimittamus et suis se finibus tenere iubeamus, ne blanditiis eius
  inlecebrisque impediatur disputandi severitas.}
%% 
%%
%% End of file `cicero.ltd.tex'.
